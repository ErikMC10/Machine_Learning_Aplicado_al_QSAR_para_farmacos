\section{Pruebas de Sistema}{
\noindent Las pruebas de sistema tienen la función de verificar de manera profunda el sistema, para comprobar  la integración del sistema de información de manera general, para corroborar el adecuado funcionamiento de las interfaces y comunicación entre los distintos módulos que son parte del sistema.\\

\noindent Estas pruebas se consideran de integracióńn del sistema de información completo, y se puede probar el sistema en su conjunto para verificar que las especificaciones funcionales y técnicas se cumplen. Dan una visión muy similar a su comportamiento en el entorno de producción.\\

\noindent Las pruebas de sistema se componen de una serie de examinaciones de diversos aspectos para asegurar que el sistema de información realiza correctamente todas las funciones que se han detallado en los requerimientos.

\begin{itemize}
    \item Pruebas de comunicaciones: Se comprueba  que las interfaces entre los módulos del  del sistema funcionan adecuadamente.
    \item Pruebas de rendimiento. Consisten en determinar los tiempos de respuesta sean aptos y/o aceptables, permitiendo un buen funcionamiento, que no se vea afectado por tiempos de funcionamiento.
    \item Pruebas de volumen. Para este aspecto, son pruebas que se realizan para revisar la capacidad del sistema ante la presencia de una gran cantidad de datos, en este caso compuestos y evaluar su comportamiento.
    \item Pruebas de disponibilidad de datos. Consisten en demostrar que el sistema de información  puede recuperarse ante fallos, tanto físicos del equipo como lógicos, sin comprometer la integridad de los datos.
    \item Pruebas de facilidad de uso. Para estas pruebas se debe revisar  la adaptabilidad del sistema a las necesidades del usuario, así asegurar que se acomoda con el uso del sistema, además se pueden determinar las facilidades  al introducir datos y obtener los resultados.
\end{itemize}
\newpage
\subsection{Pruebas de Comunicaciones}
\begin{longtable}{|l|l|}
\caption{Pruebas de sistema.}\\ 
\hline
\begin{tabular}[c]{@{}l@{}}\textbf{Pruebas de~ Comunicaciones}\\\textbf{~del Sistema. }\end{tabular} & \begin{tabular}[c]{@{}l@{}}\textbf{SisPAF }(Sistema para la Predicción\\~de Actividad Farmacológica\textbf{)}\end{tabular}  \endfirsthead 
\hline
\textbf{Prueba de~Comunicación}                                                                      & \textbf{PCN1}                                                                                                               \\* 
\hline
\multirow{3}{*}{\textbf{Criterios evaluados.} }                                                      & \textbf{Módulo: Búsqueda de Información.}                                                                                   \\* 
\cline{2-2}
                                                                                                     & \textbf{Módulo: Procesamiento de la Información.}                                                                           \\* 
\cline{2-2}
                                                                                                     & \textbf{Módulo: Ordenamiento de Resultados. }                                                                               \\
\hline
\end{longtable}


\begin{longtable}{|l|l|} 
\caption{CPC2: Módulo 1- Módulo 2.}\\ 
\hline
\begin{tabular}[c]{@{}l@{}}\textbf{Caso de Prueba para}\\\textbf{comunicaciones.}\end{tabular}                                                                            & \textbf{CPC1: Módulo 1- Módulo 2.}                                                                                                                                              \endfirsthead 
\hline
\begin{tabular}[c]{@{}l@{}}\textbf{Muestra de Resultados}\\\textbf{de búsqueda de proteínas.}\end{tabular}                                                                &                                                                                                \\ 
\hline
\begin{tabular}[c]{@{}l@{}}\textbf{Muestra de Resultados de }\\\textbf{búsqueda de compuestos.}\end{tabular}                                                              &                                                                                                \\ 
\hline
\begin{tabular}[c]{@{}l@{}}\textbf{Adquisición de la }\\\textbf{estructura PDB~ de la }\\\textbf{proteína a partir del }\\\textbf{archivo “ConjuntoP0”}\end{tabular}      &                                                                                                \\ 
\hline
\begin{tabular}[c]{@{}l@{}}\textbf{Adquisición de la estructura }\\\textbf{PDB del compuesto a partir }\\\textbf{del archivo “ConjuntoC0”}\end{tabular}                   &         \\ 
\hline
\begin{tabular}[c]{@{}l@{}}\textbf{Adquisición de los }\\\textbf{descriptores~ del }\\\textbf{compuesto a partir }\\\textbf{del archivo “ConjuntoC0”}\end{tabular}        &                                    \\ 
\hline
\begin{tabular}[c]{@{}l@{}}\textbf{Adquisición de la actividad }\\\textbf{biológica ~ del compuesto }\\\textbf{a partir del archivo }\\\textbf{“ConjuntoC0”}\end{tabular} &  \\
\hline
\end{longtable}

\begin{longtable}{|l|l|}
\caption{CPC2: Módulo 2- Módulo 3.}\\ 
\hline
\begin{tabular}[c]{@{}l@{}}\textbf{Caso de Prueba para}\\\textbf{comunicaciones.}\end{tabular}                                                        & \textbf{CPC2: Módulo 2- Módulo 3.}                                                                                                                                                                             \endfirsthead 
\hline
\begin{tabular}[c]{@{}l@{}}\textbf{Lectura de los resultados}\\\textbf{~obtenidos en el \textit{Docking}.}\end{tabular}                               &                                                                                                                               \\ 
\hline
\begin{tabular}[c]{@{}l@{}}\textbf{Lectura de los resultados}\\\textbf{generados en la Regresión }\\\textbf{Lineal. }\end{tabular}                    &   \\ 
\hline
\begin{tabular}[c]{@{}l@{}}\textbf{Adquisición del listado de}\\\textbf{resultados, ordenando los}\\\textbf{compuestos por efectividad.}\end{tabular} &                                                                              \\ 
\hline
\begin{tabular}[c]{@{}l@{}}\textbf{Adecuado almacenamiento}\\\textbf{de los resultados en un }\\\textbf{archivo de texto. }\end{tabular}              &                                                                                                  \\
\hline
\end{longtable}
\subsection{Pruebas de Rendimiento}
\begin{longtable}{|l|l|}
\caption{Pruebas de sistema.}\\ 
\hline
\begin{tabular}[c]{@{}l@{}}\textbf{Pruebas de~ Comunicaciones}\\\textbf{~del Sistema. }\end{tabular} & \begin{tabular}[c]{@{}l@{}}\textbf{SisPAF }(Sistema para la Predicción\\~de Actividad Farmacológica\textbf{)}\end{tabular}  \endfirsthead 
\hline
\textbf{Prueba de~Comunicación}                                                                      & \textbf{PCN1}                                                                                                               \\* 
\hline
\multirow{3}{*}{\textbf{Criterios evaluados.} }                                                      & \textbf{Módulo: Búsqueda de Información.}                                                                                   \\* 
\cline{2-2}
                                                                                                     & \textbf{Módulo: Procesamiento de la Información.}                                                                           \\* 
\cline{2-2}
                                                                                                     & \textbf{Módulo: Ordenamiento de Resultados. }                                                                               \\
\hline
\end{longtable}


\begin{longtable}{|l|l|}
\caption{Pruebas de sistema (Rendimiento del sistema).}\\ 
\hline
\begin{tabular}[c]{@{}l@{}}\textbf{Casos de Prueba de }\\\textbf{Rendimiento del Sistema.}\end{tabular} & \textbf{CPRS 1}                                                                                                                                                                                                                                                                                                                                                                                                                                                          \endfirsthead 
\hline
\textbf{Máquina usada:}                                                                                 & \begin{tabular}[c]{@{}l@{}}\begin{tabular}{@{\labelitemi\hspace{\dimexpr\labelsep+0.5\tabcolsep}}l}\begin{tabular}[c]{@{}l@{}}Procesador Intel Core i7-649DU\\(2,5 GHz, hasta 3,1 GHz, 4 MB\\~de caché, 4 núcleos).\end{tabular}\\Memoria RAM de 8 GB .\\\begin{tabular}[c]{@{}l@{}}Disco Duro Toshiba MQ01 de\\~1 TB.\end{tabular}\\\begin{tabular}[c]{@{}l@{}}Disco Duro Kingston SSD de 240\\GB.Gráficos de video Intel HD510\end{tabular}\end{tabular}\end{tabular}  \\ 
\hline
\textbf{Componente crítico evaluado:}                                                                   & \begin{tabular}[c]{@{}l@{}}\textbf{Búsqueda de Información}\\\textbf{}\end{tabular}                                                                                                                                                                                                                                                                                                                                                                                      \\ 
\hline
\textbf{Evidencia: }                                                                                    &                                                                                                                                                                                                                                                                                                                                                                                                                                                                          \\ 
\hline
\textbf{Componente crítico evaluado:}                                                                   & \begin{tabular}[c]{@{}l@{}}\textbf{Muestra de resultados de la }\\\textbf{búsqueda.}\end{tabular}                                                                                                                                                                                                                                                                                                                                                                        \\ 
\hline
\textbf{Evidencia:}                                                                                     &                                                                                                                                                                                                                                                                                                                                                                                                                                                                          \\ 
\hline
\textbf{Componente crítico evaluado:}                                                                   & \begin{tabular}[c]{@{}l@{}}\textbf{Análisis de la información}\\\textbf{(Docking-Regresión Lineal).}\end{tabular}                                                                                                                                                                                                                                                                                                                                                        \\ 
\hline
\textbf{Evidencia:}                                                                                     &                                                                                                                                                                                                                                                                                                                                                                                                                                                                          \\ 
\hline
\textbf{Componente crítico evaluado:}                                                                   & \begin{tabular}[c]{@{}l@{}}\textbf{Muestra de resultados}\\\textbf{(Ordenamiento).}\end{tabular}                                                                                                                                                                                                                                                                                                                                                                         \\ 
\hline
\textbf{Evidencia:}                                                                                     &                                                                                                                                                                                                                                                                                                                                                                                                                                                                          \\
\hline
\end{longtable}

\begin{longtable}{|l|l|}
\caption{Pruebas de sistema CPRS2 (Rendimiento del sistema).}\\ 
\hline
\begin{tabular}[c]{@{}l@{}}\textbf{Casos de Prueba de }\\\textbf{Rendimiento del Sistema.}\end{tabular} & \textbf{CPRS 2}                                                                                                                                                                                                                                                                                                                                                                                                                                                                                                                                                                                                                                                                                                                                                                                                                                                                                                                                                                                                                                                                                                                                                                                                                                                                                     \endfirsthead 
\hline
\textbf{Máquina usada:}                                                                                 & \begin{tabular}[c]{@{}l@{}}\begin{tabular}{@{\labelitemi\hspace{\dimexpr\labelsep+0.5\tabcolsep}}l}\begin{tabular}[c]{@{}l@{}}\textbf{Características:~}\\\begin{tabular}{@{\labelitemi\hspace{\dimexpr\labelsep+0.5\tabcolsep}}l}\begin{tabular}[c]{@{}l@{}}Procesador AMD Ryzen 3 con\\~gráficos Radeon Vega Graphics\\~Mobile Gfx (\textcolor[rgb]{0.133,0.133,0.133}{Cuatro núcleos a }\textcolor[rgb]{0.133,0.133,0.133}{3}\textcolor[rgb]{0.133,0.133,0.133}{,4}\\\textcolor[rgb]{0.133,0.133,0.133}{GHz en modo base y }\textcolor[rgb]{0.133,0.133,0.133}{3}\textcolor[rgb]{0.133,0.133,0.133}{,7 GHz en}\\\textcolor[rgb]{0.133,0.133,0.133}{modo turbo).}\end{tabular}\\Memoria RAM de 12 GB.\\Disco duro SATA3 de 500 GB SSD\end{tabular}\end{tabular}\\\begin{tabular}[c]{@{}l@{}}Procesador AMD Ryzen 3 con\\~gráficos Radeon Vega Graphics\\~Mobile Gfx (\textcolor[rgb]{0.133,0.133,0.133}{Cuatro núcleos a }\textcolor[rgb]{0.133,0.133,0.133}{3}\textcolor[rgb]{0.133,0.133,0.133}{,4}\\\textcolor[rgb]{0.133,0.133,0.133}{GHz en modo base y }\textcolor[rgb]{0.133,0.133,0.133}{3}\textcolor[rgb]{0.133,0.133,0.133}{,7 GHz en}\\\textcolor[rgb]{0.133,0.133,0.133}{modo turbo).}\end{tabular}\\Memoria RAM de 12 GB.\\Disco duro SATA3 de 500 GB SSD\end{tabular}\end{tabular}  \\ 
\hline
\textbf{Componente crítico evaluado:}                                                                   & \begin{tabular}[c]{@{}l@{}}\textbf{Búsqueda de Información}\\\textbf{}\end{tabular}                                                                                                                                                                                                                                                                                                                                                                                                                                                                                                                                                                                                                                                                                                                                                                                                                                                                                                                                                                                                                                                                                                                                                                                                                 \\ 
\hline
\textbf{Evidencia: }                                                                                    &                                                                                                                                                                                                                                                                                                                                                                                                                                                                                                                                                                                                                                                                                                                                                                                                                                                                                                                                                                                                                                                                                                                                                                                                                                                                                                     \\ 
\hline
\textbf{Componente crítico evaluado:}                                                                   & \begin{tabular}[c]{@{}l@{}}\textbf{Muestra de resultados de la }\\\textbf{búsqueda.}\end{tabular}                                                                                                                                                                                                                                                                                                                                                                                                                                                                                                                                                                                                                                                                                                                                                                                                                                                                                                                                                                                                                                                                                                                                                                                                   \\ 
\hline
\textbf{Evidencia:}                                                                                     &                                                                                                                                                                                                                                                                                                                                                                                                                                                                                                                                                                                                                                                                                                                                                                                                                                                                                                                                                                                                                                                                                                                                                                                                                                                                                                     \\ 
\hline
\textbf{Componente crítico evaluado:}                                                                   & \begin{tabular}[c]{@{}l@{}}\textbf{Análisis de la información}\\\textbf{(Docking-Regresión Lineal).}\end{tabular}                                                                                                                                                                                                                                                                                                                                                                                                                                                                                                                                                                                                                                                                                                                                                                                                                                                                                                                                                                                                                                                                                                                                                                                   \\ 
\hline
\textbf{Evidencia:}                                                                                     &                                                                                                                                                                                                                                                                                                                                                                                                                                                                                                                                                                                                                                                                                                                                                                                                                                                                                                                                                                                                                                                                                                                                                                                                                                                                                                     \\ 
\hline
\textbf{Componente crítico evaluado:}                                                                   & \begin{tabular}[c]{@{}l@{}}\textbf{Muestra de resultados}\\\textbf{(Ordenamiento).}\end{tabular}                                                                                                                                                                                                                                                                                                                                                                                                                                                                                                                                                                                                                                                                                                                                                                                                                                                                                                                                                                                                                                                                                                                                                                                                    \\ 
\hline
\textbf{Evidencia:}                                                                                     &                                                                                                                                                                                                                                                                                                                                                                                                                                                                                                                                                                                                                                                                                                                                                                                                                                                                                                                                                                                                                                                                                                                                                                                                                                                                                                     \\
\hline
\end{longtable}

\begin{longtable}{|l|l|}
\caption{Pruebas de sistema CPRS3 (Rendimiento del sistema).}\\ 
\hline
\begin{tabular}[c]{@{}l@{}}\textbf{Casos de Prueba de }\\\textbf{Rendimiento del Sistema.}\end{tabular} & \textbf{CPRS 3}                                                                                                                                                                                                                                                                                                                                                                                                                                                                                                                                                                                                                                                                                                                                                                                                                                                                  \endfirsthead 
\hline
\textbf{Máquina usada:}                                                                                 & \begin{tabular}[c]{@{}l@{}}\begin{tabular}{@{\labelitemi\hspace{\dimexpr\labelsep+0.5\tabcolsep}}l}\begin{tabular}[c]{@{}l@{}}\textbf{Características:~}\\\begin{tabular}{@{\labelitemi\hspace{\dimexpr\labelsep+0.5\tabcolsep}}l}\begin{tabular}[c]{@{}l@{}}Procesador Intel Core i7-6500U\\con gráficos Intel HD 520 (2,5 \\GHz, hasta 3,1 GHz, 4 MB de\\~caché, 2 núcleos).\end{tabular}\end{tabular}\\\\\begin{tabular}{@{\labelitemi\hspace{\dimexpr\labelsep+0.5\tabcolsep}}l}Memoria RAM de 8 GB\\Gráficos de video Intel HD 520\\disco duro SATA de 2 TB 5400 rpm\end{tabular}\end{tabular}\\\begin{tabular}[c]{@{}l@{}}Procesador Intel Core i7-6500U\\con gráficos Intel HD 520 (2,5 \\GHz, hasta 3,1 GHz, 4 MB de\\~caché, 2 núcleos).\end{tabular}\\Memoria RAM de 8 GB\\Gráficos de video Intel HD 520\\disco duro SATA de 2 TB 5400 rpm\end{tabular}\end{tabular}  \\ 
\hline
\textbf{Componente crítico evaluado:}                                                                   & \begin{tabular}[c]{@{}l@{}}\textbf{Búsqueda de Información}\\\textbf{}\end{tabular}                                                                                                                                                                                                                                                                                                                                                                                                                                                                                                                                                                                                                                                                                                                                                                                              \\ 
\hline
\textbf{Evidencia: }                                                                                    &                                                                                                                                                                                                                                                                                                                                                                                                                                                                                                                                                                                                                                                                                                                                                                                                                                                                                  \\ 
\hline
\textbf{Componente crítico evaluado:}                                                                   & \begin{tabular}[c]{@{}l@{}}\textbf{Muestra de resultados de la }\\\textbf{búsqueda.}\end{tabular}                                                                                                                                                                                                                                                                                                                                                                                                                                                                                                                                                                                                                                                                                                                                                                                \\ 
\hline
\textbf{Evidencia:}                                                                                     &                                                                                                                                                                                                                                                                                                                                                                                                                                                                                                                                                                                                                                                                                                                                                                                                                                                                                  \\ 
\hline
\textbf{Componente crítico evaluado:}                                                                   & \begin{tabular}[c]{@{}l@{}}\textbf{Análisis de la información}\\\textbf{(Docking-Regresión Lineal).}\end{tabular}                                                                                                                                                                                                                                                                                                                                                                                                                                                                                                                                                                                                                                                                                                                                                                \\ 
\hline
\textbf{Evidencia:}                                                                                     &                                                                                                                                                                                                                                                                                                                                                                                                                                                                                                                                                                                                                                                                                                                                                                                                                                                                                  \\ 
\hline
\textbf{Componente crítico evaluado:}                                                                   & \begin{tabular}[c]{@{}l@{}}\textbf{Muestra de resultados}\\\textbf{(Ordenamiento).}\end{tabular}                                                                                                                                                                                                                                                                                                                                                                                                                                                                                                                                                                                                                                                                                                                                                                                 \\ 
\hline
\textbf{Evidencia:}                                                                                     &                                                                                                                                                                                                                                                                                                                                                                                                                                                                                                                                                                                                                                                                                                                                                                                                                                                                                  \\
\hline
\end{longtable}
\subsection{Pruebas de volumen}
\begin{longtable}{|l|l|}
\caption{Caso de prueba para comunicaciones.}\\ 
\hline
\begin{tabular}[c]{@{}l@{}}\textbf{Caso de Prueba para}\\\textbf{comunicaciones.}\end{tabular} & \multicolumn{1}{c|}{\textbf{CPVS1}}                                                                                                                                                                                                                                                                                                                                                                                                                                                                      \endfirsthead 
\hline
\textbf{Máquina usada:}                                                                        & \begin{tabular}[c]{@{}l@{}}\textbf{Características:~}\\\begin{tabular}{@{\labelitemi\hspace{\dimexpr\labelsep+0.5\tabcolsep}}l}\begin{tabular}[c]{@{}l@{}}Procesador Intel Core \\i7-649DU (2,5 GHz, hasta \\3,1 GHz, 4 MB de caché,\\~4 núcleos).\end{tabular}\\Memoria RAM de 8GB .\\\begin{tabular}[c]{@{}l@{}}Disco Duro Toshiba MQ01 \\de 1TB.\end{tabular}\\\begin{tabular}[c]{@{}l@{}}Disco Duro Kingston SSD \\de 240 GB.Gráficos de \\video Intel HD510\end{tabular}\end{tabular}\end{tabular}  \\ 
\hline
\textbf{Componente crítico evaluado:}                                                          & \begin{tabular}[c]{@{}l@{}}Compuestos: 48\\Proteínas: 5\\\end{tabular}                                                                                                                                                                                                                                                                                                                                                                                                                                   \\ 
\hline
\textbf{Tiempo de ejecución:}                                                                  &                                                                                                                                                                                                                                                                                                                                                                                                                                                                                            \\ 
\hline
\textbf{Observaciones:}                                                                        &                                                                                                                                                                                                                                                                                                             \\
\hline
\end{longtable}

\begin{longtable}{|l|l|}
\caption{Caso de prueba para comunicaciones.}\\ 
\hline
\begin{tabular}[c]{@{}l@{}}\textbf{Caso de Prueba para}\\\textbf{comunicaciones.}\end{tabular} & \multicolumn{1}{c|}{\textbf{CPVS2}}                                                                                                                                                                                                                                                                                                                                                                                                                                                                                                                                                                                                                                                                                                                                                              \endfirsthead 
\hline
\textbf{Máquina usada:}                                                                        & \begin{tabular}[c]{@{}l@{}}\textbf{Características:~}\\\begin{tabular}{@{\labelitemi\hspace{\dimexpr\labelsep+0.5\tabcolsep}}l}\begin{tabular}[c]{@{}l@{}}Procesador AMD Ryzen\\3 con gráficos Radeon\\Vega Graphics Mobile Gfx \\(Cuatro núcleos a 3,4 GHz\\en modo base y 3,7 GHz \\en modo turbo).\end{tabular}\end{tabular}\\\\\begin{tabular}{@{\labelitemi\hspace{\dimexpr\labelsep+0.5\tabcolsep}}l}Memoria RAM de 12GB.\\\begin{tabular}[c]{@{}l@{}}Disco duro SATA3 de 500\\Gb SSD.\end{tabular}\end{tabular}\end{tabular}  \\ 
\hline
\textbf{Componente crítico evaluado:}                                                          & \begin{tabular}[c]{@{}l@{}}Compuestos: 48\\Proteínas: 5\\\end{tabular}                                                                                                                                                                                                                                                                                                                                                                                                                                                                                                                                                                                                                                                                                                                           \\ 
\hline
\textbf{Tiempo de ejecución:}                                                                  &                                                                                                                                                                                                                                                                                                                                                                                                                                                                                                                                                                                                                                                                                                                                                                                   \\ 
\hline
\textbf{Observaciones:}                                                                        &                                                                                                                                                                                                                                                                                                                                                                                                                                                                                                                  \\
\hline
\end{longtable}
\begin{longtable}{|l|l|}
\caption{Caso de prueba para comunicaciones.}\\ 
\hline
\begin{tabular}[c]{@{}l@{}}\textbf{Caso de Prueba para}\\\textbf{comunicaciones.}\end{tabular} & \multicolumn{1}{c|}{\textbf{CPVS3}}                                                                                                                                                                                                                                                                                                                                                                                                                                                               \endfirsthead 
\hline
\textbf{Máquina usada:}                                                                        & \begin{tabular}[c]{@{}l@{}}\textbf{Características:~}\\\begin{tabular}{@{\labelitemi\hspace{\dimexpr\labelsep+0.5\tabcolsep}}l}\begin{tabular}[c]{@{}l@{}}Procesador Intel Core \\i7-6500U con gráficos \\Intel HD 520 (2,5 GHz, \\hasta 3,1 GHz, 4 MB \\de caché, 2 núcleos).\end{tabular}\\Memoria RAM de 8GB\\\begin{tabular}[c]{@{}l@{}}Gráficos de video Intel\\~HD 520\end{tabular}\\\begin{tabular}[c]{@{}l@{}}Disco duro SATA de \\2 TB 5400 rpm.\end{tabular}\end{tabular}\end{tabular}  \\ 
\hline
\textbf{Componente crítico evaluado:}                                                          & \begin{tabular}[c]{@{}l@{}}Compuestos: 48\\Proteínas: 5\\\end{tabular}                                                                                                                                                                                                                                                                                                                                                                                                                            \\ 
\hline
\textbf{Tiempo de ejecución:}                                                                  &                                                                                                                                                                                                                                                                                                                                                                                                                                                                                  \\ 
\hline
\textbf{Observaciones:}                                                                        &                                                                                                                                                                                                                                                                                                                          \\
\hline
\end{longtable}

\newpage
\subsection{Pruebas de Disponibilidad de Catos del Sistema}
\begin{longtable}{|l|l|}
\caption{Caso de prueba para comunicaciones.}\\ 
\hline
\begin{tabular}[c]{@{}l@{}}\textbf{Caso de Prueba de }\\\textbf{Disponibilidad de }\\\textbf{Datos del Sistema. }\end{tabular} & \multicolumn{1}{c|}{\textbf{CPDDS 1}}                                                                                                                                                                            \endfirsthead 
\hline
\textbf{Criterio a evaluar:}                                                                                                   & Error de conexión.                                                                                                                                                                                               \\ 
\hline
\textbf{Descripción:}                                                                                                          &   \\ 
\hline
\textbf{Evidencia:}                                                                                                            &                                                                                                                                                                                           \\
\hline
\end{longtable}
\begin{longtable}{|l|l|}
\caption{Caso de prueba para comunicaciones.}\\ 
\hline
\begin{tabular}[c]{@{}l@{}}\textbf{Caso de Prueba de }\\\textbf{Disponibilidad de }\\\textbf{Datos del Sistema. }\end{tabular} & \multicolumn{1}{c|}{\textbf{CPDDS 2}}                                                                                                                                                                                                      \endfirsthead 
\hline
\textbf{Criterio a evaluar:}                                                                                                   & Cierre inesperado del programa.                                                                                                                                                                                                            \\ 
\hline
\textbf{Descripción:}                                                                                                          &   \\ 
\hline
\textbf{Evidencia:}                                                                                                            &                                                                                                                                                                                                                     \\
\hline
\end{longtable}
\begin{longtable}{|l|l|}
\caption{Caso de prueba para comunicaciones.}\\ 
\hline
\begin{tabular}[c]{@{}l@{}}\textbf{Caso de Prueba de }\\\textbf{Disponibilidad de }\\\textbf{Datos del Sistema. }\end{tabular} & \multicolumn{1}{c|}{\textbf{CPDDS 4}}                                                                                                                 \endfirsthead 
\hline
\textbf{Criterio a evaluar:}                                                                                                   & Carga de proyecto existente.                                                                                                                          \\ 
\hline
\textbf{Descripción:}                                                                                                          &  \\ 
\hline
\textbf{Evidencia:}                                                                                                            &                                                                                                                                 \\
\hline
\end{longtable}
\newpage
\begin{longtable}{|l|l|}
\caption{Caso de prueba para comunicaciones.}\\ 
\hline
\begin{tabular}[c]{@{}l@{}}\textbf{Caso de Prueba de }\\\textbf{Disponibilidad de }\\\textbf{Datos del Sistema. }\end{tabular} & \multicolumn{1}{c|}{\textbf{CPDDS 5}}                                                                                        \endfirsthead 
\hline
\textbf{Criterio a evaluar:}                                                                                                   & Conjunto inicial sin datos necesarios.                                                                                       \\ 
\hline
\textbf{Descripción:}                                                                                                          &   \\ 
\hline
\textbf{Evidencia:}                                                                                                            &                                                                                                      \\
\hline
\end{longtable}
}