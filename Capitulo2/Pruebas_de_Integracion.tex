\subsection{Pruebas de Integración}{

\noindent El objetivo de las pruebas de integración es verificar el correcto ensamblaje entre los distintos componentes una vez que han sido probados unitariamente con el fin de comprobar que interactúan correctamente a través de sus interfaces, tanto internas como externas, cubren la funcionalidad establecida y se ajustan a los requisitos no funcionales especificados en las verificaciones correspondientes.\\

\noindent Existen diversas formas de desarrollar las pruebas de integración, ya sea de manera incremental o no incremental. En un inicio se tienen contemplado un modelo incremental, donde se combina  un componente,para proseguir con el siguiente componente que se debe probar para obtener un  conjunto de componentes que ya están probados y se va incrementando progresivamente el número de componentes a probar.\\

\noindent Dentro de un planteamiento incremental, tenemos diversos modelos para realizar la pruebas, uno el de Arriba-abajo, De Abajo a Arriba y el Combinado.\\

\noindent De esta manera, se ha considerado, un modelo Arriba-abajo, en el cual el primer componente que se desarrolla y prueba es el primero de la jerarquía. Los componentes de nivel más bajo se sustituyen por componentes auxiliares para simular a los componentes invocados. En este caso no son necesarios componentes conductores. Una de las ventajas de aplicar esta estrategia es que las interfaces entre los distintos componentes se prueban en una fase temprana y con frecuencia.\\

\noindent De acuerdo a esta jerarquía, se contemplan 3 módulos principales, que son:

\begin{itemize}
    \item Búsqueda de Información.
    \item Procesamiento de Información.
    \item Generar listado de resultados.
\end{itemize}

\noindent Siendo eso tres módulos esenciales, continuando con lo obtenido en las pruebas unitarias, en listamos que requerimientos funcionales corresponden a cada módulo.

\begin{itemize}
    \item Búsqueda de Información.
    \begin{itemize}
        \item Adquisición del archivo base.
        \item Búsqueda de los compuestos indicados.
        \item Búsqueda de los descriptores de los compuestos.
        \item Búsqueda de las actividades biológicas de los compuestos.
        \item Búsqueda de las proteínas indicadas.
        \item Confirmación de los resultados de la búsqueda.
        \begin{itemize}
            \item Construcción del conjunto cero.
            \item Construcción del conjunto P
        \end{itemize}
    \end{itemize}
    \item Procesamiento de Información
    \begin{itemize}
        \item Adquisición y descomposición del conjunto0.
        \item Adquisición y descomposición del conjuntoP.
        \item Método de Machine Learning para optimización.
        \item Modelado QSAR.
    \end{itemize}
    \item Generar listado de resultados
    \begin{itemize}
        \item Lectura y ordenamiento del conjunto final.
        \item Diseño gráfico de resultados.
    \end{itemize}
\end{itemize}

%%%%%%%%%%%%%%%%%%%%%%%%%
\begin{longtable}{|p{4cm}|p{9.5cm}|}


\caption{Prueba Integradora MI1}\\ 

\endhead
\hline

\begin{tabular}[c]{@{}l@{}}\textbf{Módulo }\\\textbf{Funcional}\end{tabular}         & MI 1 Búsqueda De
Información                                                                                                                                                                                                                                                                                                                                                                                                                                                                                                                                                                                                                                                                                                                                                                                                                                                                                                        \endfirsthead 
\hline
\begin{tabular}[c]{@{}l@{}}\textbf{Perfiles}\\\textbf{Implicados}\end{tabular}       & \begin{tabular}[c]{@{}l@{}}- Desarrollador.\\- Tester.\end{tabular}                                                                                                                                                                                                                                                                                                                                                                                           \\ 
\hline
\begin{tabular}[c]{@{}l@{}}\textbf{Planificación }\\\textbf{temporal}\end{tabular}   & \begin{tabular}[c]{@{}l@{}} 1. Revisión de sintaxis al unir módulos secundarios.\\2. Revisión de código completo de cada submódulo.\\2.1 Adquisición del archivo base. \\2.2 Búsqueda de los compuestos indicados.\\2.3~Búsqueda de los descriptores de los compuestos.\\2.4 Búsqueda de las actividades biológicas de los \\compuestos.\\2.5 Búsqueda de las proteínas indicadas.\\2.6 Confirmación de los resultados de la búsqueda.\\2.7.1 Construcción del conjunto cero.\\2.7.2 Construcción del conjunto P\end{tabular}                                                                                                                                                                                                                                                                                                                                                                                                        \\ 
\hline
\begin{tabular}[c]{@{}l@{}}\textbf{Criterio de }\\\textbf{verificación}\end{tabular} & \begin{tabular}[p{9.5cm}]{@{}l@{}}
1. Verificación en las herramientas que ofrece el IDE,\\ no marque errores de sintaxis.\\2. Verificación de que el código agregado corresponde\\ al módulo funcional, y si este requiere cierta \\modificación para la integración, se conserve la \\estructura pragmática del módulo.\\2.1 Integridad de código para la adquisición \\del archivo base.~ \\2.2 Integridad de código para la búsqueda de los \\compuestos indicados.\\2.3 Integridad de código para la búsqueda de los \\descriptores de los compuestos.\\2.4 Integridad de código para la búsqueda de las \\actividades biológicas de los compuestos.\\2.5 Código funcional para la búsqueda de las \\proteínas indicadas.\\2.6 Funcionalidad en la confirmación de los \\resultados de la búsqueda.\\2.7.1 Código completo para la construcción del \\conjunto cero.\\2.7.2 Código completo para construcción del \\conjunto P.\end{tabular}  \\ 
\hline
\begin{tabular}[c]{@{}l@{}}\textbf{Criterio de }\\\textbf{aceptación}\end{tabular}   & \begin{tabular}[p[9.5cm]{@{}l@{}} 1. No hay errores de sintaxis al agregar un \\módulo al programa principal, manteniendo \\un orden y estructura adecuada.\\2. Al agregar un nuevo módulo, el código de \\este se preserva funcionalmente.  \\2.1 Adquisición del archivo base.~  \\2.2 Búsqueda de los compuestos indicados. \\2.3 Búsqueda de los descriptores de los compuestos. \\2.4 Búsqueda de las actividades biológicas de los \\compuestos. \\2.5 Búsqueda de las proteínas indicadas. \\2.6 Confirmación de los resultados de la búsqueda.\\2.7.1 Construcción del conjunto cero. \\2.7.2 Construcción del conjunto P  \end{tabular}                                                                                                                                                                                                                                                                                  \\ 
\hline
\textbf{Definición de verificaciones}                                                & \begin{tabular}[c]{@{}l@{}}- Errores de Compilación: Ocurren porque la \\sintaxis del lenguaje no es correcta, de cajón este \\tipo de errores no permitenque la aplicación se \\ejecute. \\-Acción de botón: representa un botón que,\\cuando es presionado, envía información al que \\pertenece. La función de~ un botón representada~ \\el contenido del elemento.\\-Visualización de pantalla interfaz de usuario.~\end{tabular}                                                                                                                                                                                                                                                                                                                                                                                                                                                                                                \\ 
\hline
\textbf{Análisis y evaluación de resultados}                                         & - Resultados:~                                                                                                                                                                                                                                                                                                                                                                                                                                                                                                                                                                                                                                                                                                                                                                                                                                                                                                                       \\ 
\hline
\textbf{Productos a entregar}                                                        & \begin{tabular}[c]{@{}l@{}}-Adquisición delarchivo base funcionando \\correctamente.\end{tabular}                                                                                                                                                                                                                                                                                                                                                                                                                                                                                                                                                                                                                                                                                                                                                                                                                                    \\
\hline
\end{longtable}


%%%%%%%%%%%%%%%%%%%%%%%%%%%%%%%%%%%%5

\begin{longtable}{|p{4cm}|p{9.5cm}|}
\caption{Caso de prueba CPI1}\\ 
\hline
\textbf{ID del Caso de prueba}                                                               & CPI1                                                                                                                                                                                                                                           \endfirsthead 
\hline
\textbf{Versión}                                                                             & 1.0                                                                                                                                                                                                                                            \\ 
\hline
\textbf{Nombre}                                                                              & \begin{tabular}[c]{@{}l@{}}Caso de prueba de integración para el módulo de \\búsqueda de información.\end{tabular}                                                                                                                              \\ 
\hline
\begin{tabular}[c]{@{}l@{}}\textbf{Identificador de }\\\textbf{requerimientos.}\end{tabular} & \begin{tabular}[c]{@{}l@{}}RF1.1, RF1.2, RF1.3, RF1.4, RF1.5, RF1.6, RF1.7.1, \\RF1.7.2\end{tabular}                                                                                                                                            \\ 
\hline
\textbf{Propósito~~}                                                                         & \begin{tabular}[c]{@{}l@{}}Poder verificar la continua funcionalidad del módulo \\principal mientras son agregados a el módulo sobre \\el que se está trabajando.\end{tabular}                                                                   \\ 
\hline
\textbf{Dependencias}                                                                        & N/A                                                                                                                                                                                                                                            \\ 
\hline
\textbf{Ambiente de prueba/configuración}                                                    & \begin{tabular}[c]{@{}l@{}}Hardware: Equipo de computo \\(preferentemente portátil)\\Software: Compilador python3, IDE y/o editor de \\texto.\end{tabular}                                                                                       \\ 
\hline
\textbf{Inicialización}                                                                      & \begin{tabular}[c]{@{}l@{}}- Se cuenta con cada uno de los requerimientos \\funcionales que conforman el módulo de la búsqueda \\de información.\\- Se posee el archivo base que es necesario para que \\todo el módulo funcione.\end{tabular}  \\ 
\hline
\textbf{Finalización}                                                                        & N/A                                                                                                                                                                                                                                            \\ 
\hline
\textbf{Acciones}                                                                            & \begin{tabular}[c]{@{}l@{}}- Agregar el módulo por modulo al programa\\ principal, verificando que cada que se agregue un \\componente.\\- Colocar el archivo en el directorio especificado.\end{tabular}                                      \\ 
\hline
\textbf{Salida esperada}                                                                     & \begin{tabular}[c]{@{}l@{}}- Notificación de los resultados obtenidos en la \\búsqueda de~información.\end{tabular}                                                                                                                              \\ 
\hline
   \textbf{Salida obtenida}                                                                  &                                                                                                                                                                                                                                                \\ 
\hline
\textbf{Resultado}~                                                                          &                                                                                                                                                                                                                                                \\ 
\hline
\textbf{Severidad}~                                                                          &                                                                                                                                                                                                                                                \\ 
\hline
\textbf{Evidencia}                                                                           &                                                                                                                                                                                                                                                \\ 
\hline
\textbf{Estado}                                                                              & No Iniciado.                                                                                                                                                                                                                                   \\
\hline
\end{longtable}

\begin{longtable}{|p{4cm}|p{9.5cm}|}
\caption{Prueba Integradora MI2}\\ 
\hline
\begin{tabular}[c]{@{}l@{}}\textbf{Módulo }\\\textbf{Funcional}\end{tabular}         & \begin{tabular}[c]{@{}l@{}}MI 2 Procesamiento de Información\end{tabular}                                                                                                                                                                                                                                                                                                                                                                                                                                                                                                                                                                                                                                                                                                                                                                                                                                                                          \endfirsthead 
\hline
\begin{tabular}[c]{@{}l@{}}\textbf{Perfiles}\\\textbf{Implicados}\end{tabular}       & \begin{tabular}[c]{@{}l@{}}- Desarrollador.\\- Tester.\end{tabular}                                                                                                                                                                                                                                                                                                                                                                                                                                                                                                                                                                                                                                                                                                                                                                                                                                                                                 \\ 
\hline
\begin{tabular}[c]{@{}l@{}}\textbf{Planificación }\\\textbf{temporal}\end{tabular}   & \begin{tabular}[c]{@{}l@{}}3. Revisión de sintaxis al unir módulos secundarios.\\4. Revisión de código completo de cada submódulo.\\Adquisición y descomposición del conjunto0.~\\2.2~ Adquisición y descomposición del conjuntoP.~\\2.3 Método de Machine Learning para optimización.~\\2.4 Modelado QSAR.\end{tabular}                                                                                                                                                                                                                                                                                                                                                                                                                                                                                                                                                                                                                                \\ 
\hline
\begin{tabular}[c]{@{}l@{}}\textbf{Criterio de }\\\textbf{verificación}\end{tabular} & \begin{tabular}[c]{@{}l@{}}1. Verificación en las herramientas que ofrece el IDE,\\no marque errores de sintaxis .\\2. Verificación de que el código agregado corresponde \\al módulo funcional, y si este requiere cierta \\modificación para la integración, se conserve la \\estructura pragmática del módulo.\\2.1 Integridad de código para la adquisición del \\archivo base.~~\\2.2 Integridad de código para la búsqueda de los \\compuestos indicados.~\\2.3 Integridad de código para la búsqueda de los \\descriptores de los compuestos.~\\2.4 Integridad de código para la búsqueda de las \\actividades biológicas de los compuestos.~\\2.5 Código funcional para la búsqueda de las \\proteínas indicadas.~\\2.6 Funcionalidad en la confirmación de los \\resultados de la búsqueda.~\\2.7.1 Código completo para la construcción del \\conjunto cero.~\\2.7.2 Código completo para construcción del \\conjunto P.\end{tabular}  \\ 
\hline
\begin{tabular}[c]{@{}l@{}}\textbf{Criterio de }\\\textbf{aceptación}\end{tabular}   & \begin{tabular}[c]{@{}l@{}}    1. No hay errores de sintaxis al agregar un módulo \\al programa principal, manteniendo un orden y \\estructura adecuada.\\2. Al agregar un nuevo módulo, el código de este se \\preserva funcionalmente.\\2.1 Adquisición del archivo base.\\2.2 Búsqueda de los compuestos indicados.\\2.3 Búsqueda de los descriptores de los compuestos.\\2.4 Búsqueda de las actividades biológicas de los \\compuestos.\\2.5 Búsqueda de las proteínas indicadas.\\2.6 Confirmación de los resultados de la búsqueda.\\2.7.1 Construcción del conjunto cero.~\\2.7.2 Construcción del conjunto P.\end{tabular}                                                                                                                                                                                                                                                                                                               \\ 
\hline
\textbf{Definición de verificaciones}                                                & \begin{tabular}[c]{@{}l@{}}- Errores de Compilación: Ocurren porque la \\sintaxis del lenguaje no es correcta, de cajón \\este tipo de errores no permitenque la \\aplicación se ejecute. \\-Acción de botón: representa un botón que,\\cuando es presionado, envía información al que \\pertenece. La función de~ un botón representada~ \\el contenido del elemento.\\-Visualización de pantalla interfaz de usuario.~\end{tabular}                                                                                                                                                                                                                                                                                                                                                                                                                                                                                                               \\ 
\hline
\textbf{Análisis y evaluación de resultados}                                         & - Resultados:~                                                                                                                                                                                                                                                                                                                                                                                                                                                                                                                                                                                                                                                                                                                                                                                                                                                                                                                                      \\ 
\hline
\textbf{Productos a entregar}                                                        & \begin{tabular}[c]{@{}l@{}}-Adquisición delarchivo base funcionando \\correctamente.\end{tabular}                                                                                                                                                                                                                                                                                                                                                                                                                                                                                                                                                                                                                                                                                                                                                                                                                                                   \\
\hline
\end{longtable}


%%%%%%%%%%%%%%%%%%%%%%%%%%%%%%%%%%%

\begin{longtable}{|p{4cm}|p{9.5cm}|}
\caption{Caso de prueba CPI2}\\ 
\hline
 \textbf{ID del Caso de prueba}                                                                 & CPI2                                                                                                                                                                                                                                                                                                                        \endfirsthead 
\hline
\textbf{Versión}                                                                                & 1.0                                                                                                                                                                                                                                                                                                                         \\ 
\hline
\textbf{Nombre}                                                                                 & \begin{tabular}[c]{@{}l@{}}Caso de prueba de integración para el módulo de \\búsqueda de información.\end{tabular}                                                                                                                                                                                                           \\ 
\hline
\begin{tabular}[c]{@{}l@{}}\textbf{Identificador de }\\\textbf{requerimientos.} \end{tabular}   & RF2.1, RF2.2, RF2.3, RF2.4                                                                                                                                                                                                                                                                                                  \\ 
\hline
\textbf{Propósito}                                                                              & \begin{tabular}[c]{@{}l@{}}Revisar el adecuado y continuo funcionamiento \\de cada uno de los componentes que conforman \\el módulo de procesamiento de la información. \\Detectar si algún módulo falla en su\\funcionamiento esencial o al momento de recibir \\información o en la transmisión de resultados.\end{tabular}  \\ 
\hline
\textbf{Dependencias}                                                                           & N/A                                                                                                                                                                                                                                                                                                                         \\ 
\hline
\textbf{Ambiente de prueba/configuración}                                                       & \begin{tabular}[c]{@{}l@{}}Hardware: Equipo de cómputo (preferentemente \\portátil)\\Software: Compilador python3, IDE y/o editor de \\texto.\end{tabular}                                                                                                                                                                     \\ 
\hline
\textbf{Inicialización}                                                                         & \begin{tabular}[c]{@{}l@{}}- Se Revisa que previamente cada uno de los\\requerimientos que conforman el módulo hayan \\pasado el plan de pruebas unitario.\\- Se cuentan con los datos suficientes para \\poderejecutar los componentes.\end{tabular}                                                                        \\ 
\hline
\textbf{Finalización}                                                                           & N/A                                                                                                                                                                                                                                                                                                                         \\ 
\hline
\textbf{Acciones}                                                                               & \begin{tabular}[c]{@{}l@{}}- Agregar el módulo por modulo al programa\\principal, verificando que cada que se agregue un \\componente.\\- Verificar que los conjuntos 0 y P de cada\\compuesto y proteína están íntegros.\end{tabular}                                                                                      \\ 
\hline
\begin{tabular}[c]{@{}l@{}}\textbf{Descripción de los }\\\textbf{datos de entrada}\end{tabular} & \begin{tabular}[c]{@{}l@{}}- Conjunto 0 de cada compuesto disponible u\\encontrado.\\- Conjunto P de cada compuesto disponible u\\encontrado.\\- Aprobación del usuario para iniciar el procesamiento\\de información.\end{tabular}                                                                                         \\ 
\hline
\textbf{Salida esperada}                                                                        & \begin{tabular}[c]{@{}l@{}}- Notificación de los resultados obtenidos en la\\búsqueda de información. Notificación que informe \\si existió una falla durante el procesamiento.\end{tabular}                                                                                                                                \\ 
\hline
 \textbf{Salida obtenida}                                                                       &                                                                                                                                                                                                                                                                                                                             \\ 
\hline
\textbf{Resultado}                                                                              &                                                                                                                                                                                                                                                                                                                             \\ 
\hline
\textbf{Severidad}                                                                              &                                                                                                                                                                                                                                                                                                                             \\ 
\hline
\textbf{Evidencia}                                                                              &                                                                                                                                                                                                                                                                                                                             \\ 
\hline
\textbf{Estado}                                                                                 & No Iniciado.                                                                                                                                                                                                                                                                                                                \\
\hline
\end{longtable}

\begin{longtable}{|p{4cm}|p{9.5cm}|}
\caption{Prueba Integradora MI3}\\ 
\hline
\begin{tabular}[c]{@{}l@{}} \textbf{Módulo }\\\textbf{Funcional} \end{tabular}        & MI 3 Generar listado de resultados                                                                                                                                                                                                                                                                                                                                                                                                                                                                                             \endfirsthead 
\hline
\begin{tabular}[c]{@{}l@{}}\textbf{Perfiles}\\\textbf{Implicados} \end{tabular}       & \begin{tabular}[c]{@{}l@{}}- Desarrollador.\\- Tester. \end{tabular}                                                                                                                                                                                                                                                                                                                                                                                                                                                           \\ 
\hline
\begin{tabular}[c]{@{}l@{}}\textbf{Planificación }\\\textbf{temporal} \end{tabular}   & \begin{tabular}[c]{@{}l@{}}1. Revisión de sintaxis al unir módulos secundarios.\\2. Revisión de código completo de cada submódulo.\\2.1 Lectura y ordenamiento del conjunto final.\\2.2 Diseño gráfico de resultados.\end{tabular}                                                                                                                                                                                                                                                                                                \\ 
\hline
\begin{tabular}[c]{@{}l@{}}\textbf{Criterio de }\\\textbf{verificación} \end{tabular} & \begin{tabular}[c]{@{}l@{}}1. Verificación en las herramientas que ofrece el IDE,\\no marque errores de sintaxis .\\2. Verificación de que el código agregado \\corresponde al módulo funcional, y si este requiere\\cierta modificación para la integración, se conserve \\la estructura pragmática del módulo.\\2.1 Integridad de código para lalectura y el \\ordenamiento de resultados obtenidos en el \\conjunto final.\\2.2 Integridad de código para el gráfico deresultados.\end{tabular}                               \\ 
\hline
\begin{tabular}[c]{@{}l@{}}\textbf{Criterio de }\\\textbf{aceptación} \end{tabular}   & \begin{tabular}[c]{@{}l@{}} 1. No hay errores de sintaxis al agregar un módulo \\al programa principal, manteniendo un orden y \\estructura adecuada.\\2. Al agregar un nuevo módulo, el código de este se \\preserva funcionalmente.\\2.1 Lectura y ordenamiento deresultados \\provenientes del conjunto final.~\\2.2 Gráfico de los resultados~\end{tabular}                                                                                                                                                                \\ 
\hline
\textbf{Definición de verificaciones}                                                 & \begin{tabular}[c]{@{}l@{}}- Errores de Compilación: Ocurren porque la \\sintaxis del lenguaje no es correcta, de cajón \\este tipo de errores no permiten que la aplicación \\se ejecute. \\Errores de sintaxis: Un \textbf{error de sintaxis} en \\informáticay programación es una violación a las \\reglas de sintaxis en los lenguajes de programación.\\Se producen cuando la estructura de una de las \\instrucciones infringe una o varias reglas sintácticas \\definidas en ese lenguaje de programación\end{tabular}  \\ 
\hline
\textbf{Análisis y evaluación de resultados}                                          & - Resultados:                                                                                                                                                                                                                                                                                                                                                                                                                                                                                                                  \\ 
\hline
\textbf{Productos a entregar}                                                         & \begin{tabular}[c]{@{}l@{}}-Adquisición del archivo base funcionando \\correctamente.\end{tabular}                                                                                                                                                                                                                                                                                                                                                                                                                              \\
\hline
\end{longtable}

%%%%%%%%%%%%%%%%%%%%%%%%%%%%%%%%%%%%%%%%%%%%%%%%

\begin{longtable}{|p{4cm}|p{9.5cm}|}
\caption{Caso de prueba CPI1}\\ 
\hline
 \textbf{ID del Caso de prueba}                                                                 & CPI3                                                                                                                                                                                                                                                                                                                  \endfirsthead 
\hline
\textbf{Versión}                                                                                & 1.0                                                                                                                                                                                                                                                                                                                   \\ 
\hline
\textbf{Nombre}                                                                                 & \begin{tabular}[c]{@{}l@{}}Caso de prueba de integración para el módulo de \\búsqueda de información.\end{tabular}                                                                                                                                                                                                     \\ 
\hline
\begin{tabular}[c]{@{}l@{}}\textbf{Identificador de }\\\textbf{requerimientos.} \end{tabular}   & RF3.1, RF3.2                                                                                                                                                                                                                                                                                                          \\ 
\hline
\textbf{Propósito}                                                                              & \begin{tabular}[c]{@{}l@{}}Revisar el adecuado ycontinuo funcionamiento de\\cada uno de los componentes que conforman el\\módulode procesamiento de la información. \\Detectar si algún módulo falla en su\\funcionamiento esencial o al momento de recibir \\información o en la muestra deresultados.\end{tabular}  \\ 
\hline
\textbf{Dependencias}                                                                           & N/A                                                                                                                                                                                                                                                                                                                   \\ 
\hline
\textbf{Ambiente de prueba/configuración}                                                       & \begin{tabular}[c]{@{}l@{}}Hardware: Equipo de cómputo (preferentemente \\portátil)\\Software: Compilador python3, IDE y/o editor de \\texto.\end{tabular}                                                                                                                                                               \\ 
\hline
\textbf{Inicialización}                                                                         & \begin{tabular}[c]{@{}l@{}}- Se Revisa que previamente cada uno de los\\requerimientos que conforman el módulo hayan \\pasado el plan de pruebasunitario.\\- Se cuentan con los datos suficientes para \\poderejecutar los componentes.\end{tabular}                                                                  \\ 
\hline
\textbf{Finalización}                                                                           & N/A                                                                                                                                                                                                                                                                                                                   \\ 
\hline
\textbf{Acciones}                                                                               & \begin{tabular}[c]{@{}l@{}}- Agregar el módulo por modulo al programa\\principal, verificando que cada que se agregue \\un componente.\\-~Verificar que losresultados plasmados en el \\conjunto final se en el formato adecuado y \\legibles.\end{tabular}                                                           \\ 
\hline
\begin{tabular}[c]{@{}l@{}}\textbf{Descripción de los }\\\textbf{datos de entrada}\end{tabular} & \begin{tabular}[c]{@{}l@{}}- Conjunto 0 de cada compuesto disponible u\\encontrado.~\\- Aprobación del usuario para iniciar el\\procesamiento de información.\end{tabular}                                                                                                                                            \\ 
\hline
\textbf{Salida esperada}                                                                        & \begin{tabular}[c]{@{}l@{}}- Notificación de errores en la obtención de resultados \\tras la función del módulo de procesamiento.\\- Notificación que informe un error en la lectura\\ de el contenido del conjunto final.\end{tabular}                                                                                \\ 
\hline
 \textbf{Salida obtenida}                                                                       &                                                                                                                                                                                                                                                                                                                       \\ 
\hline
\textbf{Resultado}                                                                              &                                                                                                                                                                                                                                                                                                                       \\ 
\hline
\textbf{Severidad}                                                                              &                                                                                                                                                                                                                                                                                                                       \\ 
\hline
\textbf{Evidencia}                                                                              &                                                                                                                                                                                                                                                                                                                       \\ 
\hline
\textbf{Estado}                                                                                 & No Iniciado.                                                                                                                                                                                                                                                                                                          \\
\hline
\end{longtable}




%%%%%%%%%%%%%

}