% Please add the following required packages to your document preamble:
% \usepackage{longtable}
% Note: It may be necessary to compile the document several times to get a multi-page table to line up properly
\begin{longtable}{|l|l|}
\caption{Prueba unitaria RF3.2}
\label{PU_RF3_2}\\
\hline
\textbf{Requerimiento Funcional}                                                        & \textbf{RF 3.2  Diseño gráfico de resultados.}                                                                                                                                                                                                                                                                                                                                                                                                                                                                                                                                                                                                                                                                      \\ \hline
\endfirsthead
%
\multicolumn{2}{c}%
{{\bfseries Tabla \thetable\ Continuación de la página anterior}} \\
\endhead
%
\textbf{Perfiles Implicados}                                                            & \begin{tabular}[c]{@{}l@{}}- Desarrollador.\\ - Tester.\end{tabular}                                                                                                                                                                                                                                                                                                                                                                                                                                                                                                                                                                                                                                                \\ \hline
\textbf{Planificación temporal}                                                         & \begin{tabular}[c]{@{}l@{}}1. Prueba de compilación.\\ 2. Prueba de funcionamiento.\\ 2.1 El sistema puede acceder al archivo \\ conjunto final.\\ 2.2 Lectura del contenido existente en el \\ conjunto final.\\ 2.3 Adquisición  de los datos necesarios para \\ la graficación de los mismos.\\ 2.4 Graficación de los resultados.\end{tabular}                                                                                                                                                                                                                                                                                                                                                                  \\ \hline
\textbf{Criterio de verificación}                                                       & \begin{tabular}[c]{@{}l@{}}1. Compilación del código.\\ 2. Ejecutable del módulo, hay muestras de que \\ el código hace algo\\ (algo: definido por los siguientes puntos).\\ 2.1 Verificación de acceso al archivo \\ conjunto final.\\ 2.2 Rectificar lectura de datos existentes \\ en el archivo.\\ 2.3 Verificar correcta graficación de datos.\end{tabular}                                                                                                                                                                                                                                                                                                                                                    \\ \hline
\textbf{Criterio de aceptación}                                                         & \begin{tabular}[c]{@{}l@{}}1. No hay errores que impidan la compilación \\ del código.\\ 2. Al usar el ejecutable del módulo, hay \\ muestras de que el código hace algo\\ (algo: definido por los siguientes puntos).\\ 2.1 Es posible el acceso al archivo \\ conjunto final.\\ 2.2 El sistema notifica errores existente \\ al intentar accesar al archivo.\\ 2.3 El sistema obtiene toda la información \\ existente de dicho archivo.\\ 2.4 Se grafican correctamente los resultados.\\ 2.5. Se notifica la adecuada obtención de los \\ descriptores del compuesto.\\ 2.6 El sistema informa la correcta creación \\ del archivo.\\ 2.7 El archivo y la información que contiene \\ es integra.\end{tabular} \\ \hline
\textbf{Definición de verificaciones}                                                   & \begin{tabular}[c]{@{}l@{}}- Errores de Compilación: Ocurren porque \\ la sintaxis del lenguaje no es correcta, de \\ cajón este tipo de errores no permiten que \\ la aplicación se ejecute.\end{tabular}                                                                                                                                                                                                                                                                                                                                                                                                                                                                                                          \\ \hline
\textbf{\begin{tabular}[c]{@{}l@{}}Análisis y \\ evaluación de resultados\end{tabular}} & - Resultados:                                                                                                                                                                                                                                                                                                                                                                                                                                                                                                                                                                                                                                                                                                       \\ \hline
\textbf{Productos  a entregar}                                                          & \begin{tabular}[c]{@{}l@{}}- Diseño  gráfico de resultados funcionando \\ correctamente.\end{tabular}                                                                                                                                                                                                                                                                                                                                                                                                                                                                                                                                                                                                               \\ \hline
\end{longtable}