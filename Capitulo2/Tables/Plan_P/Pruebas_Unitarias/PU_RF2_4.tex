% Please add the following required packages to your document preamble:
% \usepackage{longtable}
% Note: It may be necessary to compile the document several times to get a multi-page table to line up properly
\begin{longtable}{|l|l|}
\caption{Prueba unitaria RF2.4}
\label{PU_RF_2_4}\\
\hline
\textbf{Requerimiento Funcional}                                                        & \textbf{RF 2.4  Modelado QSAR.}                                                                                                                                                                                                                                                                                                                                                                                                                                                                                                                                                                                                                                                                                                                                                                        \\ \hline
\endfirsthead
%
\multicolumn{2}{c}%
{{\bfseries Tabla \thetable\ Continuación de la página anterior}} \\
\endhead
%
\textbf{Perfiles Implicados}                                                            & \begin{tabular}[c]{@{}l@{}}- Desarrollador.\\ - Tester.\end{tabular}                                                                                                                                                                                                                                                                                                                                                                                                                                                                                                                                                                                                                                                                                                                                   \\ \hline
\textbf{Planificación temporal}                                                         & \begin{tabular}[c]{@{}l@{}}1. Prueba de compilación.\\ 2. Prueba de funcionamiento.\\ 2.1 Verificar que el sistema mantiene el valor \\ para los datos del compuesto.\\ 2.2 Verificar que el sistema mantiene el valor \\ para los datos de la proteína.\\ 2.3 Comprobar que la QSAR recibe los datos \\ adecuados.\\ 2.4 Verificar el adecuado funcionamiento de \\ cada una de las etapas de errores.\\ 2.5 Revisión de notificaciones de errores en \\ etapas de QSAR. \\ 2.6 Análisis del resultado arrojado por QSAR.\end{tabular}                                                                                                                                                                                                                                                                \\ \hline
\textbf{Criterio de verificación}                                                       & \begin{tabular}[c]{@{}l@{}}1. Compilación del código.\\ 2. Ejecutable del módulo, hay muestras de que \\ el código hace algo(algo: definido por los \\ siguientes puntos).\\ 2.1 Verificación de cada una de las variables \\ que poseen datos del compuesto.\\ 2.2 Verificación de cada una de las variables \\ que poseen datos de la proteína\\ 2.3 Revisión de las etapas por las que se \\ efectúa QSAR.\\ 2.4 Corroborar vistas y mensajes de errores \\ durante el modelado QSAR.\\ 2.5 Revisión del resultado de QSAR.\end{tabular}                                                                                                                                                                                                                                                            \\ \hline
\textbf{Criterio de aceptación}                                                         & \begin{tabular}[c]{@{}l@{}}1. No hay errores que impidan la compilación \\ del código.\\ 2. Al usar el ejecutable del módulo, hay \\ muestras de que el código hace algo\\ (algo: definido por los siguientes puntos).\\ 2.1 No existe error o ha sido cambiado el \\ valor de alguna de las variables perteneciente \\ al compuesto.\\ 2.2 2.1 No existe error o ha sido cambiado el \\ valor de alguna de las variables perteneciente \\ a la proteína.\\ 2.3 El sistema desarrolla correctamente el \\ modelado QSAR.\\ 2.4 El sistema informa errores existentes \\ durante el desarrollo de QSAR.\\ 2.5. Correcta visualización de los mensajes \\ que notifican los errores.\\ 2.6 El sistema informa si la obtención del \\ resultado.\\ 2.7 El sistema respalda dicho resultado.\end{tabular} \\ \hline
\textbf{Definición de verificaciones}                                                   & \begin{tabular}[c]{@{}l@{}}- Errores de Compilación: Ocurren porque la \\ sintaxis del lenguaje no es correcta, de cajón \\ este tipo de errores no permiten que la aplicación \\ se ejecute.\end{tabular}                                                                                                                                                                                                                                                                                                                                                                                                                                                                                                                                                                                             \\ \hline
\textbf{\begin{tabular}[c]{@{}l@{}}Análisis y \\ evaluación de resultados\end{tabular}} & - Resultados:                                                                                                                                                                                                                                                                                                                                                                                                                                                                                                                                                                                                                                                                                                                                                                                          \\ \hline
\textbf{Productos  a entregar}                                                          & -Búsqueda del Modelado QSAR.                                                                                                                                                                                                                                                                                                                                                                                                                                                                                                                                                                                                                                                                                                                                                                           \\ \hline
\end{longtable}