% Please add the following required packages to your document preamble:
% \usepackage{longtable}
% Note: It may be necessary to compile the document several times to get a multi-page table to line up properly
\begin{longtable}{|l|l|}
\caption{Prueba unitaria RF3.1}
\label{PU_RF3_1}\\
\hline
\textbf{Requerimiento Funcional}                                                        & \textbf{\begin{tabular}[c]{@{}l@{}}RF 3.1  Lectura y ordenamiento del \\ conjunto final.\end{tabular}}                                                                                                                                                                                                                                                                                                                                                                                                                                                                                \\ \hline
\endfirsthead
%
\multicolumn{2}{c}%
{{\bfseries Tabla \thetable\ Continuación de la página anterior}} \\
\endhead
%
\textbf{Perfiles Implicados}                                                            & \begin{tabular}[c]{@{}l@{}}- Desarrollador.\\ - Tester.\end{tabular}                                                                                                                                                                                                                                                                                                                                                                                                                                                                                                                  \\ \hline
\textbf{Planificación temporal}                                                         & \begin{tabular}[c]{@{}l@{}}1 Prueba de compilación.\\ 2. Prueba de funcionamiento.\\ 2.1 Verificar que  el resultado obtenido \\ ha sido agregado al conjunto final.\\ 2.2 Comprobar estado del archivo que \\ contiene todos los resultados finales.\\ 2.3 Ordenamiento de los resultados almacenados \\ al conjunto final.\\ 2.4 Almacenado del conjunto final.\end{tabular}                                                                                                                                                                                                        \\ \hline
\textbf{Criterio de verificación}                                                       & \begin{tabular}[c]{@{}l@{}}1. Compilación del código.\\ Ejecutable del módulo, hay muestras de que el \\ código hace algo\\ (algo: definido por los siguientes puntos).\\ 2.1 Verificación de integridad del archivo \\ conjunto final.\\ 2.2 Comprobar existencia del archivo.\\ 2.3 Revisión de la estructura de dicho archivo.\\ 2.3.1 Correcto almacenamiento del archivo.\end{tabular}                                                                                                                                                                                           \\ \hline
\textbf{Criterio de aceptación}                                                         & \begin{tabular}[c]{@{}l@{}}No hay errores que impidan la compilación \\ del código.\\ Al usar el ejecutable del módulo, hay muestras \\ de que el código hace algo\\ (algo: definido por los siguientes puntos).\\ 2.1 No se distorsiona el resultado previamente \\ adquirido, almacenado en el conjunto final.\\ 2.2 El sistema presenta mensajes de estatus \\ del archivo conjunto final.\\ 2.3 Correcto almacenado del \\ conjunto final.\end{tabular}                                                                                                                           \\ \hline
\textbf{Definición de verificaciones}                                                   & \begin{tabular}[c]{@{}l@{}}- Errores de Compilación: Ocurren porque la \\ sintaxis del lenguaje no es correcta, de cajón \\ este tipo de errores no permiten que la aplicación \\ se ejecute.\\ - Visualización de pantalla interfaz de usuario.\\ - Manejo de Archivos: Un programa no puede\\  manipular los datos de un archivo directamente. \\ Para usar un archivo, un programa siempre \\ abrir el archivo y asignarlo a una variable, \\ que llamaremos el archivo lógico. \\ Todas las operaciones sobre un archivo se \\ realizan a través del archivo lógico.\end{tabular} \\ \hline
\textbf{\begin{tabular}[c]{@{}l@{}}Análisis y \\ evaluación de resultados\end{tabular}} & - Resultados:                                                                                                                                                                                                                                                                                                                                                                                                                                                                                                                                                                         \\ \hline
\textbf{Productos  a entregar}                                                          & \begin{tabular}[c]{@{}l@{}}- Lectura y ordenamiento del conjunto final \\ funcionando correctamente.\end{tabular}                                                                                                                                                                                                                                                                                                                                                                                                                                                                     \\ \hline
\end{longtable}