% Please add the following required packages to your document preamble:
% \usepackage{longtable}
% Note: It may be necessary to compile the document several times to get a multi-page table to line up properly
\begin{longtable}{|l|l|}
\caption{Prueba unitaria RF2.3}
\label{PU_RF_2_3}\\
\hline
\textbf{Requerimiento Funcional}                                                        & \textbf{\begin{tabular}[c]{@{}l@{}}RF 2.3 Método de \textit{Machine Learning} para \\ optimización.\end{tabular}}                                                                                                                                                                                                                                                                                                                                                                                                                                                                                                                                                                                                                                                                                                                                                                                                                  \\ \hline
\endfirsthead
%
\multicolumn{2}{c}%
{{\bfseries Tabla \thetable\ Continuación de la página anterior}} \\
\endhead
%
\textbf{Perfiles Implicados}                                                            & \begin{tabular}[c]{@{}l@{}}- Desarrollador.\\ - Tester.\end{tabular}                                                                                                                                                                                                                                                                                                                                                                                                                                                                                                                                                                                                                                                                                                                                                                                                                                                      \\ \hline
\textbf{Planificación temporal}                                                         & \begin{tabular}[c]{@{}l@{}}1. Prueba de compilación.\\ 2. Prueba de funcionamiento.\\ 2.1 Verificar que el sistema posee las \\ variables contenedoras de los datos \\ obtenidos por el conjunto0 y el conjuntoP.\\ 2.1.1 Verificación de existencia correcta \\ de las variables para la estructura molecular \\ del compuesto.\\ 2.1.2 Verificación de existencia correcta de \\ las variables para la estructura molecular de \\ la(s) proteina(s).\\ 2.1.3 Verificación de existencia correcta de \\ las variables de los descriptores del \\ compuesto.\\ 2.1.4 Verificación de existencia correcta \\ de las variables de actividad biológica \\ del compuesto.\\ 2.2 Manejo de los datos para asignarlos a las \\ entradas del algoritmo de \textit{Machine Learning}.\\ 2.3  Revisión de cada una de las etapas del \\ procesamiento con \textit{Machine Learning}. \\ 2.4 Obtención del análisis y preprocesamiento.\end{tabular} \\ \hline
\textbf{Criterio de verificación}                                                       & \begin{tabular}[c]{@{}l@{}}1. Compilación del código.\\ 2. Ejecutable del módulo, hay muestras de que \\ el código hace algo\\ (algo: definido por los siguientes puntos).\\ 2.1 Verificación de cada una de las variables \\ que poseen datos del compuesto.\\ 2.1 Verificación de cada una de las variables \\ que poseen datos de la proteína.\\ 2.3 Asignación de entradas y variables al \\ algoritmo de \textit{Machine Learning}.\\ 2.4 Supervisión del continuo funcionamiento \\ de \textit{Machine Learning}.\\ 2.5 Revisión del resultado obtenido tras el \\ procesamiento de \textit{Machine Learning}.\end{tabular}                                                                                                                                                                                                                                                                                                                    \\ \hline
\textbf{Criterio de aceptación}                                                         & \begin{tabular}[c]{@{}l@{}}1. No existen fallas en la adquisición de \\ variables para el procesamiento de Machine \\ Learning.\\ 1.1 De existir una falla durante este proceso, \\ se informa al usuario. Al usar el ejecutable del \\ módulo, hay muestras de que el código hace algo\\ (algo: definido por los siguientes puntos).\\ 2.1. Correcto procesamiento de los datos por el \\ algoritmo de \textit{Machine Learning}.\\ 2.2 Visualización de mensajes de error durante \\ el procesamiento.\\ 2.3 Resultado coherente y conciso de acuerdo \\ a lo esperado.\\ 2.4 El sistema muestra el proceso en el que se \\ presentaron errores en caso de existir.\end{tabular}                                                                                                                                                                                                                                                 \\ \hline
\textbf{Definición de verificaciones}                                                   & \begin{tabular}[c]{@{}l@{}}- Errores de Compilación: Ocurren porque la \\ sintaxis del lenguaje no es correcta, de cajón \\ este tipo de errores no permiten que la aplicación \\ se ejecute.\end{tabular}                                                                                                                                                                                                                                                                                                                                                                                                                                                                                                                                                                                                                                                                                                                \\ \hline
\textbf{\begin{tabular}[c]{@{}l@{}}Análisis y \\ evaluación de resultados\end{tabular}} & - Resultados:                                                                                                                                                                                                                                                                                                                                                                                                                                                                                                                                                                                                                                                                                                                                                                                                                                                                                                             \\ \hline
\textbf{Productos  a entregar}                                                          & \begin{tabular}[c]{@{}l@{}}-Adquisición y descomposición del \\ conjuntoP funcionando correctamente.\end{tabular}                                                                                                                                                                                                                                                                                                                                                                                                                                                                                                                                                                                                                                                                                                                                                                                                         \\ \hline
\end{longtable}