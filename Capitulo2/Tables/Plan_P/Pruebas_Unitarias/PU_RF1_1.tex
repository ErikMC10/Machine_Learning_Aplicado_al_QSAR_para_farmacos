% Please add the following required packages to your document preamble:
% \usepackage{longtable}
% Note: It may be necessary to compile the document several times to get a multi-page table to line up properly
\begin{longtable}{|l|l|}
\caption{Prueba unitaria RF1.1}
\label{PU_RF1_1}\\
\hline

\textbf{Requerimiento Funcional}                                                       & \textbf{RF1.1 Adquisición del archivo base.}                                                                                                                                                                                                                                                                                                                                                                                                                                                                                                                                                                                                                                                     \\ \hline
\endfirsthead
%
\multicolumn{2}{c}%
{{\bfseries Tabla \thetable\ Continuación de la página anterior}} \\
\endhead
%
\textbf{Perfiles Implicados}                                                           & \begin{tabular}[c]{@{}l@{}}- Desarrollador.\\ - Tester.\end{tabular}                                                                                                                                                                                                                                                                                                                                                                                                                                                                                                                                                                                                                             \\ \hline
\textbf{Planificación temporal}                                                        & \begin{tabular}[c]{@{}l@{}}1. Prueba de compilación.\\ 2. Prueba de funcionamiento.\\ 2.1 Func. botón adquisición archivo.\\ 2.2 Muestra de pantalla selección de\\ archivo.\\ 2.3 Acceso a directorio.\\ 2.4 Carga de archivo.\\ 2.5 Func. botón buscar.\end{tabular}                                                                                                                                                                                                                                                                                                                                                                                                                           \\ \hline
\textbf{Criterio de verificación}                                                      & \begin{tabular}[c]{@{}l@{}}1. Compilación del código.\\ 2. Ejecutable del módulo, hay muestras\\ de que el código hace algo (algo: definido por los \\ siguientes puntos).\\ 2.1 Funcionamiento del botón de Archivo base.\\ 2.2 Visualización de selección de archivo.\\ 2.3 Acceso al directorio correcto o permite acceder \\ a uno distinto desde la pantalla de selección de \\ archivo.\\ 2.4 Carga del archivo correcto.\\ 2.5 El botón se ve adecuadamente y accede al\\ archivo indicado y logra una lectura correcta.\end{tabular}                                                                                                                                                     \\ \hline
\textbf{Criterio de aceptación}                                                        & \begin{tabular}[c]{@{}l@{}}1. No hay errores que impidan la compilación \\ del código.\\ 2. Al usar el ejecutable del módulo, hay muestras \\ de que el código hace algo \\ (algo: definido por los siguientes puntos).\\ 2.1 Se muestra el botón correspondiente, y realiza la \\ acción indicada.\\ 2.2 Es visualizada la pantalla de selección de archivos \\ con cada uno de los componentes.\\ 2.3 La selección de archivos se encuentra en el \\ directorio correcto o permite acceder a uno distinto.\\ 2.4 Solo permite la carga de un archivo de texto \\ plano, y es cargado adecuadamente.\\ 2.5 El botón se ve adecuada, y realiza la acción \\ indicada correctamente.\end{tabular} \\ \hline
\textbf{\begin{tabular}[c]{@{}l@{}}Definición de\\ verificaciones\end{tabular}}        & \begin{tabular}[c]{@{}l@{}}- Errores de Compilación: Ocurren porque la sintaxis \\ del lenguaje no es correcta, de cajón este tipo de \\ errores no permiten que la aplicación se ejecute. \\ \\ - Acción de botón: representa un botón que, cuando \\ es presionado, envía información al que pertenece. \\ La función de  un botón representada  el contenido\\ del elemento.\\ \\ -Visualización de pantalla interfaz de usuario.\end{tabular}                                                                                                                                                                                                                                                \\ \hline
\textbf{\begin{tabular}[c]{@{}l@{}}Análisis y evaluación\\ de resultados\end{tabular}} &                                                                                                                                                                                                                                                                                                                                                                                                                                                                                                                                                                                                                                                                                                  \\ \hline
\textbf{Productos  a entregar}                                                         & \begin{tabular}[c]{@{}l@{}}- Adquisición del archivo base funcionando \\ correctamente.\end{tabular}                                                                                                                                                                                                                                                                                                                                                                                                                                                                                                                                                                                             \\ \hline

\end{longtable}