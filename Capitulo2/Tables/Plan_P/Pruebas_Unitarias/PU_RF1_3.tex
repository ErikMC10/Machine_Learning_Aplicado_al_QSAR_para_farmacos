% Please add the following required packages to your document preamble:
% \usepackage{longtable}
% Note: It may be necessary to compile the document several times to get a multi-page table to line up properly
\begin{longtable}{|l|l|}
\caption{Prueba unitaria RF1.3}
\label{PU_RF1_3}\\
\hline
\textbf{Requerimiento Funcional}                                                       & \textbf{\begin{tabular}[c]{@{}l@{}}RF1.3 Búsqueda de los descriptores de los \\ compuestos.\end{tabular}}                                                                                                                                                                                                                                                                                                                                                                                                                                                                                                                                                                                                                                                                                                                                                                                      \\ \hline
\endfirsthead
%
\multicolumn{2}{c}%
{{\bfseries Tabla \thetable\ Continuación de la página anterior}} \\
\endhead
%
\textbf{Perfiles Implicados}                                                           & \begin{tabular}[c]{@{}l@{}}- Desarrollador.\\ - Tester.\end{tabular}                                                                                                                                                                                                                                                                                                                                                                                                                                                                                                                                                                                                                                                                                                                                                                                                                           \\ \hline
\textbf{Planificación temporal}                                                        & \begin{tabular}[c]{@{}l@{}}1. Prueba de compilación.\\ 2. Prueba de funcionamiento.\\ 2.1 Verificar que el sistema mantiene el nombre \\ del compuesto sobre del que están obteniendo \\ los datos.\\ 2.2 Comprobar conexión a la base de datos \\ “ChemSpider” y “PubChem”.\\ 2.3 Búsqueda del compuesto en las bases de \\ datos.\\ 2.3.1  Existe el compuesto con el nombre \\ exacto en “PubChem”. \\ 2.3.2 Existe el compuesto con el nombre \\ exacto en “ChemSpider”.\\ 2.4 Adquisición de los descriptores del \\ compuesto en las bases de datos.\\ 2.3.1  Se adquieren  los descriptores  existentes en \\ “PubChem” del compuesto. \\ 2.3.2 Se adquieren  los descriptores  existentes en \\ “ChemSpider” del compuesto.\\ 2.4 Creación del archivo “Descriptors”.\\ 2.5 Almacenado de los datos adquiridos en \\ el archivo correspondiente , recientemente creado.\end{tabular} \\ \hline
\textbf{Criterio de verificación}                                                      & \begin{tabular}[c]{@{}l@{}}1. Compilación del código.\\ 2. Ejecutable del módulo, hay muestras de que el \\ código hace algo\\ (algo: definido por los siguientes puntos).\\ 2.1Verificación de integridad de la variable que \\ contiene el nombre del compuesto.\\ 2.2 Conexión a  “ChemSpider” y “PubChem” \\ y por lo tanto a Internet.\\ 2.3 Adecuada búsqueda del del compuesto exacto \\ en la base de datos.\\ 2.3.1 Correcta adquisición de los descriptores en\\ “PubChem”.\\ 2.3.2 Correcta adquisición de los descriptores en\\ “ChemSpider”.\\ 2.4 Creación del archivo “Descriptors” \\ correspondiente con el compuesto.\\ 2.5 Corroborar guardado de la información \\ obtenida en el archivo creado.\end{tabular}                                                                                                                                                          \\ \hline
\textbf{Criterio de aceptación}                                                        & \begin{tabular}[c]{@{}l@{}}1. No hay errores que impidan la compilación \\ del código.\\ 2. Al usar el ejecutable del módulo, hay muestras \\ de que el código hace algo\\ (algo: definido por los siguientes puntos).\\ 2.1 No se distorsiona el nombre del compuesto \\ de interés.\\ 2.2 El sistema informa la correcta conexión a\\ “ChemSpider”.\\ 2.3 El sistema informa la correcta conexión a \\ “PubChem”.\\ 2.4 El sistema informa que el mismo compuesto \\ existe en ambos repositorios.\\ 2.5. Se notifica la adecuada obtención de los \\ descriptores del compuesto.\\ 2.6 El sistema informa la correcta creación del\\  archivo.\\ 2.7 El archivo y la información que contiene es \\ integra.\end{tabular}                                                                                                                                                                   \\ \hline
\textbf{\begin{tabular}[c]{@{}l@{}}Definición de\\ verificaciones\end{tabular}}        & \begin{tabular}[c]{@{}l@{}}- Errores de Compilación: Ocurren porque la \\ sintaxis del lenguaje no es correcta, de cajón \\ este tipo de errores no permiten que la \\ aplicación se ejecute. \\ \\ - Conexión a base de datos online: \\ Una conexión a base de datos es un archivo \\ de configuración donde se especifica los \\ detalles físicos de una base de datos como por \\ ejemplo el tipo de base de datos y la versión, y \\ los parámetros que permiten una conexión\end{tabular}                                                                                                                                                                                                                                                                                                                                                                                                \\ \hline
\textbf{\begin{tabular}[c]{@{}l@{}}Análisis y evaluación\\ de resultados\end{tabular}} & - Resultados:                                                                                                                                                                                                                                                                                                                                                                                                                                                                                                                                                                                                                                                                                                                                                                                                                                                                                  \\ \hline
\textbf{Productos  a entregar}                                                         & \begin{tabular}[c]{@{}l@{}}- Búsqueda de los descriptores correspondientes\\  al compuesto de interés.\end{tabular}                                                                                                                                                                                                                                                                                                                                                                                                                                                                                                                                                                                                                                                                                                                                                                            \\ \hline
\end{longtable}