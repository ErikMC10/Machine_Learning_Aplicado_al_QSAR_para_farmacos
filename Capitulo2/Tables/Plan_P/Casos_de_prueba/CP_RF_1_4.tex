% Please add the following required packages to your document preamble:
% \usepackage{longtable}
% Note: It may be necessary to compile the document several times to get a multi-page table to line up properly
\begin{longtable}{|l|l|}
\caption{Caso de prueba RF1.4}
\label{CP_RF_4}\\
\hline
\textbf{ID del Caso de prueba}                                                          & CPRF4                                                                                                                                                                                                                                                                                                                         \\ \hline
\endfirsthead
%
\multicolumn{2}{c}%
{{\bfseries Tabla \thetable\ Continuación de la página anterior}} \\
\endhead
%
\textbf{Versión}                                                                        & 1.0                                                                                                                                                                                                                                                                                                                           \\ \hline
\textbf{Nombre}                                                                         & \begin{tabular}[c]{@{}l@{}}Caso de prueba para la búsqueda de los \\ mecanismos de acción de los compuestos.\end{tabular}                                                                                                                                                                                                             \\ \hline
\textbf{\begin{tabular}[c]{@{}l@{}}Identificador de \\ requerimiento\end{tabular}}      & RF1.4                                                                                                                                                                                                                                                                                                                         \\ \hline
\textbf{Propósito}                                                                      & \begin{tabular}[c]{@{}l@{}}Identificar el funcionamiento del sistema al \\ momento de re-conectarse o mantenerse conectado\\ a la base de datos contenedora de los datos \\ cuantitativos representantes de la actividad \\ biológica de los compuestos, “DrugBank”.\end{tabular}                                            \\ \hline
\textbf{Dependencias}                                                                   & \begin{tabular}[c]{@{}l@{}}- Correcta obtención del archivo base.\\ - Correcta adquisición de la estructura molecular \\ del compuesto.\\ - Adecuada conexión de la base de datos \\ “DrugBank”.\end{tabular}                                                                                                                 \\ \hline
\textbf{\begin{tabular}[c]{@{}l@{}}Ambiente de \\ prueba/configuración\end{tabular}}    & \begin{tabular}[c]{@{}l@{}}- Hardware: Equipo de computo\\ (preferentemente portatíl)\\ - Software: Compilador python3, \\ IDE y/o editor de texto.\end{tabular}                                                                                                                                                              \\ \hline
\textbf{Inicialización}                                                                 & \begin{tabular}[c]{@{}l@{}}- Codificación correspondiente al \\ requerimiento.\\ - Creación del archivo base.\end{tabular}                                                                                                                                                                                                    \\ \hline
\textbf{Finalización}                                                                   & N/A                                                                                                                                                                                                                                                                                                                           \\ \hline
\textbf{Acciones}                                                                       & \begin{tabular}[c]{@{}l@{}}. Compilar el código correspondiente.\\ - Contar con el archivo base previamente \\ cargado.\end{tabular}                                                                                                                                                                                          \\ \hline
\textbf{\begin{tabular}[c]{@{}l@{}}Descripción de los \\ datos de entrada\end{tabular}} & \begin{tabular}[c]{@{}l@{}}- Nombre del compuesto.\\ - Dirección para la conexión a “DrugBank”.\end{tabular}                                                                                                                                                                                                                  \\ \hline
\textbf{Salida esperada}                                                                & \begin{tabular}[c]{@{}l@{}}- Notificación de adecuada estado del \\ compuesto en “DrugBank”. (Existente o no).\\ - De existir, informar la adecuada obtención \\ de los datos que conforman la actividad \\ biológica del compuesto.\\ - Informar correcta creación del archivo \\ “BioActivity” y su contenido.\end{tabular} \\ \hline
\textbf{Salida obtenida}                                                                &                                                                                                                                                                                                                                                                                                                               \\ \hline
\textbf{Resultado}                                                                      &                                                                                                                                                                                                                                                                                                                               \\ \hline
\textbf{Severidad}                                                                      &                                                                                                                                                                                                                                                                                                                               \\ \hline
\textbf{Evidencia}                                                                      &                                                                                                                                                                                                                                                                                                                               \\ \hline
\textbf{Estado}                                                                         & No Iniciado.                                                                                                                                                                                                                                                                                                                  \\ \hline
\end{longtable}