% Please add the following required packages to your document preamble:
% \usepackage{longtable}
% Note: It may be necessary to compile the document several times to get a multi-page table to line up properly
\begin{longtable}{|l|l|l|l|l|l|}
\caption{Requerimientos funcionales}
\label{RF}\\
\hline
No & Concepto & Requerimiento & N\textbackslash{}D & Descripción & Naturaleza \\ \hline
\endfirsthead
%
\multicolumn{5}{c}%
{{\bfseries Tabla \thetable\ Continuación de la página anterior}} \\
\endhead
%
1 & Función & \begin{tabular}[c]{@{}l@{}}Búsqueda de\\  información\end{tabular} & Necesidad & \begin{tabular}[c]{@{}l@{}}La búsqueda de\\ información\\ referente a los\\ fármacos, sus\\ propiedades y \\ la patología\\ especificadas, \\ se realiza \\ utilizando las \\ bases de datos\\ especializadas\\ en compuestos,\\ descriptores\\ moleculares,\\ tablas de\\ propiedades\\ físico-químicas \\ y bancos de\\ proteínas.\end{tabular} & Cualitativa \\ \hline
1.1 & Función & \begin{tabular}[c]{@{}l@{}}Obtención y\\ lectura del\\ archivo base\end{tabular} & Necesidad & \begin{tabular}[c]{@{}l@{}}El sistema\\ recibe del\\ usuario un\\ archivo de\\ texto plano en\\ el cual se tiene\\ una lista con\\ los nombres\\ de los\\ compuestos y\\ nombres de\\ las proteínas\\ de la patología\\ que forman\\ parte del\\ objeto de\\ estudio.\end{tabular} & Cualitativa \\ \hline
1.2 & Función & \begin{tabular}[c]{@{}l@{}}Búsqueda\\ de los\\ compuestos\\ indicados.\end{tabular} & Necesidad & \begin{tabular}[c]{@{}l@{}}Por cada\\ elemento de la\\ lista de\\ compuestos en\\ el archivo\\ ingresado por\\ el usuario, el\\ sistema los\\ buscará en la\\ base de datos\\ en línea\\ “DrugBank”.\end{tabular} & Cualitativa \\ \hline
1.3 & Función & \begin{tabular}[c]{@{}l@{}}Búsqueda de\\ los\\ descriptores\\ de los\\ compuestos.\end{tabular} & Necesidad & \begin{tabular}[c]{@{}l@{}}El sistema,\\ utilizando\\ como\\ referencia la\\ lista de\\ compuestos\\ ingresada por\\ el usuario,\\ realiza una\\ búsqueda de los\\ \\ descriptores\\ físico químicos\\ por cada uno\\ de los elementos\\ de dicha lista en\\ las bases de\\ datos\\ “PubChem” y\\ “ChemSpider”.\end{tabular} & Cualitativa \\ \hline
1.4 & Función & \begin{tabular}[c]{@{}l@{}}Búsqueda de\\ los\\ mecanismos de\\ acción de los\\ compuestos.\end{tabular} & Necesidad & \begin{tabular}[c]{@{}l@{}}El sistema,\\ utilizando\\ como\\ referencia la\\ lista de\\ compuestos\\ ingresada por\\ el usuario,\\ realiza la\\ búsqueda de\\ las actividades\\ biológicas por\\ cada uno de\\ los elementos\\ de dicha lista\\ en la base de\\ datos\\ “DrugBank”.\end{tabular} & Cualitativa \\ \hline
1.5 & Función & \begin{tabular}[c]{@{}l@{}}Búsqueda de\\ las proteínas\\ indicadas.\end{tabular} & Necesidad & \begin{tabular}[c]{@{}l@{}}Por cada\\ elemento de la\\ lista de\\ proteínas en el\\ archivo\\ ingresado por\\ el usuario, el\\ sistema las\\ buscará en la\\ base de datos\\ en línea\\ “PDB”.\end{tabular} & Cualitativa \\ \hline
1.6 & Función & \begin{tabular}[c]{@{}l@{}}Confirmación\\ de los\\ resultados de\\ la búsqueda.\end{tabular} & Necesidad & \begin{tabular}[c]{@{}l@{}}El sistema\\ presenta al\\ usuario los\\ resultados de\\ la búsqueda y\\ este último\\ valida que\\ sean\\ correctos.\end{tabular} & Cualitativa \\ \hline
1.7 & Función & \begin{tabular}[c]{@{}l@{}}Construcción\\ de conjuntos.\end{tabular} & Necesidad & \begin{tabular}[c]{@{}l@{}}El sistema\\ organiza la\\ información\\ recabada\\ durante el\\ proceso de\\ búsqueda, en\\ archivos con\\ estructuras\\ definidas,\\ denominados\\ conjuntos, que\\ permiten\\ continuar a la\\ fase de\\ procesamiento\\ de\\ información.\end{tabular} & Cualitativa \\ \hline
1.7.1 & Función & \begin{tabular}[c]{@{}l@{}}Construcción\\ del conjunto\\ cero.\end{tabular} & Necesidad & \begin{tabular}[c]{@{}l@{}}El sistema\\ agrupa los\\ datos de los\\ compuestos en\\ un archivo\\ denominado\\ conjunto cero.\end{tabular} & Cualitativa \\ \hline
1.7.2 & Función & \begin{tabular}[c]{@{}l@{}}Construcción\\ del conjunto P\end{tabular} & Necesidad & \begin{tabular}[c]{@{}l@{}}El sistema\\ agrupa la\\ información de\\ la(s)\\ proteína(s)\\ (nombre de la\\ proteína,\\ estructura\\ molecular de la\\ proteína) de la\\ patología\\ indicada por el\\ usuario, en un\\ archivo\\ denominado\\ conjunto P.\end{tabular} & Cualitativa \\ \hline
2 & Función & \begin{tabular}[c]{@{}l@{}}Procesamiento \\ y obtención de\\ resultados.\end{tabular} & Necesidad & \begin{tabular}[c]{@{}l@{}}El sistema\\ obtiene los\\ datos tanto de \\ los compuestos\\ como de la(s)\\ proteína(s) a\\ partir de su \\ conjunto de\\ información\\ correspondientes.\end{tabular} & Cualitativa \\ \hline
2.1 & Función & \begin{tabular}[c]{@{}l@{}}Adquisición y \\ descomposición\\  del conjunto 0.\end{tabular} & Necesidad & \begin{tabular}[c]{@{}l@{}}El sistema lee el\\ conjunto 0\\ (previamente\\ definido)\\ correspondiente\\ a un compuesto,\\ para\\ descomponerlo y \\ obtener nombre,\\ descriptores ,\\ estructura\\ molecular y\\ actividad\\ biológica.\end{tabular} & Cualitativa \\ \hline
2.2 & Función & \begin{tabular}[c]{@{}l@{}}Adquisición y\\ descomposición\\ del conjunto P.\end{tabular} & Necesidad & \begin{tabular}[c]{@{}l@{}}El sistema\\ consigue el\\ nombre de la\\ proteína de\\ interés así como\\ su estructura\\ molecular esta\\ información\\ proveniente del\\ conjunto P.\end{tabular} & Cualitativa \\ \hline
2.3 & Función & \begin{tabular}[c]{@{}l@{}}Método de\\ Machine\\ Learning para\\ optimización.\end{tabular} & Necesidad & \begin{tabular}[c]{@{}l@{}}El sistema,   a\\ través de este\\ módulo trabaja \\ con los datos\\ correspondientes\\ al compuesto  y\\ proteína con el\\ objetivo de\\ optimizar la\\ obtención de\\ resultados.\end{tabular} & Cualitativa \\ \hline
2.4 & Función & \begin{tabular}[c]{@{}l@{}}Modelado \\ QSAR.\end{tabular} & Necesidad & \begin{tabular}[c]{@{}l@{}}El sistema\\ producirá con los\\ datos\\ pertenecientes a\\ la proteína y \\ compuesto de\\ interés  el\\ modelo QSAR\\ propio de la\\ actividad\\ farmacológica\\ entre el fármaco\\ y la proteína\\ objetivo.\end{tabular} & Cualitativa \\ \hline
3 & Función & \begin{tabular}[c]{@{}l@{}}Administración\\ y presentación\\ de resultados-\end{tabular} & Necesidad & \begin{tabular}[c]{@{}l@{}}El sistema recibe\\ el conjunto final\\ que está\\ constituido del\\ nombre del\\ compuesto y su\\ aptitud, datos\\ necesarios para\\ ordenar y\\ mostrar los\\ resultados.\end{tabular} & Cualitativa \\ \hline
3.1 & Función & \begin{tabular}[c]{@{}l@{}}Lectura y\\ ordenamiento\\ del conjunto\\ final.\end{tabular} & Necesidad & \begin{tabular}[c]{@{}l@{}}El sistema recibe\\ el conjunto final\\ que contiene el\\ nombre del\\ compuesto y la \\ aptitud, dicha\\ información será\\ leída y ordena\\ del más apto al\\ menos apto para\\ enfrentar la\\ proteína de la \\ patología\\ indicada. A la\\ salida se tiene\\ un archivo de\\ texto plano que\\ contiene una\\ lista ordenada.\end{tabular} & Cualitativa \\ \hline
3.2 & Función & \begin{tabular}[c]{@{}l@{}}Diseño gráfico\\ de resultados.\end{tabular} & Necesidad & \begin{tabular}[c]{@{}l@{}}El sistema\\ obtiene el\\ archivo que\\ contiene la lista\\ ordenada  la cual\\ será utilizada\\ para desplegar\\ de manera\\ gráfica  al\\ usuario.\end{tabular} & Cualitativa \\ \hline
\end{longtable}