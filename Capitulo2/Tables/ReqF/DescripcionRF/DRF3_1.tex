% Please add the following required packages to your document preamble:
% \usepackage{longtable}
% Note: It may be necessary to compile the document several times to get a multi-page table to line up properly
\begin{longtable}{|l|l|}
\caption{Requerimiento funcional 3.1}
\label{RF3_1}\\
\hline
\begin{tabular}[c]{@{}l@{}}Nombre del requerimiento funcional:\\ Lectura y ordenamiento del conjunto final\end{tabular}                                                                                                                                                           & ID: RF3.1                                                                    \\ \hline
\endfirsthead
%
\multicolumn{2}{c}%
{{\bfseries Tabla \thetable\ Continuación de la página anterior}} \\
\endhead
%
\multicolumn{2}{|l|}{Área: Muestra de resultados finales.}                                                                                                                                                                                                                                                                                                       \\ \hline
\multicolumn{2}{|l|}{Actor(es): Compuestos óptimos resultantes.}                                                                                                                                                                                                                                                                                                 \\ \hline
\multicolumn{2}{|l|}{Interesados: Sistema.}                                                                                                                                                                                                                                                                                                                      \\ \hline
\multicolumn{2}{|l|}{\begin{tabular}[c]{@{}l@{}}Descripción: El sistema recibe el conjunto final que contiene el nombre del\\ compuesto y la  aptitud, dicha información será leída y ordena del más apto\\ al menos apto para enfrentar la proteína de la  patología indicada. A la salida\\ se tiene un archivo que contiene una lista ordenada.\end{tabular}} \\ \hline
\multicolumn{2}{|l|}{\begin{tabular}[c]{@{}l@{}}Evento desencadenador: Tras haber realizado los respectivos cálculos\\ entre los efectos de cada uno de los compuestos de interés de evaluacion\\ del sistema.\end{tabular}}                                                                                                                                     \\ \hline
\multicolumn{2}{|l|}{Tipo de desencadenador: Temporal.}                                                                                                                                                                                                                                                                                                          \\ \hline
Pasos realizados (ruta principal)                                                                                                                                                                                                                                                 & Información para los pasos                                                   \\ \hline
\begin{tabular}[c]{@{}l@{}}1.- El sistema recibe el conjunto final, dicho\\ conjunto está constituido por el nombre del\\ compuesto y su aptitud.\end{tabular}                                                                                                                    & No requiere información.                                                     \\ \hline
\begin{tabular}[c]{@{}l@{}}2.-  El sistema hace una lectura del conjunto y\\ comienza el ordenamiento de los compuestos\\ usando la aptitud como métrica. El\\ ordenamiento  se dará del más apto al menos\\ apto.\end{tabular}                                                   & Conjunto final.                                                              \\ \hline
\begin{tabular}[c]{@{}l@{}}3.- El sistema envía el archivo ordenado al\\ módulo de graficación.\end{tabular}                                                                                                                                                                      & No requiere información.                                                     \\ \hline
\multicolumn{2}{|l|}{Extensiones o Rutas alternativas.}                                                                                                                                                                                                                                                                                                          \\ \hline
\multicolumn{1}{|c|}{N/A}                                                                                                                                                                                                                                                           & \multicolumn{1}{c|}{N/A}                                                       \\ \hline
\multicolumn{2}{|l|}{\begin{tabular}[c]{@{}l@{}}Precondiciones: El cálculo de las actividades e interacciones entre\\ compuestos y patología se realizó con éxito.\end{tabular}}                                                                                                                                                                                 \\ \hline
\multicolumn{2}{|l|}{\begin{tabular}[c]{@{}l@{}}Postcondiciones: El sistema hace la lectura correcta del compuesto y logra\\ ordenar con éxito las aptitudes los mismos, se hace la entrega correcta del\\ archivo ordenado al módulo de graficación.\end{tabular}}                                                                                              \\ \hline
\multicolumn{2}{|l|}{\begin{tabular}[c]{@{}l@{}}Suposiciones: El cálculo de los compuestos se realizó con éxito y el\\ sistema recibe el conjunto final sin errores.\end{tabular}}                                                                                                                                                                               \\ \hline
\multicolumn{2}{|l|}{\begin{tabular}[c]{@{}l@{}}Garantía de éxito: Se lee y ordena el conjunto final de manera correcta sin\\ generar errores.\end{tabular}}                                                                                                                                                                                                     \\ \hline
\multicolumn{2}{|l|}{\begin{tabular}[c]{@{}l@{}}Garantía mínima: Se lee el conjunto final y se pasa al siguiente módulo de\\ manera correcta.\end{tabular}}                                                                                                                                                                                                      \\ \hline
\multicolumn{2}{|l|}{\begin{tabular}[c]{@{}l@{}}Requerimientos cumplidos: Validación de los datos obtenidos de los\\ compuestos y patología; Cálculo de las aptitudes con éxito.\end{tabular}}                                                                                                                                                                   \\ \hline
\multicolumn{2}{|l|}{\begin{tabular}[c]{@{}l@{}}Cuestiones pendientes: Considerar el formato de muestra de resultados\\ obtenidos por el sistema.\end{tabular}}                                                                                                                                                                                                  \\ \hline
\multicolumn{2}{|l|}{Prioridad: Alta.}                                                                                                                                                                                                                                                                                                                           \\ \hline
\multicolumn{2}{|l|}{Riesgo: Alto.}                                                                                                                                                                                                                                                                                                                              \\ \hline
\end{longtable}