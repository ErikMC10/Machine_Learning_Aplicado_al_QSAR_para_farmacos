% Please add the following required packages to your document preamble:
% \usepackage{longtable}
% Note: It may be necessary to compile the document several times to get a multi-page table to line up properly
\begin{longtable}{|l|l|}
\caption{Requerimiento funcional 1.1}
\label{RF11}\\
\hline
\begin{tabular}[c]{@{}l@{}}Nombre del requerimiento funcional:\\ Adquisición del archivo base.\end{tabular}                                                                                                                                                                               & ID: RF1.1                                                                                                                                     \\ \hline
\endfirsthead
%
\multicolumn{2}{c}%
{{\bfseries Tabla \thetable\ Continuación de la página anterior}} \\
\endhead
%
\multicolumn{2}{|l|}{Área: Búsqueda de información.}                                                                                                                                                                                                                                                                                                                                                                                      \\ \hline
\multicolumn{2}{|l|}{Actor(es): Usuario del sistema.}                                                                                                                                                                                                                                                                                                                                                                                     \\ \hline
\multicolumn{2}{|l|}{Interesados: Sistema.}                                                                                                                                                                                                                                                                                                                                                                                               \\ \hline
\multicolumn{2}{|l|}{Descripción: Permitir al usuario cargar un archivo de texto plano a la aplicación.}                                                                                                                                                                                                                                                                                                                                  \\ \hline
\multicolumn{2}{|l|}{\begin{tabular}[c]{@{}l@{}}Evento desencadenador: El usuario, ya estando en la pantalla inicial del sistema,\\ da clic en el botón que contiene un icono de suma “+” para proceder a cargar\\ un archivo. \end{tabular}}                                                                                                                                                                                                                                                                                                                               \\ \hline
\multicolumn{2}{|l|}{Tipo de desencadenador: Externo.}                                                                                                                                                                                                                                                                                                                                                                                    \\ \hline
Pasos realizados (ruta principal)                                                                                                                                                                                                                                                         & Información para los pasos                                                                                                                    \\ \hline
\begin{tabular}[c]{@{}l@{}}1.- El usuario presiona el botón para cargar un\\ archivo al sistema.\end{tabular}                                                                                                                                                                             & \begin{tabular}[c]{@{}l@{}}No se requiere información para\\  este paso.\end{tabular}                                                         \\ \hline
\begin{tabular}[c]{@{}l@{}}2.- El sistema abre una ventana con los\\ directorios del ordenador donde se encuentra\\ el usuario, para que este último proceda a la\\ búsqueda del archivo que desea cargar.\end{tabular}                                                                   & \begin{tabular}[c]{@{}l@{}}No se requiere información para\\  este paso.\end{tabular}                                                         \\ \hline
\begin{tabular}[c]{@{}l@{}}3.- El usuario selecciona el archivo que desea\\ cargar (el sistema por defecto solo admite el\\ formato de texto plano para el archivo que va\\ a recibir), y presiona el botón “Abrir”.\end{tabular}                                                         & \begin{tabular}[c]{@{}l@{}}Archivo que se desea cargar al \\ sistema.\end{tabular}                                                                             \\ \hline
\multicolumn{2}{|l|}{\begin{tabular}[c]{@{}l@{}}4.- El usuario hace clic en el botón “Buscar”\\ para proceder a la lectura del archivo\\ ingresado.\end{tabular}}                                                                                                                                             &  \begin{tabular}[c]{@{}l@{}}No se requiere información para\\ este paso.\end{tabular}                                                                                 \\ \hline
\multicolumn{2}{|l|}{Extensiones o Rutas alternativas.}                                                                                                                                                                                                                                                                                                                                                                                   \\ \hline
\multicolumn{1}{|c|}{N/A}                                                                                                                                                                                                                                                                   & \multicolumn{1}{c|}{N/A}                                                                                                                        \\ \hline
\multicolumn{2}{|l|}{\begin{tabular}[c]{@{}l@{}}Precondiciones: El usuario ingresa un archivo de texto plano, el archivo cumple\\ con la estructura indicada para el archivo base (el cual se encuentra\\ definido en el diccionario de datos con el ID: AB) para su correcta lectura.\end{tabular}}                                                                                                                                                                                    \\ \hline
\multicolumn{2}{|l|}{\begin{tabular}[c]{@{}l@{}}Postcondiciones: El sistema ha podido leer y registrar los datos en el archivo\\ ingresado por el usuario.\end{tabular}}                                                                                                                                                                                                                                                                  \\ \hline
\multicolumn{2}{|l|}{\begin{tabular}[c]{@{}l@{}}Suposiciones: El sistema puede recibir y leer archivos de texto plano, el archivo\\ ingresado por el usuario es válido tanto en formato como en estructura. Si el\\ archivo no es de texto plano, o no cumple con la estructura que solicita el sistema,\\ no se procederá con la ejecución de la aplicación y se notificará al usuario de que\\ el archivo no pudo leerse.\end{tabular}} \\ \hline
\multicolumn{2}{|l|}{\begin{tabular}[c]{@{}l@{}}Garantía de éxito: El sistema obtiene la información necesaria para el modelado\\ QSAR utilizando como referencia la información contenida en el archivo que el\\ usuario ingresó.\end{tabular}}                                                                                                                                                                                          \\ \hline
\multicolumn{2}{|l|}{Garantía mínima: No hay garantía mínima.}                                                                                                                                                                                                                                                                                                                                                                            \\ \hline
\multicolumn{2}{|l|}{Requerimientos cumplidos: Adquisición del archivo base.}                                                                                                                                                                                                                                                                                                                                                             \\ \hline
\multicolumn{2}{|l|}{Prioridad: Alta.}                                                                                                                                                                                                                                                                                                                                                                                                    \\ \hline
\multicolumn{2}{|l|}{Riesgo: Medio.}                                                                                                                                                                                                                                                                                                                                                                                                      \\ \hline
\end{longtable}