% Please add the following required packages to your document preamble:
% \usepackage{longtable}
% Note: It may be necessary to compile the document several times to get a multi-page table to line up properly
\begin{longtable}{|l|l|}
\caption{Requerimiento funcional 2.3}
\label{RF2_3}\\
\hline
\begin{tabular}[c]{@{}l@{}}Nombre del requerimiento funcional:\\ Método de Machine Learning para \\ optimización.\end{tabular}                                    & ID: RF2.3                                                                                                                                                                                                               \\ \hline
\endfirsthead
%
\multicolumn{2}{c}%
{{\bfseries Tabla \thetable\ Continuación de la página anterior}} \\
\endhead
%
\multicolumn{2}{|l|}{Área: Procesamiento de datos  y obtención de resultados.}                                                                                                                                                                                                                                                                                                              \\ \hline
\multicolumn{2}{|l|}{Actor(es): Sistema.}                                                                                                                                                                                                                                                                                                                                                   \\ \hline
\multicolumn{2}{|l|}{Interesados: Sistema.}                                                                                                                                                                                                                                                                                                                                                 \\ \hline
\multicolumn{2}{|l|}{\begin{tabular}[c]{@{}l@{}}Descripción: Este módulo tiene definido el otro método de Machine Learning,\\ para identificar cálculos previos y optimizar la obtención de resultados.\end{tabular}}                                                                                                                                                                       \\ \hline
\multicolumn{2}{|l|}{\begin{tabular}[c]{@{}l@{}}Evento desencadenador: El sistema adquirió exitosamente los datos\\ provenientes del conjunto 0 y conjunto P.\end{tabular}}                                                                                                                                                                                                                 \\ \hline
\multicolumn{2}{|l|}{Tipo de desencadenador: Temporal.}                                                                                                                                                                                                                                                                                                                                     \\ \hline
Pasos realizados (ruta principal)                                                                                                                                 & Información para los pasos                                                                                                                                                                                              \\ \hline
\begin{tabular}[c]{@{}l@{}}1.Se reciben los datos petenecientes del\\ compuesto obtenidos del conjunto 0 \\ y los de la proteína del conjunto P.\end{tabular}     & \begin{tabular}[c]{@{}l@{}}- Nombre Compuesto\\ - Estructura\\ molecular(compuesto).\\ - Descriptores Físico\\ Químicos.\\ - Actividad Biológica\\ - Nombre proteína\\ - Estructura\\ molecular(proteína).\end{tabular} \\ \hline
\begin{tabular}[c]{@{}l@{}}2.- El sistema realiza procesamiento con los\\ datos para refinar la información a ser\\ empleada en el siguiente módulo.\end{tabular} & No requiere información.                                                                                                                                                                                                \\ \hline
\begin{tabular}[c]{@{}l@{}}3.- Son transferidos los datos pre-procesados\\ al módulo de modelado QSAR.\end{tabular}                                               & No requiere información.                                                                                                                                                                                                \\ \hline
\multicolumn{2}{|l|}{Extensiones o Rutas alternativas.}                                                                                                                                                                                                                                                                                                                                     \\ \hline
\begin{tabular}[c]{@{}l@{}}1.1 Los datos no se obtuvieron\\ adecuadamente.\\ 2.1 Se generó un error en el procesamiento\\ de Machine Learning.\end{tabular}       & \begin{tabular}[c]{@{}l@{}}- Se notifica un error en la\\ adquisición de datos.\\ - Se le informa al usuario con\\ qué dato(s) se provocó la\\ falla.\end{tabular}                                                      \\ \hline
\multicolumn{2}{|l|}{\begin{tabular}[c]{@{}l@{}}Precondiciones: Adecuado estado de los datos existentes de compuesto y\\ proteína.\end{tabular}}                                                                                                                                                                                                                                            \\ \hline
\multicolumn{2}{|l|}{\begin{tabular}[c]{@{}l@{}}Postcondiciones: El sistema identifica cómo abordar la información para la\\ obtención del modelado QSAR(módulo siguiente).\end{tabular}}                                                                                                                                                                                                   \\ \hline
\multicolumn{2}{|l|}{\begin{tabular}[c]{@{}l@{}}Suposiciones: Existe una adecuada administración de toda la información\\ existente del compuesto y proteína.\end{tabular}}                                                                                                                                                                                                                 \\ \hline
\multicolumn{2}{|l|}{\begin{tabular}[c]{@{}l@{}}Garantía de éxito: El sistema establece la manera en que se trabajara con la\\ información en el módulo de “Modelado QSAR”.\end{tabular}}                                                                                                                                                                                                   \\ \hline
\multicolumn{2}{|l|}{Garantía mínima: Los datos son adquiridos por el módulo.}                                                                                                                                                                                                                                                                                                              \\ \hline
\multicolumn{2}{|l|}{Requerimientos cumplidos: Método de Machine Learning para optimización.}                                                                                                                                                                                                                                                                                               \\ \hline
\multicolumn{2}{|l|}{Cuestiones pendientes: Algoritmo a implementar en el módulo.}                                                                                                                                                                                                                                                                                                          \\ \hline
\multicolumn{2}{|l|}{Prioridad: Alta.}                                                                                                                                                                                                                                                                                                                                                      \\ \hline
\multicolumn{2}{|l|}{Riesgo: Alto.}                                                                                                                                                                                                                                                                                                                                                         \\ \hline
\end{longtable}