% Please add the following required packages to your document preamble:
% \usepackage{longtable}
% Note: It may be necessary to compile the document several times to get a multi-page table to line up properly
\begin{longtable}{|l|l|}
\caption{Requerimiento funcional 2}
\label{RF2}\\
\hline
\begin{tabular}[c]{@{}l@{}}Nombre del requerimiento funcional:\\ Procesamiento y obtención de resultados.\end{tabular}                                                                 & ID: RF2                                                                                                                                                                                                                                                                                                                                                                                                              \\ \hline
\endfirsthead
%
\multicolumn{2}{c}%
{{\bfseries Tabla \thetable\ Continuación de la página anterior}} \\
\endhead
%
\multicolumn{2}{|l|}{Área: Procesamiento de datos  y obtención de resultados.}                                                                                                                                                                                                                                                                                                                                                                                                                                                                                                                                \\ \hline
\multicolumn{2}{|l|}{Actor(es): Sistema y usuario del sistema.}                                                                                                                                                                                                                                                                                                                                                                                                                                                                                                                                               \\ \hline
\multicolumn{2}{|l|}{Interesados: Sistema.}                                                                                                                                                                                                                                                                                                                                                                                                                                                                                                                                                                   \\ \hline
\multicolumn{2}{|l|}{\begin{tabular}[c]{@{}l@{}}Descripción: El sistema obtiene los datos tanto de los compuestos como de\\ la(s) proteína(s) a partir de su  conjunto de información correspondientes.\end{tabular}}                                                                                                                                                                                                                                                                                                                                                                                         \\ \hline
\multicolumn{2}{|l|}{\begin{tabular}[c]{@{}l@{}}Evento desencadenador: Se ha comprobado la integridad de la información\\ recolectada de la bases de datos, y esta ha sido empaquetada en el \\ conjunto 0” para el compuesto, igualmente con el conjunto P.\end{tabular}}                                                                                                                                                                                                                                                                                                                                    \\ \hline
\multicolumn{2}{|l|}{Tipo de desencadenador: Temporal.}                                                                                                                                                                                                                                                                                                                                                                                                                                                                                                                                                       \\ \hline
Pasos realizados (ruta principal)                                                                                                                                                      & Información para los pasos                                                                                                                                                                                                                                                                                                                                                                                           \\ \hline
\begin{tabular}[c]{@{}l@{}}1.- El sistema adquiere el conjunto 0\\ correspondiente a un compuesto en\\ específico.\end{tabular}                                                        & \begin{tabular}[c]{@{}l@{}}Conjunto 0 y \\ confirmación del \\ usuario.\end{tabular}                                                                                                                                                                                                                                                                                                                                 \\ \hline
\begin{tabular}[c]{@{}l@{}}2.- El sistema recibe el conjunto P para\\ conseguir la estructura molecular y nombre de\\ la proteína.\end{tabular}                                        & \begin{tabular}[c]{@{}l@{}}Conjunto P y \\ confirmación del \\ usuario.\end{tabular}                                                                                                                                                                                                                                                                                                                                 \\ \hline
\begin{tabular}[c]{@{}l@{}}3.-Uso del  algoritmo  de Machine Learning\\ que recibe todos los datos obtenidos del \\ conjunto 0 y P.\end{tabular}                                       & \begin{tabular}[c]{@{}l@{}}- Nombre Compuesto\\ - Estructura molecular\\ (compuesto).\\ - Descriptores Físico \\ Químicos.\\ - Actividad Biológica\\ - Nombre proteína\\ - Estructura molecular\\ (proteína).\end{tabular}                                                                                                                                                                                           \\ \hline
\begin{tabular}[c]{@{}l@{}}4.- Tras pasar por el módulo de Machine\\ Learning, el sistema realiza el modelo QSAR\\ de cada una de las interacciones\\ compuesto-proteína.\end{tabular} & \begin{tabular}[c]{@{}l@{}}- Nombre Compuesto\\ - Estructura molecular\\ (compuesto).\\ - Descriptores Físico \\ Químicos.\\ - Actividad Biológica\\ - Nombre proteína\\ - Estructura molecular\\ (proteína).\end{tabular}                                                                                                                                                                                           \\ \hline
\multicolumn{2}{|l|}{Extensiones o Rutas alternativas.}                                                                                                                                                                                                                                                                                                                                                                                                                                                                                                                                                       \\ \hline
\begin{tabular}[c]{@{}l@{}}1.1- El conjunto 0 está incompleto(No posee\\ datos necesarios).\\ 2.2- El conjunto P está incompleto(No posee\\ datos necesarios).\end{tabular}            & \begin{tabular}[c]{@{}l@{}}-Se informa qué datos \\ faltan o son \\ incompletos.\\ - Opción a intentar \\ adquirir  nuevamente  \\ los datos o se omite \\ obtener resultados \\ para ese compuesto.\\ -Se informa qué datos \\ faltan o son \\ incompletos.\\ - Se da opción a \\ intentar obtener\\ nuevamente los datos \\ de la proteína o \\ reformular  la \\ cantidad de proteínas \\ objetivos.\end{tabular} \\ \hline
\multicolumn{2}{|l|}{\begin{tabular}[c]{@{}l@{}}Precondiciones: Conexión a internet, Conjunto 0 y Conjunto P, ambos\\ completos .\end{tabular}}                                                                                                                                                                                                                                                                                                                                                                                                                                                               \\ \hline
\multicolumn{2}{|l|}{Postcondiciones: Se obtienen los resultados de interés para el usuario.}                                                                                                                                                                                                                                                                                                                                                                                                                                                                                                                 \\ \hline
\multicolumn{2}{|l|}{\begin{tabular}[c]{@{}l@{}}Suposiciones: Se tuvo éxito en módulos previos para la obtención correcta de\\ los datos de compuestos y patología.\end{tabular}}                                                                                                                                                                                                                                                                                                                                                                                                                             \\ \hline
\multicolumn{2}{|l|}{\begin{tabular}[c]{@{}l@{}}Garantía de éxito: El sistema obtiene resultados correctos y fiables de cada\\ uno de los compuestos enlistados, para la patología de interés.\end{tabular}}                                                                                                                                                                                                                                                                                                                                                                                                  \\ \hline
\multicolumn{2}{|l|}{\begin{tabular}[c]{@{}l@{}}Garantía mínima: Se obtienen los resultados que sean posibles, de no poder\\ obtener ninguno se informarán los errores.\end{tabular}}                                                                                                                                                                                                                                                                                                                                                                                                                         \\ \hline
\multicolumn{2}{|l|}{Requerimientos cumplidos: Confirmación y construcción del conjunto 0 .}                                                                                                                                                                                                                                                                                                                                                                                                                                                                                                                  \\ \hline
\multicolumn{2}{|l|}{Cuestiones pendientes: Los resultados a obtener.}                                                                                                                                                                                                                                                                                                                                                                                                                                                                                                                                        \\ \hline
\multicolumn{2}{|l|}{Prioridad: Alta.}                                                                                                                                                                                                                                                                                                                                                                                                                                                                                                                                                                        \\ \hline
\multicolumn{2}{|l|}{Riesgo: Alto.}                                                                                                                                                                                                                                                                                                                                                                                                                                                                                                                                                                           \\ \hline
\end{longtable}