% Please add the following required packages to your document preamble:
% \usepackage{longtable}
% Note: It may be necessary to compile the document several times to get a multi-page table to line up properly
\begin{longtable}{|l|l|}
\caption{Requerimiento funcional 1.3}
\label{RF1_3}\\
\hline
\begin{tabular}[c]{@{}l@{}}Nombre del requerimiento funcional:\\ Búsqueda de los descriptores de los compuestos.\end{tabular}                                                                                                                                                                                                               & ID: RF1.3                                                                                                                      \\ \hline
\endfirsthead
%
\multicolumn{2}{c}%
{{\bfseries Tabla \thetable\ Continuación de la página anterior}} \\
\endhead
%
\multicolumn{2}{|l|}{Área: Búsqueda de información.}                                                                                                                                                                                                                                                                                                                                                                                                                         \\ \hline
\multicolumn{2}{|l|}{Actor(es): Sistema.}                                                                                                                                                                                                                                                                                                                                                                                                                                    \\ \hline
\multicolumn{2}{|l|}{Interesados: Sistema, usuario del sistema.}                                                                                                                                                                                                                                                                                                                                                                                                             \\ \hline
\multicolumn{2}{|l|}{\begin{tabular}[c]{@{}l@{}}Descripción: Utilizar las bases de datos “ChemSpider” y “PubMed” para recopilar\\ toda la información posible acerca de las propiedades fisicoquímicas de cada\\ compuesto validado por el sistema.\end{tabular}}                                                                                                                                                                                                            \\ \hline
\multicolumn{2}{|l|}{\begin{tabular}[c]{@{}l@{}}Evento desencadenador: Luego de que el sistema haya realizado la búsqueda\\ para identificar los compuestos ingresados por el usuario, inmediatamente se\\ procederá a realizar la búsqueda de los descriptores para los compuestos de los\\ que el sistema tenga confirmada su existencia.\end{tabular}}                                                                                                                    \\ \hline
\multicolumn{2}{|l|}{Tipo de desencadenador: Temporal.}                                                                                                                                                                                                                                                                                                                                                                                                                      \\ \hline
Pasos realizados (ruta principal)                                                                                                                                                                                                                                                                                                           & Información para los pasos                                                                                                     \\ \hline
\begin{tabular}[c]{@{}l@{}}1.- El sistema se conecta a las bases de datos\\ “ChemSpider” y “PubChem”.\end{tabular}                                                                                                                                                                                                                          & \begin{tabular}[c]{@{}l@{}}No se requiere información\\ para este paso.\end{tabular}                                           \\ \hline
\begin{tabular}[c]{@{}l@{}}2.-  El sistema realiza la consulta correspondiente\\ a los descriptores para cada uno de los\\ compuestos.\end{tabular}                                                                                                                                                                                         & \begin{tabular}[c]{@{}l@{}}Lista de nombres de\\ compuestos.\end{tabular}                                                      \\ \hline
\begin{tabular}[c]{@{}l@{}}3.- El sistema crea un archivo de texto plano\\ (temporal) donde en la primera línea indica el\\ nombre del compuesto al cual pertenece la\\ información ahí descrita, para luego copiar los\\ resultados obtenidos de la consulta realizada en\\ el paso anterior a dicho archivo.\end{tabular} & \begin{tabular}[c]{@{}l@{}}Nombre del compuesto al que\\ pertenece la información de\\ los descriptores obtenida.\end{tabular} \\ \hline
\multicolumn{2}{|l|}{Extensiones o Rutas alternativas.}                                                                                                                                                                                                                                                                                                                                                                                                                      \\ \hline
\multicolumn{1}{|c|}{N/A}                                                                                                                                                                                                                                                                                                                     & \multicolumn{1}{c|}{N/A}                                                                                                         \\ \hline
\multicolumn{2}{|l|}{\begin{tabular}[c]{@{}l@{}}Precondiciones: El sistema posee una lista de compuestos de los cuales ya ha\\ verificado su existencia.\end{tabular}}                                                                                                                                                                                                                                                                                                       \\ \hline
\multicolumn{2}{|l|}{\begin{tabular}[c]{@{}l@{}}Postcondiciones: El sistema obtiene la información de los descriptores de los\\ compuestos, necesaria para el proceso del modelado QSAR.\end{tabular}}                                                                                                                                                                                                                                                                       \\ \hline
\multicolumn{2}{|l|}{\begin{tabular}[c]{@{}l@{}}Suposiciones: El sistema ya ha verificado los compuestos de los cuales desea\\ obtener sus descriptores y las bases de datos “PubChem”, “ChemSpider” se\\ encuentran accesibles.\end{tabular}}                                                                                                                                                                                                                               \\ \hline
\multicolumn{2}{|l|}{\begin{tabular}[c]{@{}l@{}}Garantía de éxito: El sistema recopila la información proporcionada por los\\ descriptores fisicoquímicos de cada uno de los compuestos de la lista dada por el\\ usuario.\end{tabular}}                                                                                                                                                                                                                                     \\ \hline
\multicolumn{2}{|l|}{\begin{tabular}[c]{@{}l@{}}Garantía mínima: El sistema realiza la búsqueda sobre las base de datos\\ “PubChem” y “ChemSpider, y almacena los descriptores fisicoquímicos de los\\ compuestos encontrados.\end{tabular}}                                                                                                                                                                                                                                 \\ \hline
\multicolumn{2}{|l|}{Requerimientos cumplidos: Búsqueda de los descriptores de los compuestos.}                                                                                                                                                                                                                                                                                                                                                                              \\ \hline
\multicolumn{2}{|l|}{Prioridad: Alta.}                                                                                                                                                                                                                                                                                                                                                                                                                                       \\ \hline
\multicolumn{2}{|l|}{Riesgo: Alta}                                                                                                                                                                                                                                                                                                                                                                                                                                           \\ \hline
\end{longtable}