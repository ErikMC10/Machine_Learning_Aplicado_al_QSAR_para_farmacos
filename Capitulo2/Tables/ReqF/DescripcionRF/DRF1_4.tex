% Please add the following required packages to your document preamble:
% \usepackage{longtable}
% Note: It may be necessary to compile the document several times to get a multi-page table to line up properly
\begin{longtable}{|l|l|}
\caption{Requerimiento funcional 1.4}
\label{RF1_4}\\
\hline
\begin{tabular}[c]{@{}l@{}}Nombre del requerimiento funcional:\\ Búsqueda de los mecanismos de acción de los\\ compuestos.\end{tabular}                                                                                                                                                             & ID: RF1.4                                                                            \\ \hline
\endfirsthead
%
\multicolumn{2}{c}%
{{\bfseries Tabla \thetable\ Continuación de la página anterior}} \\
\endhead
%
\multicolumn{2}{|l|}{Área: Búsqueda de información.}                                                                                                                                                                                                                                                                                                                                         \\ \hline
\multicolumn{2}{|l|}{Actor(es): Sistema.}                                                                                                                                                                                                                                                                                                                                                    \\ \hline
\multicolumn{2}{|l|}{Interesados: Sistema, usuario del sistema.}                                                                                                                                                                                                                                                                                                                             \\ \hline
\multicolumn{2}{|l|}{\begin{tabular}[c]{@{}l@{}}Descripción: Obtener, utilizando la base de datos “DrugBank”, toda la\\ información posible acerca de las actividades biológicas de los compuestos\\ validados por el sistema.\end{tabular}}                                                                                                                                                 \\ \hline
\multicolumn{2}{|l|}{\begin{tabular}[c]{@{}l@{}}Evento desencadenador: Luego de que el sistema haya realizado la búsqueda\\ para identificar los compuestos ingresados por el usuario, inmediatamente se\\ procederá a realizar la búsqueda de las actividades biológicas para los\\ compuestos de los que el sistema tenga confirmada su existencia.\end{tabular}}                          \\ \hline
\multicolumn{2}{|l|}{Tipo de desencadenador: Temporal.}                                                                                                                                                                                                                                                                                                                                      \\ \hline
Pasos realizados (ruta principal)                                                                                                                                                                                                                                                                     & Información para los pasos                                                           \\ \hline
\begin{tabular}[c]{@{}l@{}}1.- El sistema se conecta a la base de datos\\ “Drugbank”.\end{tabular}                                                                                                                                                                                                    & \begin{tabular}[c]{@{}l@{}}No se requiere información\\ para este paso.\end{tabular} \\ \hline
\begin{tabular}[c]{@{}l@{}}2.- El sistema realiza la consulta\\ correspondiente a las actividades biológicas\\ para cada compuesto.\end{tabular}                                                                                                                                                      & Nombre del compuesto.                                                                \\ \hline
\begin{tabular}[c]{@{}l@{}}3.- Se almacena toda la información obtenida\\ en un archivo de texto plano temporal\\ denominado “MechOfAct”, donde se le agrega\\ una línea de texto indicando el nombre del\\ compuesto al que pertenece la información de\\ las actividades biológicas.\end{tabular} & \begin{tabular}[c]{@{}l@{}}No se requiere información para\\ este paso.\end{tabular} \\ \hline
\multicolumn{2}{|l|}{Extensiones o Rutas alternativas.}                                                                                                                                                                                                                                                                                                                                      \\ \hline
\multicolumn{1}{|c|}{N/A}                                                                                                                                                                                                                                                                               & \multicolumn{1}{c|}{N/A}                                                               \\ \hline
\multicolumn{2}{|l|}{\begin{tabular}[c]{@{}l@{}}Precondiciones: El sistema posee una lista de compuestos de los cuales ya ha\\ verificado su existencia.\end{tabular}}                                                                                                                                                                                                                       \\ \hline
\multicolumn{2}{|l|}{\begin{tabular}[c]{@{}l@{}}Postcondiciones: El sistema obtiene la información de las actividades biológicas\\ de los compuestos, necesaria para el proceso del modelado QSAR.\end{tabular}}                                                                                                                                                                             \\ \hline
\multicolumn{2}{|l|}{\begin{tabular}[c]{@{}l@{}}Suposiciones: El sistema ya ha verificado los compuestos de los cuales desea\\ obtener sus actividades biológicas y las base de datos “DrugBank” se encuentra\\ accesible.\end{tabular}}                                                                                                                                                     \\ \hline
\multicolumn{2}{|l|}{\begin{tabular}[c]{@{}l@{}}Garantía de éxito: El sistema recopila la información proporcionada por las\\ actividades biológicas de cada uno de los compuestos de la lista dada por el\\ usuario.\end{tabular}}                                                                                                                                                          \\ \hline
\multicolumn{2}{|l|}{\begin{tabular}[c]{@{}l@{}}Garantía mínima: El sistema realiza la búsqueda sobre la base de datos\\ “DrugBank” y almacena las actividades biológicas de los compuestos\\ encontrados.\end{tabular}}                                                                                                                                                                     \\ \hline
\multicolumn{2}{|l|}{\begin{tabular}[c]{@{}l@{}}Requerimientos cumplidos: Búsqueda de las actividades biológicas de los\\ compuestos.\end{tabular}}                                                                                                                                                                                                                                          \\ \hline
\multicolumn{2}{|l|}{Prioridad: Alta.}                                                                                                                                                                                                                                                                                                                                                       \\ \hline
\multicolumn{2}{|l|}{Riesgo: Alta}                                                                                                                                                                                                                                                                                                                                                           \\ \hline
\end{longtable}