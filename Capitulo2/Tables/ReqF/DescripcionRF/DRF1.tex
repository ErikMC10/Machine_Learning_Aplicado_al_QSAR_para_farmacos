% Please add the following required packages to your document preamble:
% \usepackage{longtable}
% Note: It may be necessary to compile the document several times to get a multi-page table to line up properly
\begin{longtable}{|l|l|}
\caption{Requerimiento funcional 1}
\label{RF1}\\
\hline
\begin{tabular}[c]{@{}l@{}}Nombre del requerimiento funcional:\\ Búsqueda de información de las variables\\  de entrada del sistema\end{tabular}                                                                                   & ID: RF1                                                                               \\ \hline
\endfirsthead
%
\multicolumn{2}{c}%
{{\bfseries Tabla \thetable\ Continuación de la página anterior}} \\
\endhead
%
\multicolumn{2}{|l|}{Área: Búsqueda de información.}                                                                                                                                                                                                                                                                       \\ \hline
\multicolumn{2}{|l|}{Actor(es): Usuario del sistema.}                                                                                                                                                                                                                                                                      \\ \hline
\multicolumn{2}{|l|}{Interesados: Sistema.}                                                                                                                                                                                                                                                                                \\ \hline
\multicolumn{2}{|l|}{\begin{tabular}[c]{@{}l@{}}Descripción: Obtener información de los datos ingresados por el usuario \\ mediante la lectura de un archivo y la búsqueda de su contenido en las bases de\\ datos en línea correspondientes.\end{tabular}}                                                             \\ \hline
\multicolumn{2}{|l|}{Evento desencadenador:}                                                                                                                                                                                                                                                                               \\ \hline
\multicolumn{2}{|l|}{Tipo de desencadenador: Externo y temporal.}                                                                                                                                                                                                                                                          \\ \hline
Pasos realizados (ruta principal)                                                                                                                                                                                                  & Información para los pasos                                                            \\ \hline
\begin{tabular}[c]{@{}l@{}}1.- El usuario inicia el programa esperando\\ que este termine de cargar su interfaz.\end{tabular}                                                                                                      & \begin{tabular}[c]{@{}l@{}}No se requiere información para\\  este paso.\end{tabular} \\ \hline
\begin{tabular}[c]{@{}l@{}}2.- El usuario ingresa un archivo de texto \\ plano reconocido por el sistema como \\ “archivo base” y da clic en el botón “buscar”.\end{tabular}                                                       & \begin{tabular}[c]{@{}l@{}}Archivo que se desea cargar al \\ sistema.\end{tabular}    \\ \hline
\begin{tabular}[c]{@{}l@{}}3.-  El sistema lee el archivo y procede a\\ buscar la información de cada elemento \\ contenido en dicho archivo.\end{tabular}                                                                         & \begin{tabular}[c]{@{}l@{}}No se requiere información para\\ este paso.\end{tabular}  \\ \hline
\begin{tabular}[c]{@{}l@{}}4.- El sistema almacena la información \\ obtenida y al finalizar su búsqueda presenta la \\ información al usuario con la finalidad de que \\ este observe los resultados de la búsqueda.\end{tabular} & \begin{tabular}[c]{@{}l@{}}No se requiere información para\\ este paso.\end{tabular}  \\ \hline
\multicolumn{2}{|l|}{Extensiones o Rutas alternativas.}                                                                                                                                                                                                                                                                    \\ \hline
\multicolumn{1}{|c|}{N/A}                                                                                                                                                                                                            & \multicolumn{1}{c|}{N/A}                                                                \\ \hline
\multicolumn{2}{|l|}{\begin{tabular}[c]{@{}l@{}}Precondiciones: El usuario ingresa un archivo de texto plano, el archivo cumple \\ con la estructura indicada por el sistema para su correcta lectura.\end{tabular}}                                                                                                       \\ \hline
\multicolumn{2}{|l|}{\begin{tabular}[c]{@{}l@{}}Postcondiciones: El usuario puede observar los datos que el sistema ha \\ obtenido y verificar que son correctos.\end{tabular}}                                                                                                                                            \\ \hline
\multicolumn{2}{|l|}{\begin{tabular}[c]{@{}l@{}}Suposiciones: El usuario tiene una buena conexión a internet, el programa \\ instalado y el dato especificado existe en las bases de datos.\end{tabular}}                                                                                                                  \\ \hline
\multicolumn{2}{|l|}{\begin{tabular}[c]{@{}l@{}}Garantía de éxito: El usuario puede obtener y seleccionar el dato que coincida en\\ la mayor exactitud con el que él desea.\end{tabular}}                                                                                                                                  \\ \hline
\multicolumn{2}{|l|}{Garantía mínima: El usuario ingresa datos reales que comprueba el sistema.}                                                                                                                                                                                                                           \\ \hline
\multicolumn{2}{|l|}{\begin{tabular}[c]{@{}l@{}}Requerimientos cumplidos: Búsqueda de información sobre las variables de\\ entrada del sistema.\end{tabular}}                                                                                                                                                              \\ \hline
\multicolumn{2}{|l|}{Prioridad: Alta.}                                                                                                                                                                                                                                                                                     \\ \hline
\multicolumn{2}{|l|}{Riesgo: Alto.}                                                                                                                                                                                                                                                                                        \\ \hline
\end{longtable}