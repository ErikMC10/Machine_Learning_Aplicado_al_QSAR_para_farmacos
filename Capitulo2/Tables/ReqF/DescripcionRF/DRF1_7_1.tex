% Please add the following required packages to your document preamble:
% \usepackage{longtable}
% Note: It may be necessary to compile the document several times to get a multi-page table to line up properly
\begin{longtable}{|l|l|}
\caption{Requerimiento funcional 1.7.1}
\label{RF1_7_1}\\
\hline
\begin{tabular}[c]{@{}l@{}}Nombre del requerimiento funcional:\\ Construcción del conjunto cero.\end{tabular}                                                                                                                                                                           & ID: RF1.7.1                                                                                                    \\ \hline
\endfirsthead
%
\multicolumn{2}{c}%
{{\bfseries Tabla \thetable\ Continuación de la página anterior}} \\
\endhead
%
\multicolumn{2}{|l|}{Área: Organización de resultados de la búsqueda de información.}                                                                                                                                                                                                                                                                                                                    \\ \hline
\multicolumn{2}{|l|}{Actor(es): Sistema.}                                                                                                                                                                                                                                                                                                                                                                \\ \hline
\multicolumn{2}{|l|}{Interesados: Sistema.}                                                                                                                                                                                                                                                                                                                                                              \\ \hline
\multicolumn{2}{|l|}{\begin{tabular}[c]{@{}l@{}}Descripción: Ordenar el nombre y la estructura de cada compuesto obtenido\\ por el sistema (a partir de la lista ingresada por el usuario) en un archivo con\\ formato pdb denominado “conjunto 0”.\end{tabular}}                                                                                                                                        \\ \hline
\multicolumn{2}{|l|}{\begin{tabular}[c]{@{}l@{}}Evento desencadenador: Luego de que el usuario haya validado los resultados\\ de la búsqueda de información, se procede inmediatamente a ordenar la\\ información del nombre y la estructura de los compuestos verificados, en un\\ archivo en formato pdb.\end{tabular}}                                                                                \\ \hline
\multicolumn{2}{|l|}{Tipo de desencadenador: Temporal.}                                                                                                                                                                                                                                                                                                                                                  \\ \hline
Pasos realizados (ruta principal)                                                                                                                                                                                                                                                       & Información para los pasos                                                                                     \\ \hline
\begin{tabular}[c]{@{}l@{}}1.- El sistema obtiene la información\\ correspondiente al nombre del compuesto y\\ su estructura molecular, para cada uno de los\\ compuestos encontrados.\end{tabular}                                                                                     & \begin{tabular}[c]{@{}l@{}}Nombres de los compuestos,\\ estructura molecular de los\\ compuestos.\end{tabular} \\ \hline
\begin{tabular}[c]{@{}l@{}}2.- El sistema construye un archivo en formato\\ pdb (que es el conjunto 0, el cual está\\ identificado como C0 en el diccionario de\\ datos) que se denomina como el nombre del\\ compuesto al que pertenece la información,\\ donde agrega como “REMARK” (identificado\\ en el diccionario de datos como RM) el\\ nombre del compuesto seguido de su\\ estructura molecular, para cada uno de los\\ compuestos encontrados.\end{tabular} & \begin{tabular}[c]{@{}l@{}}Archivo pdb (conjunto 0) donde\\ se guarda la información.\end{tabular}             \\ \hline
\multicolumn{2}{|l|}{Extensiones o Rutas alternativas.}                                                                                                                                                                                                                                                                                                                                                  \\ \hline
\multicolumn{1}{|c|}{N/A}                                                                                                                                                                                                                                                                 & \multicolumn{1}{c|}{N/A}                                                                                         \\ \hline
\multicolumn{2}{|l|}{Precondiciones: El sistema termina la búsqueda de información.}                                                                                                                                                                                                                                                                                                                     \\ \hline
\multicolumn{2}{|l|}{\begin{tabular}[c]{@{}l@{}}Postcondiciones: El sistema construye los conjuntos de datos que contendrán\\ la información sobre cada uno de los compuestos y sus estructuras moleculares.\end{tabular}}                                                                                                                                                                               \\ \hline
\multicolumn{2}{|l|}{\begin{tabular}[c]{@{}l@{}}Suposiciones: El sistema tiene acceso al almacenamiento del ordenador donde\\ se está ejecutando, tiene permitido la creación de archivos y el usuario ha\\ verificado los resultados de la búsqueda.\end{tabular}}                                                                                                                                      \\ \hline
\multicolumn{2}{|l|}{\begin{tabular}[c]{@{}l@{}}Garantía de éxito: El sistema organiza la información del nombre y la estructura\\ de los compuestos en archivos que son reconocibles para todos los módulos\\ que lo componen.\end{tabular}}                                                                                                                                                            \\ \hline
\multicolumn{2}{|l|}{Garantía mínima: No hay garantía mínima.}                                                                                                                                                                                                                                                                                                                                           \\ \hline
\multicolumn{2}{|l|}{Requerimientos cumplidos: Construcción del conjunto cero.}                                                                                                                                                                                                                                                                                                                          \\ \hline
\multicolumn{2}{|l|}{Prioridad: Media.}                                                                                                                                                                                                                                                                                                                                                                  \\ \hline
\multicolumn{2}{|l|}{Riesgo: Alto.}                                                                                                                                                                                                                                                                                                                                                                      \\ \hline
\end{longtable}