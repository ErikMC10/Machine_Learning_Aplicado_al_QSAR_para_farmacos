% Please add the following required packages to your document preamble:
% \usepackage{longtable}
% Note: It may be necessary to compile the document several times to get a multi-page table to line up properly
\begin{longtable}{|l|l|}
\caption{Requerimiento funcional 1.7.2}
\label{RF1_7_2}\\
\hline
\begin{tabular}[c]{@{}l@{}}Nombre del requerimiento funcional:\\ Construcción del conjunto P.\end{tabular}                                                                                                                                                                              & ID: RF1.7.2                                                                                                  \\ \hline
\endfirsthead
%
\multicolumn{2}{c}%
{{\bfseries Tabla \thetable\ Continuación de la página anterior}} \\
\endhead
%
\multicolumn{2}{|l|}{Área: Organización de resultados de la búsqueda de información.}                                                                                                                                                                                                                                                                                                                  \\ \hline
\multicolumn{2}{|l|}{Actor(es): Sistema.}                                                                                                                                                                                                                                                                                                                                                              \\ \hline
\multicolumn{2}{|l|}{Interesados: Sistema.}                                                                                                                                                                                                                                                                                                                                                            \\ \hline
\multicolumn{2}{|l|}{\begin{tabular}[c]{@{}l@{}}Descripción: Ordenar el nombre y la estructura de cada proteína obtenido por\\ el sistema (a partir de la lista ingresada por el usuario) en un archivo con\\ formato pdb denominado “conjunto P”.\end{tabular}}                                                                                                                                       \\ \hline
\multicolumn{2}{|l|}{\begin{tabular}[c]{@{}l@{}}Evento desencadenador: Luego de que el usuario haya validado los\\ resultados de la búsqueda de información, se procede inmediatamente a\\ ordenar la información del nombre y la estructura de las proteínas verificadas,\\ en un archivo en formato pdb.\end{tabular}}                                                                               \\ \hline
\multicolumn{2}{|l|}{Tipo de desencadenador: Temporal.}                                                                                                                                                                                                                                                                                                                                                \\ \hline
Pasos realizados (ruta principal)                                                                                                                                                                                                                                                       & Información para los pasos                                                                                   \\ \hline
\begin{tabular}[c]{@{}l@{}}1.- El sistema obtiene la información\\ correspondiente al nombre de cada proteína y\\ su respectiva estructura molecular.\end{tabular}                                                                                                                      & \begin{tabular}[c]{@{}l@{}}Nombres de las proteínas,\\ estructura molecular de las\\ proteínas.\end{tabular} \\ \hline
\begin{tabular}[c]{@{}l@{}}2.- El sistema construye un archivo en formato\\ pdb (que es el conjunto P, el cual está\\ identificado como CP en el diccionario de\\ datos) donde agrega como “REMARK” el\\ nombre de la proteína seguido de su\\ estructura molecular, para cada uno de los\\ proteínas encontradas.\end{tabular} & \begin{tabular}[c]{@{}l@{}}Archivo pdb (conjunto P)\\ donde se guarda la\\ información.\end{tabular}         \\ \hline
\multicolumn{2}{|l|}{Extensiones o Rutas alternativas.}                                                                                                                                                                                                                                                                                                                                                \\ \hline
\multicolumn{1}{|c|}{N/A}                                                                                                                                                                                                                                                                 & \multicolumn{1}{c|}{N/A}                                                                                       \\ \hline
\multicolumn{2}{|l|}{Precondiciones: El sistema termina la confirmación de la información.}                                                                                                                                                                                                                                                                                                            \\ \hline
\multicolumn{2}{|l|}{\begin{tabular}[c]{@{}l@{}}Postcondiciones: El sistema construye los conjuntos de datos que contendrán\\ la información sobre cada una de las proteínas y sus estructuras moleculares.\end{tabular}}                                                                                                                                                                              \\ \hline
\multicolumn{2}{|l|}{\begin{tabular}[c]{@{}l@{}}Suposiciones: El sistema tiene acceso al almacenamiento del ordenador\\ donde se está ejecutando, tiene permitido la creación de archivos y el usuario\\ ha verificado los resultados de la búsqueda.\end{tabular}}                                                                                                                                    \\ \hline
\multicolumn{2}{|l|}{\begin{tabular}[c]{@{}l@{}}Garantía de éxito: El sistema organiza la información del nombre y la\\ estructura de las proteínas en archivos que son reconocibles para todos los\\ módulos que lo componen.\end{tabular}}                                                                                                                                                           \\ \hline
\multicolumn{2}{|l|}{Garantía mínima: No hay garantía mínima.}                                                                                                                                                                                                                                                                                                                                         \\ \hline
\multicolumn{2}{|l|}{Requerimientos cumplidos: Construcción del conjunto P.}                                                                                                                                                                                                                                                                                                                           \\ \hline
\multicolumn{2}{|l|}{Prioridad: Media.}                                                                                                                                                                                                                                                                                                                                                                \\ \hline
\multicolumn{2}{|l|}{Riesgo: Alto.}                                                                                                                                                                                                                                                                                                                                                                    \\ \hline
\end{longtable}