% Please add the following required packages to your document preamble:
% \usepackage{longtable}
% Note: It may be necessary to compile the document several times to get a multi-page table to line up properly
\begin{longtable}{|l|l|}
\caption{Requerimiento funcional 2.2}
\label{RF2_2}\\
\hline
\begin{tabular}[c]{@{}l@{}}Nombre del requerimiento funcional:\\ Adquisición y descomposición del conjunto P.\end{tabular}                                            & ID: RF2.2                                                                                                                                                                          \\ \hline
\endfirsthead
%
\multicolumn{2}{c}%
{{\bfseries Tabla \thetable\ Continuación de la página anterior}} \\
\endhead
%
\multicolumn{2}{|l|}{Área: Procesamiento de datos  y obtención de resultados.}                                                                                                                                                                                                                                                                             \\ \hline
\multicolumn{2}{|l|}{Actor(es): Sistema.}                                                                                                                                                                                                                                                                                                                  \\ \hline
\multicolumn{2}{|l|}{Interesados: Sistema.}                                                                                                                                                                                                                                                                                                                \\ \hline
\multicolumn{2}{|l|}{\begin{tabular}[c]{@{}l@{}}Descripción: El sistema consigue el nombre de la proteína de interés así\\ como su estructura molecular esta información proveniente del conjunto P.\end{tabular}}                                                                                                                                         \\ \hline
\multicolumn{2}{|l|}{Evento desencadenador: Confirmación de adquisición de conjunto 0 (Sistema ).}                                                                                                                                                                                                                                                         \\ \hline
\multicolumn{2}{|l|}{Tipo de desencadenador: Externo.}                                                                                                                                                                                                                                                                                                     \\ \hline
Pasos realizados (ruta principal)                                                                                                                                     & Información para los pasos                                                                                                                                                         \\ \hline
\begin{tabular}[c]{@{}l@{}}1.-El sistema obtiene el conjunto P\\ perteneciente a un compuesto específico.\end{tabular}                                                & No requiere información.                                                                                                                                                           \\ \hline
\begin{tabular}[c]{@{}l@{}}2.- El sistema lee  el contenido de\\ perteneciente al conjunto P recién adquirido\\ para obtener los datos necesarios.\end{tabular}       & Conjunto P.                                                                                                                                                                        \\ \hline
\multicolumn{2}{|l|}{Extensiones o Rutas alternativas.}                                                                                                                                                                                                                                                                                                    \\ \hline
\begin{tabular}[c]{@{}l@{}}1.1- Error en la adquisición del conjunto P.\\  2.1 Dato inexistente o incompleto en el\\ conjunto P de reciente adquisición.\end{tabular} & \begin{tabular}[c]{@{}l@{}}- Se informa error en la\\ adquisición y motivo.\\  \\ - Se indica que  dato(s) \\ están incompletos o no se\\ encuentran en el archivo.\end{tabular} \\ \hline
\multicolumn{2}{|l|}{\begin{tabular}[c]{@{}l@{}}Precondiciones: Conexión a internet, Conjunto P íntegro y datos completos,\\ así como la exitosa adquisición previa del conjunto 0.\end{tabular}}                                                                                                                                                          \\ \hline
\multicolumn{2}{|l|}{\begin{tabular}[c]{@{}l@{}}Postcondiciones: Se adquieren los datos necesarios de la proteína de\\ interés.\end{tabular}}                                                                                                                                                                                                              \\ \hline
\multicolumn{2}{|l|}{\begin{tabular}[c]{@{}l@{}}Suposiciones: Los datos que componen el conjunto P no han sido\\ corrompidos, están completos y están en el directorio correcto y la adquisición\\ del Conjunto 0.\end{tabular}}                                                                                                                           \\ \hline
\multicolumn{2}{|l|}{\begin{tabular}[c]{@{}l@{}}Garantía de éxito: El sistema continúa su ejecución, haciendo uso de los\\ datos obtenidos.\end{tabular}}                                                                                                                                                                                                  \\ \hline
\multicolumn{2}{|l|}{\begin{tabular}[c]{@{}l@{}}Garantía mínima: El sistema busca e intenta adquirir el conjunto P respectivo\\ a un compuesto.\end{tabular}}                                                                                                                                                                                              \\ \hline
\multicolumn{2}{|l|}{Requerimientos cumplidos: Adquisición y descomposición del conjunto P.}                                                                                                                                                                                                                                                               \\ \hline
\multicolumn{2}{|l|}{Cuestiones pendientes: Orden de lectura y adquisición de datos.}                                                                                                                                                                                                                                                                      \\ \hline
\multicolumn{2}{|l|}{Prioridad: Alta.}                                                                                                                                                                                                                                                                                                                     \\ \hline
\multicolumn{2}{|l|}{Riesgo: Alto.}                                                                                                                                                                                                                                                                                                                        \\ \hline
\end{longtable}