% Please add the following required packages to your document preamble:
% \usepackage{longtable}
% Note: It may be necessary to compile the document several times to get a multi-page table to line up properly
\begin{longtable}{|l|l|}
\caption{Requerimiento funcional 1.2}
\label{RF1_2}\\
\hline
\begin{tabular}[c]{@{}l@{}}Nombre del requerimiento funcional:\\ Búsqueda de los compuestos indicados.\end{tabular}                                                                                   & ID: RF1.2                                                                                                \\ \hline
\endfirsthead
%
\multicolumn{2}{c}%
{{\bfseries Tabla \thetable\ Continuación de la página anterior}} \\
\endhead
%
\multicolumn{2}{|l|}{Área: Búsqueda de información.}                                                                                                                                                                                                                                                             \\ \hline
\multicolumn{2}{|l|}{Actor(es): Usuario del sistema.}                                                                                                                                                                                                                                                            \\ \hline
\multicolumn{2}{|l|}{Interesados: Sistema.}                                                                                                                                                                                                                                                                      \\ \hline
\multicolumn{2}{|l|}{\begin{tabular}[c]{@{}l@{}}Descripción: Buscar si el nombre de cada compuesto en la lista dada por el\\ usuario existe en la base de datos “DrugBank” para obtener su estructura\\ molecular en un archivo con formato pdb..\end{tabular}}                                                  \\ \hline
\multicolumn{2}{|l|}{\begin{tabular}[c]{@{}l@{}}Evento desencadenador: Luego de que el sistema haya realizado la lectura del \\archivo base ingresado por el usuario, inmediatamente se procederá a realizar la \\búsqueda de los compuestos que se indican en el archivo mencionado anteriormente.\end{tabular}}                                                                                                                                                                                                                                                                     \\ \hline
\multicolumn{2}{|l|}{Tipo de desencadenador: Externo.}                                                                                                                                                                                                                                                           \\ \hline
Pasos realizados (ruta principal)                                                                                                                                                                     & Información para los pasos                                                                               \\ \hline
\begin{tabular}[c]{@{}l@{}}1.- El sistema lee el archivo ingresado,\\ busca la etiqueta “Compounds:” y \\obtiene una lista de compuestos.\end{tabular}                                                & Archivo cargado al sistema.                                                                              \\ \hline
\begin{tabular}[c]{@{}l@{}}2.-  El sistema se conecta a la base de\\ datos en línea “DrugBank”.\end{tabular}                                                                                          & \begin{tabular}[c]{@{}l@{}}No se requiere información para\\  este paso.\end{tabular}                    \\ \hline
\begin{tabular}[c]{@{}l@{}}3.- El sistema realiza la consulta\\ correspondiente a los nombres para cada\\ uno de los compuestos y obtiene su\\ estructura molecular.\end{tabular}                     & Lista de nombres de compuestos.                                                                          \\ \hline
\begin{tabular}[c]{@{}l@{}}4.- El sistema obtiene un archivo con \\extensión pdb para cada compuesto \\validado y lo almacena en el ordenador.\end{tabular}                           & \begin{tabular}[c]{@{}l@{}}No se requiere información para\\ este paso.\end{tabular}                     \\ \hline
\multicolumn{2}{|l|}{Extensiones o Rutas alternativas.}                                                                                                                                                                                                                                                          \\ \hline
\multicolumn{1}{|c|}{N/A}                                                                                                                                                                               & \multicolumn{1}{c|}{N/A}                                                                                   \\ \hline
\multicolumn{2}{|l|}{Precondiciones: El sistema ya ha validado y leído el archivo base.}                                                                                                                                                                                                                         \\ \hline
\multicolumn{2}{|l|}{\begin{tabular}[c]{@{}l@{}}Postcondiciones: El sistema valida la existencia de los compuestos dados por el\\ usuario en la base de datos “DrugBank”, obtiene sus estructuras moleculares en\\ archivos con extensión “pdb”, y puede continuar con su búsqueda de información.\end{tabular}} \\ \hline
\multicolumn{2}{|l|}{\begin{tabular}[c]{@{}l@{}}Suposiciones: El usuario ha ingresado correctamente el nombre de cada\\ compuesto, el archivo base se ha leído correctamente y la base de datos\\ “DrugBank” se encuentra accesible.\end{tabular}}                                                               \\ \hline
\multicolumn{2}{|l|}{\begin{tabular}[c]{@{}l@{}}Garantía de éxito: El sistema confirma que los compuestos dados por el usuario\\ existen, almacena la estructura molecular de cada compuesto, y se permite\\ proceder a obtener el resto de la información de cada compuesto.\end{tabular}}                      \\ \hline
\multicolumn{2}{|l|}{\begin{tabular}[c]{@{}l@{}}Garantía mínima: El sistema realiza la búsqueda sobre la base de datos\\ “Drugbank” y almacena las estructuras moleculares de los compuestos\\ encontrados.\end{tabular}}                                                                                        \\ \hline
\multicolumn{2}{|l|}{Requerimientos cumplidos: Búsqueda de los compuestos indicados.}                                                                                                                                                                                                                            \\ \hline
\multicolumn{2}{|l|}{Prioridad: Alta.}                                                                                                                                                                                                                                                                           \\ \hline
\multicolumn{2}{|l|}{Riesgo: Medio.}                                                                                                                                                                                                                                                                             \\ \hline
\end{longtable}