% Please add the following required packages to your document preamble:
% \usepackage{longtable}
% Note: It may be necessary to compile the document several times to get a multi-page table to line up properly
\begin{longtable}{|l|l|l|}
\caption{Elementos de software disponibles para el desarrollo del sistema}
\label{Est_Soft}\\
\hline
\textbf{NOMBRE}    & \textbf{VERSIÓN}                                                                                                     & \textbf{DESCRIPCIÓN}                                                                                                                                    \\ \hline
\endfirsthead
%
\multicolumn{3}{c}%
{{\bfseries Tabla \thetable\ Continuación de la página anterior}} \\
\endhead
%
Linux              & \begin{tabular}[c]{@{}l@{}}Cualquier distribución en \\ su última versión estable.\\ (Basada en Debian)\end{tabular} & \begin{tabular}[c]{@{}l@{}}1GNU/Linux  es un sistema \\ operativo libre tipo Unix \\ POSIX; multiplataforma, \\ multiusuario y multitarea.\end{tabular} \\ \hline
Visual Studio Code & 1.38.0 o última versión estable.                                                                                     & \begin{tabular}[c]{@{}l@{}}Editor de texto enfocado \\ a la creación de código fuente.\end{tabular}                                                     \\ \hline
Python             & 3.7.4 o última versión estable.                                                                                      & \begin{tabular}[c]{@{}l@{}}Lenguaje de programación \\ de alto nivel multipropósito.\end{tabular}                                                       \\ \hline
Git                & 2.23.0 o última versión estable.                                                                                     & Sistema de control de versiones.                                                                                                                        \\ \hline
\end{longtable}