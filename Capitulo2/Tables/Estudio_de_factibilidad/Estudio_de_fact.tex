\subsection{Factibilidad técnica}
\noindent En las tablas \ref{Est_Hard} y \ref{Est_Soft} se en listan los elementos de desarrollo que hacen posible la construcción del sistema.\\

- Hardware
% Please add the following required packages to your document preamble:
% \usepackage{longtable}
% Note: It may be necessary to compile the document several times to get a multi-page table to line up properly
\begin{longtable}{|l|l|c|}
\caption{Elementos de hardware disponibles para el desarrollo del sistema}
\label{Est_Hard}\\
\hline
\textbf{EQUIPO} & \textbf{DESCRIPCIÓN}                                                                                                                         & \multicolumn{1}{l|}{\textbf{CANTIDAD}} \\ \hline
\endfirsthead
%
\multicolumn{3}{c}%
{{\bfseries Tabla \thetable\ Continuación de la página anterior}} \\
\endhead
%
Computador      & \begin{tabular}[c]{@{}l@{}}Modelo: HP 15-ac128la\\ Procesador: Intel Core i7-6500U 2.50 GHz\\ RAM: 8 GB\\ Disco duro: 2 TB\end{tabular}      & 1                                      \\ \hline
Computador      & \begin{tabular}[c]{@{}l@{}}Modelo: HP Pavilion 15-cw1092lm\\ Procesador: AMD Ryzen 3 2300U \\ RAM: 12 GB\\ Disco duro: 1 TB\end{tabular}     & 1                                      \\ \hline
Computador      & \begin{tabular}[c]{@{}l@{}}Modelo: TOSHIBA MQ01ABD100\\ Procesador: Intel Core i7-6498DU 2.50GHz\\ RAM: 4 GB\\ Disco duro: 1 TB\end{tabular} & 1                                      \\ \hline
\end{longtable}

\noindent Computador HP 15-ac128la: Este computador se utilizará para desarrollar el sistema.\\
Computador HP Pavilion 15-cw1092lm: Este computador tendrá las tareas de desarrollo y pruebas del sistema.\\
Computador TOSHIBA MQ01ABD100: Este computador se utilizará para desarrollar el sistema.\\

- Software: Todos los computadores mencionados anteriormente tiene los mismos elementos de software.\\
% Please add the following required packages to your document preamble:
% \usepackage{longtable}
% Note: It may be necessary to compile the document several times to get a multi-page table to line up properly
\begin{longtable}{|l|l|l|}
\caption{Elementos de software disponibles para el desarrollo del sistema}
\label{Est_Soft}\\
\hline
\textbf{NOMBRE}    & \textbf{VERSIÓN}                                                                                                     & \textbf{DESCRIPCIÓN}                                                                                                                                    \\ \hline
\endfirsthead
%
\multicolumn{3}{c}%
{{\bfseries Tabla \thetable\ Continuación de la página anterior}} \\
\endhead
%
Linux              & \begin{tabular}[c]{@{}l@{}}Cualquier distribución en \\ su última versión estable.\\ (Basada en Debian)\end{tabular} & \begin{tabular}[c]{@{}l@{}}1GNU/Linux  es un sistema \\ operativo libre tipo Unix \\ POSIX; multiplataforma, \\ multiusuario y multitarea.\end{tabular} \\ \hline
Visual Studio Code & 1.38.0 o última versión estable.                                                                                     & \begin{tabular}[c]{@{}l@{}}Editor de texto enfocado \\ a la creación de código fuente.\end{tabular}                                                     \\ \hline
Python             & 3.7.4 o última versión estable.                                                                                      & \begin{tabular}[c]{@{}l@{}}Lenguaje de programación \\ de alto nivel multipropósito.\end{tabular}                                                       \\ \hline
Git                & 2.23.0 o última versión estable.                                                                                     & Sistema de control de versiones.                                                                                                                        \\ \hline
\end{longtable}
\noindent - Sistema operativo: Se considera el uso del sistema operativo [TAL] para las tareas de diseño, desarrollo y pruebas del sistema ya que ofrece mejores interfaces, es menos restrictivo y permite mayor comodidad a los desarrolladores en cuanto al cumplimiento de las tareas mencionadas previamente.\\

\noindent - Editor de texto: Este editor tiene como objetivo facilitar el proceso de creación de código, haciéndolo más cómodo y comprensible para los desarrolladores.\\

\noindent - Python: Python es un lenguaje de programación interpretado, de alto nivel y de propósito general. Creado por Guido van Rossum y lanzado por primera vez en 1991, la filosofía de diseño de Python enfatiza la legibilidad del código con su uso notable de espacios en blanco significativos.\\

\noindent - Git: Git es un sistema distribuido de control de versiones para rastrear cambios en el código fuente durante el desarrollo de software. Está diseñado para coordinar el trabajo entre programadores, pero se puede usar para rastrear cambios en cualquier conjunto de archivos.\\

\subsection{Factibilidad económica}
- Costos de software:\\
A continuación se presenta una descripción de los costos operativos necesarios para ejecutar los procedimientos del sistema propuesto, lo cual permite apreciar de mejor manera las bondades del mismo.\\
% Please add the following required packages to your document preamble:
% \usepackage{longtable}
% Note: It may be necessary to compile the document several times to get a multi-page table to line up properly
\begin{longtable}{|l|l|l|}
\caption{Costos por software}
\label{Costos_por_Software}\\
\hline
\textbf{NOMBRE}    & \textbf{CANTIDAD} & \textbf{COSTO} \\ \hline
\endfirsthead
%
\multicolumn{3}{c}%
{{\bfseries Tabla \thetable\ Continuación de la página anterior}} \\
\endhead
%
Linux              & 3                 & \$0.00         \\ \hline
Visual Studio Code & 3                 & \$0.00         \\ \hline
Python             & 3                 & \$0.00         \\ \hline
Git                & 2                 & \$0.00         \\ \hline
\multicolumn{2}{|l|}{TOTAL}            & \$0.00               \\ \hline
\end{longtable}
El costo de software para el sistema propuesto es nulo, pues se basa en la adquisición y uso de programas de software libre, que pueden ser obtenidos desde la web sin costo alguno.\\

- Costos de hardware:
La siguiente tabla muestra la depreciación de cada uno de los elementos de hardware que deberán comprarse para soportar la aplicación.\\
% Please add the following required packages to your document preamble:
% \usepackage{longtable}
% Note: It may be necessary to compile the document several times to get a multi-page table to line up properly
\begin{longtable}{|l|l|l|}
\caption{Costos por hardware}
\label{Costos_por_Hardware}\\
\hline
\textbf{NOMBRE}                     & \textbf{CANTIDAD} & \textbf{COSTO} \\ \hline
\endfirsthead
%
\multicolumn{3}{c}%
{{\bfseries Tabla \thetable\ Continuación de la página anterior}} \\
\endhead
%
Computador HP 15-ac128la            & 1                 & \$0.00         \\ \hline
Computador  HP Pavilion 15-cw1092lm & 1                 & \$0.00         \\ \hline
Computador TOSHIBA MQ01ABD100       & 1                 & \$0.00         \\ \hline
\multicolumn{2}{|l|}{TOTAL}                             & \$0.00         \\ \hline
\end{longtable}

Una vez más, puesto que los computadores previamente especificados no generan depreciación en el sentido de si son capaces o no de brindar soporte a la aplicación, el costo vuelve a ser nulo.\\

- Costos de mano de obra:
En la siguiente tabla se agrega de manera meramente informativa, la proposición de la mano de obra requerida para realizar este proyecto y los respectivos sueldos por las horas trabajadas.
% Please add the following required packages to your document preamble:
% \usepackage{longtable}
% Note: It may be necessary to compile the document several times to get a multi-page table to line up properly
\begin{longtable}{|l|l|l|l|}
\caption{Costos por mano de obra}
\label{Costos_por_mano_de_obra}\\
\hline
\textbf{NOMBRE}              & \textbf{CANTIDAD} & \textbf{TIEMPO} & \textbf{\begin{tabular}[c]{@{}l@{}}COSTO EN EL\\ MERCADO\end{tabular}} \\ \hline
\endfirsthead
%
\multicolumn{4}{c}%
{{\bfseries Tabla \thetable\ Continuación de la página anterior}} \\
\endhead
%
Project Manager              & 1                 & 12 meses        & \$15,000.00                                                            \\ \hline
Tester                       & 1                 & 12 meses        & \$17,000.00                                                            \\ \hline
Desarrollador Front-End y UI & 1                 & 12 meses        & \$15,000.00                                                            \\ \hline
Desarrollador Back-End       & 3                 & 12 meses        & \$20,000.00                                                            \\ \hline
TOTAL                        &                   &                 & \$107,000.00                                                           \\ \hline
\end{longtable}
Finalmente, la tabla \ref{Costos_por_mano_de_obra} indica el costo total del sistema propuesto.
% Please add the following required packages to your document preamble:
% \usepackage{longtable}
% Note: It may be necessary to compile the document several times to get a multi-page table to line up properly
\begin{longtable}{|l|c|}
\caption{Costos generados}
\label{Costos_generados}\\
\hline
\textbf{DESCRIPCIÓN} & \multicolumn{1}{l|}{\textbf{COSTO TOTAL}} \\ \hline
\endfirsthead
%
\multicolumn{2}{c}%
{{\bfseries Tabla \thetable\ Continuación de la página anterior}} \\
\endhead
%
Costos software      & \$0.00                                    \\ \hline
Costos hardware      & \$0.00                                    \\ \hline
TOTAL                & \$0.00                                    \\ \hline
\end{longtable}
Como se puede observar, el sistema no genera ningún costo.

\section{Factibilidad Operativa}
Esta parte del estudio se basó en la recolección de datos mediante encuestas realizadas al público objetivo del sistema que se desea desarrollar.
Los resultados arrojan que el sistema es factible operacionalmente, pues la mayoría de las personas encuestadas están de acuerdo con la solución que plantea el sistema al uso de los seres vivos en la experimentación.\\

- Encuesta.
Esta encuesta estuvo dirigida al personal que labora como profesor y a los alumnos de la carrera de Químico Farmacéutico Industrial de la Escuela Nacional de Ciencias Biológicas.\\

\noindent ¿Considera que el proceso para desarrollar nuevos fármacos, es eficiente?\\

\noindent ¿Piensa que las ciencias de la computación podrían ayudar a mejorar o cambiar por completo el proceso para desarrollar nuevos fármacos?\\

\noindent ¿Considera que el trato a los animales en la experimentación es justificable con el fin de mantener la salud de los seres humanos?\\

\noindent ¿Estaría de acuerdo en automatizar el proceso de la experimentación para el desarrollo de nuevos fármacos?\\

\noindent ¿Consideraría viable la construcción de un sistema que predice la efectividad de un fármaco frente a cierta patología?\\

\noindent ¿Cuál de las siguientes palabras cree que describen mejor a un sistema que predice la efectividad de los fármacos?\\
% Please add the following required packages to your document preamble:
% \usepackage{longtable}
% Note: It may be necessary to compile the document several times to get a multi-page table to line up properly
\begin{longtable}{|l|c|}
\caption{Resultados de encuesta}
\label{Resul_encu}\\
\hline
Posibles respuestas & \multicolumn{1}{l|}{Resultados} \\ \hline
\endfirsthead
%
\multicolumn{2}{c}%
{{\bfseries Tabla \thetable\ Continuación de la página anterior}} \\
\endhead
%
Rápido              & 3                               \\ \hline
Intuitivo           & 2                               \\ \hline
Preciso             & 15                              \\ \hline
Ordenado            & 10                              \\ \hline
Fácil de utilizar   & 12                              \\ \hline
Económico           & 8                               \\ \hline
Seguro              & 10                              \\ \hline
\end{longtable}


