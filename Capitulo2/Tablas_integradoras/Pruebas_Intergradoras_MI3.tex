\begin{longtable}{|p{4cm}|p{9.5cm}|}
\caption{Prueba Integradora MI3}\\ 
\hline
\begin{tabular}[c]{@{}l@{}} \textbf{Módulo }\\\textbf{Funcional} \end{tabular}        & MI 3 Generar listado de resultados                                                                                                                                                                                                                                                                                                                                                                                                                                                                                             \endfirsthead 
\hline
\begin{tabular}[c]{@{}l@{}}\textbf{Perfiles}\\\textbf{Implicados} \end{tabular}       & \begin{tabular}[c]{@{}l@{}}- Desarrollador.\\- Tester. \end{tabular}                                                                                                                                                                                                                                                                                                                                                                                                                                                           \\ 
\hline
\begin{tabular}[c]{@{}l@{}}\textbf{Planificación }\\\textbf{temporal} \end{tabular}   & \begin{tabular}[c]{@{}l@{}}1. Revisión de sintaxis al unir módulos secundarios.\\2. Revisión de código completo de cada submódulo.\\2.1 Lectura y ordenamiento del conjunto final.\\2.2 Diseño gráfico de resultados.\end{tabular}                                                                                                                                                                                                                                                                                                \\ 
\hline
\begin{tabular}[c]{@{}l@{}}\textbf{Criterio de }\\\textbf{verificación} \end{tabular} & \begin{tabular}[c]{@{}l@{}}1. Verificación en las herramientas que ofrece el IDE,\\no marque errores de sintaxis .\\2. Verificación de que el código agregado \\corresponde al módulo funcional, y si este requiere\\cierta modificación para la integración, se conserve \\la estructura pragmática del módulo.\\2.1 Integridad de código para lalectura y el \\ordenamiento de resultados obtenidos en el \\conjunto final.\\2.2 Integridad de código para el gráfico deresultados.\end{tabular}                               \\ 
\hline
\begin{tabular}[c]{@{}l@{}}\textbf{Criterio de }\\\textbf{aceptación} \end{tabular}   & \begin{tabular}[c]{@{}l@{}} 1. No hay errores de sintaxis al agregar un módulo \\al programa principal, manteniendo un orden y \\estructura adecuada.\\2. Al agregar un nuevo módulo, el código de este se \\preserva funcionalmente.\\2.1 Lectura y ordenamiento deresultados \\provenientes del conjunto final.~\\2.2 Gráfico de los resultados~\end{tabular}                                                                                                                                                                \\ 
\hline
\textbf{Definición de verificaciones}                                                 & \begin{tabular}[c]{@{}l@{}}- Errores de Compilación: Ocurren porque la \\sintaxis del lenguaje no es correcta, de cajón \\este tipo de errores no permiten que la aplicación \\se ejecute. \\Errores de sintaxis: Un \textbf{error de sintaxis} en \\informáticay programación es una violación a las \\reglas de sintaxis en los lenguajes de programación.\\Se producen cuando la estructura de una de las \\instrucciones infringe una o varias reglas sintácticas \\definidas en ese lenguaje de programación\end{tabular}  \\ 
\hline
\textbf{Análisis y evaluación de resultados}                                          & - Resultados:                                                                                                                                                                                                                                                                                                                                                                                                                                                                                                                  \\ 
\hline
\textbf{Productos a entregar}                                                         & \begin{tabular}[c]{@{}l@{}}-Adquisición del archivo base funcionando \\correctamente.\end{tabular}                                                                                                                                                                                                                                                                                                                                                                                                                              \\
\hline
\end{longtable}

%%%%%%%%%%%%%%%%%%%%%%%%%%%%%%%%%%%%%%%%%%%%%%%%

\begin{longtable}{|p{4cm}|p{9.5cm}|}
\caption{Caso de prueba CPI1}\\ 
\hline
 \textbf{ID del Caso de prueba}                                                                 & CPI3                                                                                                                                                                                                                                                                                                                  \endfirsthead 
\hline
\textbf{Versión}                                                                                & 1.0                                                                                                                                                                                                                                                                                                                   \\ 
\hline
\textbf{Nombre}                                                                                 & \begin{tabular}[c]{@{}l@{}}Caso de prueba de integración para el módulo de \\búsqueda de información.\end{tabular}                                                                                                                                                                                                     \\ 
\hline
\begin{tabular}[c]{@{}l@{}}\textbf{Identificador de }\\\textbf{requerimientos.} \end{tabular}   & RF3.1, RF3.2                                                                                                                                                                                                                                                                                                          \\ 
\hline
\textbf{Propósito}                                                                              & \begin{tabular}[c]{@{}l@{}}Revisar el adecuado ycontinuo funcionamiento de\\cada uno de los componentes que conforman el\\módulode procesamiento de la información. \\Detectar si algún módulo falla en su\\funcionamiento esencial o al momento de recibir \\información o en la muestra deresultados.\end{tabular}  \\ 
\hline
\textbf{Dependencias}                                                                           & N/A                                                                                                                                                                                                                                                                                                                   \\ 
\hline
\textbf{Ambiente de prueba/configuración}                                                       & \begin{tabular}[c]{@{}l@{}}Hardware: Equipo de cómputo (preferentemente \\portátil)\\Software: Compilador python3, IDE y/o editor de \\texto.\end{tabular}                                                                                                                                                               \\ 
\hline
\textbf{Inicialización}                                                                         & \begin{tabular}[c]{@{}l@{}}- Se Revisa que previamente cada uno de los\\requerimientos que conforman el módulo hayan \\pasado el plan de pruebasunitario.\\- Se cuentan con los datos suficientes para \\poderejecutar los componentes.\end{tabular}                                                                  \\ 
\hline
\textbf{Finalización}                                                                           & N/A                                                                                                                                                                                                                                                                                                                   \\ 
\hline
\textbf{Acciones}                                                                               & \begin{tabular}[c]{@{}l@{}}- Agregar el módulo por modulo al programa\\principal, verificando que cada que se agregue \\un componente.\\-~Verificar que losresultados plasmados en el \\conjunto final se en el formato adecuado y \\legibles.\end{tabular}                                                           \\ 
\hline
\begin{tabular}[c]{@{}l@{}}\textbf{Descripción de los }\\\textbf{datos de entrada}\end{tabular} & \begin{tabular}[c]{@{}l@{}}- Conjunto 0 de cada compuesto disponible u\\encontrado.~\\- Aprobación del usuario para iniciar el\\procesamiento de información.\end{tabular}                                                                                                                                            \\ 
\hline
\textbf{Salida esperada}                                                                        & \begin{tabular}[c]{@{}l@{}}- Notificación de errores en la obtención de resultados \\tras la función del módulo de procesamiento.\\- Notificación que informe un error en la lectura\\ de el contenido del conjunto final.\end{tabular}                                                                                \\ 
\hline
 \textbf{Salida obtenida}                                                                       &                                                                                                                                                                                                                                                                                                                       \\ 
\hline
\textbf{Resultado}                                                                              &                                                                                                                                                                                                                                                                                                                       \\ 
\hline
\textbf{Severidad}                                                                              &                                                                                                                                                                                                                                                                                                                       \\ 
\hline
\textbf{Evidencia}                                                                              &                                                                                                                                                                                                                                                                                                                       \\ 
\hline
\textbf{Estado}                                                                                 & No Iniciado.                                                                                                                                                                                                                                                                                                          \\
\hline
\end{longtable}
