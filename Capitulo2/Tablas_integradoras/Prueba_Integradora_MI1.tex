\begin{longtable}{|p{4cm}|p{9.5cm}|}


\caption{Prueba Integradora MI1}\\ 

\endhead
\hline

\begin{tabular}[c]{@{}l@{}}\textbf{Módulo }\\\textbf{Funcional}\end{tabular}         & MI 1 Búsqueda De
Información                                                                                                                                                                                                                                                                                                                                                                                                                                                                                                                                                                                                                                                                                                                                                                                                                                                                                                        \endfirsthead 
\hline
\begin{tabular}[c]{@{}l@{}}\textbf{Perfiles}\\\textbf{Implicados}\end{tabular}       & \begin{tabular}[c]{@{}l@{}}- Desarrollador.\\- Tester.\end{tabular}                                                                                                                                                                                                                                                                                                                                                                                           \\ 
\hline
\begin{tabular}[c]{@{}l@{}}\textbf{Planificación }\\\textbf{temporal}\end{tabular}   & \begin{tabular}[c]{@{}l@{}} 1. Revisión de sintaxis al unir módulos secundarios.\\2. Revisión de código completo de cada submódulo.\\2.1 Adquisición del archivo base. \\2.2 Búsqueda de los compuestos indicados.\\2.3~Búsqueda de los descriptores de los compuestos.\\2.4 Búsqueda de las actividades biológicas de los \\compuestos.\\2.5 Búsqueda de las proteínas indicadas.\\2.6 Confirmación de los resultados de la búsqueda.\\2.7.1 Construcción del conjunto cero.\\2.7.2 Construcción del conjunto P\end{tabular}                                                                                                                                                                                                                                                                                                                                                                                                        \\ 
\hline
\begin{tabular}[c]{@{}l@{}}\textbf{Criterio de }\\\textbf{verificación}\end{tabular} & \begin{tabular}[p{9.5cm}]{@{}l@{}}
1. Verificación en las herramientas que ofrece el IDE,\\ no marque errores de sintaxis.\\2. Verificación de que el código agregado corresponde\\ al módulo funcional, y si este requiere cierta \\modificación para la integración, se conserve la \\estructura pragmática del módulo.\\2.1 Integridad de código para la adquisición \\del archivo base.~ \\2.2 Integridad de código para la búsqueda de los \\compuestos indicados.\\2.3 Integridad de código para la búsqueda de los \\descriptores de los compuestos.\\2.4 Integridad de código para la búsqueda de las \\actividades biológicas de los compuestos.\\2.5 Código funcional para la búsqueda de las \\proteínas indicadas.\\2.6 Funcionalidad en la confirmación de los \\resultados de la búsqueda.\\2.7.1 Código completo para la construcción del \\conjunto cero.\\2.7.2 Código completo para construcción del \\conjunto P.\end{tabular}  \\ 
\hline
\begin{tabular}[c]{@{}l@{}}\textbf{Criterio de }\\\textbf{aceptación}\end{tabular}   & \begin{tabular}[p[9.5cm]{@{}l@{}} 1. No hay errores de sintaxis al agregar un \\módulo al programa principal, manteniendo \\un orden y estructura adecuada.\\2. Al agregar un nuevo módulo, el código de \\este se preserva funcionalmente.  \\2.1 Adquisición del archivo base.~  \\2.2 Búsqueda de los compuestos indicados. \\2.3 Búsqueda de los descriptores de los compuestos. \\2.4 Búsqueda de las actividades biológicas de los \\compuestos. \\2.5 Búsqueda de las proteínas indicadas. \\2.6 Confirmación de los resultados de la búsqueda.\\2.7.1 Construcción del conjunto cero. \\2.7.2 Construcción del conjunto P  \end{tabular}                                                                                                                                                                                                                                                                                  \\ 
\hline
\textbf{Definición de verificaciones}                                                & \begin{tabular}[c]{@{}l@{}}- Errores de Compilación: Ocurren porque la \\sintaxis del lenguaje no es correcta, de cajón este \\tipo de errores no permitenque la aplicación se \\ejecute. \\-Acción de botón: representa un botón que,\\cuando es presionado, envía información al que \\pertenece. La función de~ un botón representada~ \\el contenido del elemento.\\-Visualización de pantalla interfaz de usuario.~\end{tabular}                                                                                                                                                                                                                                                                                                                                                                                                                                                                                                \\ 
\hline
\textbf{Análisis y evaluación de resultados}                                         & - Resultados:~                                                                                                                                                                                                                                                                                                                                                                                                                                                                                                                                                                                                                                                                                                                                                                                                                                                                                                                       \\ 
\hline
\textbf{Productos a entregar}                                                        & \begin{tabular}[c]{@{}l@{}}-Adquisición delarchivo base funcionando \\correctamente.\end{tabular}                                                                                                                                                                                                                                                                                                                                                                                                                                                                                                                                                                                                                                                                                                                                                                                                                                    \\
\hline
\end{longtable}


%%%%%%%%%%%%%%%%%%%%%%%%%%%%%%%%%%%%5

\begin{longtable}{|p{4cm}|p{9.5cm}|}
\caption{Caso de prueba CPI1}\\ 
\hline
\textbf{ID del Caso de prueba}                                                               & CPI1                                                                                                                                                                                                                                           \endfirsthead 
\hline
\textbf{Versión}                                                                             & 1.0                                                                                                                                                                                                                                            \\ 
\hline
\textbf{Nombre}                                                                              & \begin{tabular}[c]{@{}l@{}}Caso de prueba de integración para el módulo de \\búsqueda de información.\end{tabular}                                                                                                                              \\ 
\hline
\begin{tabular}[c]{@{}l@{}}\textbf{Identificador de }\\\textbf{requerimientos.}\end{tabular} & \begin{tabular}[c]{@{}l@{}}RF1.1, RF1.2, RF1.3, RF1.4, RF1.5, RF1.6, RF1.7.1, \\RF1.7.2\end{tabular}                                                                                                                                            \\ 
\hline
\textbf{Propósito~~}                                                                         & \begin{tabular}[c]{@{}l@{}}Poder verificar la continua funcionalidad del módulo \\principal mientras son agregados a el módulo sobre \\el que se está trabajando.\end{tabular}                                                                   \\ 
\hline
\textbf{Dependencias}                                                                        & N/A                                                                                                                                                                                                                                            \\ 
\hline
\textbf{Ambiente de prueba/configuración}                                                    & \begin{tabular}[c]{@{}l@{}}Hardware: Equipo de computo \\(preferentemente portátil)\\Software: Compilador python3, IDE y/o editor de \\texto.\end{tabular}                                                                                       \\ 
\hline
\textbf{Inicialización}                                                                      & \begin{tabular}[c]{@{}l@{}}- Se cuenta con cada uno de los requerimientos \\funcionales que conforman el módulo de la búsqueda \\de información.\\- Se posee el archivo base que es necesario para que \\todo el módulo funcione.\end{tabular}  \\ 
\hline
\textbf{Finalización}                                                                        & N/A                                                                                                                                                                                                                                            \\ 
\hline
\textbf{Acciones}                                                                            & \begin{tabular}[c]{@{}l@{}}- Agregar el módulo por modulo al programa\\ principal, verificando que cada que se agregue un \\componente.\\- Colocar el archivo en el directorio especificado.\end{tabular}                                      \\ 
\hline
\textbf{Salida esperada}                                                                     & \begin{tabular}[c]{@{}l@{}}- Notificación de los resultados obtenidos en la \\búsqueda de~información.\end{tabular}                                                                                                                              \\ 
\hline
   \textbf{Salida obtenida}                                                                  &                                                                                                                                                                                                                                                \\ 
\hline
\textbf{Resultado}~                                                                          &                                                                                                                                                                                                                                                \\ 
\hline
\textbf{Severidad}~                                                                          &                                                                                                                                                                                                                                                \\ 
\hline
\textbf{Evidencia}                                                                           &                                                                                                                                                                                                                                                \\ 
\hline
\textbf{Estado}                                                                              & No Iniciado.                                                                                                                                                                                                                                   \\
\hline
\end{longtable}
