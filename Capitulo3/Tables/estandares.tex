% Please add the following required packages to your document preamble:
% \usepackage{longtable}
% Note: It may be necessary to compile the document several times to get a multi-page table to line up properly
\begin{longtable}{|l|p{3.7cm}|p{4cm}|p{4.7cm}|}
\caption{Especificación de Estándares y Normas de Diseño y Construcción}
\label{estandares}\\
\hline
\textbf{Norma}                                      & \textbf{Descripción}                                                                                                                                                                                                                                                                                                                                                                                        & \textbf{Tareas (Métricas)}                                                                                                                                                                                                                                                         & \textbf{Aplicación al sistema}                                                                                                                                                                                                                                                                                                                                                                                                                                                                                                                                    \\ \hline
\endfirsthead
%
\multicolumn{4}{c}%
{{\bfseries Tabla \thetable\ Continuación de la página anterior}} \\
\endhead
%
\begin{tabular}[c]{@{}l@{}}ISO \\ 9126\end{tabular} & \begin{tabular}[c]{@{}l@{}}El estándar ISO\\ 9126 ha sido\\ desarrollado en un\\ intento de identificar\\ los atributos clave\\ de calidad para el\\ software evalúa los\\ productos de\\ software, esta\\ norma nos indica\\ las características\\ de la calidad y los\\ lineamientos para\\ su uso. El estándar\\ identifica 6 atributos\\ clave de calidad.\end{tabular}                                 & \begin{tabular}[c]{@{}l@{}}Funcionalidad: El\\ grado en que el\\ software satisface las\\ necesidades indicadas\\ por los siguientes\\ subatributos:\\ idoneidad, corrección,\\ interoperatividad,\\ conformidad y\\ seguridad.\end{tabular}                                       & \begin{tabular}[c]{@{}l@{}}- Adquisición del \\archivo base. \\ - Conexión a las bases\\ de datos.\\ - Obtención de los datos \\del compuesto.\\  - Obtención de los datos \\del compuesto.\\  - Correcta creación de \\los archivos contenedores\\ de datos del compuesto.\\  - Correcta creación de los\\ archivos contenedores de \\datos del compuesto.\\  - Lectura de los datos\\ perteneciente a compuestos.\\  - Lectura de los datos\\ perteneciente a la proteína.\\  - Obtención de\\ resultados.\end{tabular} \\ \hline
\begin{tabular}[c]{@{}l@{}}ISO \\ 9126\end{tabular} &                                                                                                                                                                                                                                                                                                                                                                                                             & \begin{tabular}[c]{@{}l@{}}Confiabilidad:\\ Cantidad de tiempo\\ que el software está\\ disponible para su\\ uso. Está referido por\\ los siguientes\\ subatributos: madurez,\\ tolerancia a fallos y\\ facilidad de\\ recuperación.\end{tabular}                                  & \begin{tabular}[c]{@{}l@{}}- Adquisición de los los \\datos de los compuestos.\\- Adquisición de  los datos \\de  la(s) proteína(s).\\- Conexión a las bases de \\datos (PubChem, \\DrugBank, \\ PDB).\\- Generación de resultados.\end{tabular}                                                                                                                                                                                                                                                                                 \\ \hline
\begin{tabular}[c]{@{}l@{}}ISO \\ 9126\end{tabular} &                                                                                                                                                                                                                                                                                                                                                                                                             & \begin{tabular}[c]{@{}l@{}}Usabilidad: Grado en\\ que el software es fácil\\ de usar. Viene\\ reflejado por los\\ siguientes\\ subatributos: facilidad\\ de comprensión,\\ facilidad de\\ aprendizaje y\\ operatividad.\end{tabular}                                               & \begin{tabular}[c]{@{}l@{}}- Interfaces.\\- Adquisición  del archivo \\base.\\- El usuario únicamente \\ debe cargar con  la \\estructura adecuada el \\“ArchivoBase”\end{tabular}                                                                                                                                                                                                                                                                                                                                                                        \\ \hline
\begin{tabular}[c]{@{}l@{}}ISO \\ 9126\end{tabular} &                                                                                                                                                                                                                                                                                                                                                                                                             & \begin{tabular}[c]{@{}l@{}}Eficiencia: Grado en\\ que el software hace\\ óptimo el uso de los\\ recursos del sistema.\\ Está indicado por los\\ siguientes\\ subatributos: tiempo\\ de uso y recursos\\ utilizados.\end{tabular}                                                   & \begin{tabular}[c]{@{}l@{}}- Ponderación asignada \\en el plan de pruebas.\\- Obtención de Resultados.\end{tabular}                                                                                                                                                                                                                                                                                                                                                                                                                               \\ \hline
\begin{tabular}[c]{@{}l@{}}ISO \\ 9126\end{tabular} &                                                                                                                                                                                                                                                                                                                                                                                                             & \begin{tabular}[c]{@{}l@{}}Facilidad de\\ mantenimiento: La\\ facilidad con que una\\ modificación puede\\ ser realizada. Está\\ indicada por los\\ siguientes\\ subatributos: facilidad\\ de análisis, facilidad\\ de cambio, estabilidad\\ y facilidad de prueba.\end{tabular}   & \begin{tabular}[c]{@{}l@{}}- El sistema ha sido \\ dividido en módulos  y \\submódulos permitiendo\\ de ser necesario, el \\ mantenimiento unitario \\y por integración de ellos.\end{tabular}                                                                                                                                                                                                                                                                                                                                                             \\ \hline
\begin{tabular}[c]{@{}l@{}}ISO \\ 9126\end{tabular} &                                                                                                                                                                                                                                                                                                                                                                                                             & \begin{tabular}[c]{@{}l@{}}Portabilidad: La\\ facilidad con que el\\ software puede ser\\ llevado de un entorno\\ a otro. Está referido\\ por los siguientes\\ subatributos: facilidad\\ de instalación,\\ facilidad de ajuste,\\ facilidad de\\ adaptación al cambio\end{tabular} & - TBD(Por definirse)                                                                                                                                                                                                                                                                                                                                                                                                                                                                                                                                               \\ \hline
\begin{tabular}[c]{@{}l@{}}ISO \\ 9241\end{tabular} & \begin{tabular}[c]{@{}l@{}}ISO 9241-210:2010\\ constituye un marco\\ de trabajo para  el\\ diseño  centrado  en\\ las  personas  al\\ integrar diferentes\\ procesos de diseño y\\ desarrollo apropiados\\ a un  contexto  en\\ particular;\\ complementando  las\\ metodologías de\\ diseño  existentes.\\ Para evaluar las\\ interfaces, esta se\\ centra en 6\\ actividades\\ primordiales.\end{tabular} & \begin{tabular}[c]{@{}l@{}}Consistencia: \\ La apariencia de toda \\la interfaz debe ser\\ consistente con los\\ estándares de la\\ industria del software.\end{tabular}                                                                                                           & \begin{tabular}[c]{@{}l@{}}El sistema cuenta con la\\ misma topología de\\ letras, tamaño, colores\\ que se usan e imágenes\\ lo cual lo hace constante.\end{tabular}                                                                                                                                                                                                                                                                                                                                                                                            \\ \hline
\begin{tabular}[c]{@{}l@{}}ISO \\ 9241\end{tabular} &                                                                                                                                                                                                                                                                                                                                                                                                             & \begin{tabular}[c]{@{}l@{}}Facilidad de uso:\\ Deben proporcionarse\\ “atajos” de teclado\\ para los usuarios más\\ experimentados.\end{tabular}                                                                                                                                   & \begin{tabular}[c]{@{}l@{}}El sistema implementa\\ atajos para cargar el\\ archivo con mayor\\ rapidez.\end{tabular}                                                                                                                                                                                                                                                                                                                                                                                                                                               \\ \hline
\begin{tabular}[c]{@{}l@{}}ISO \\ 9241\end{tabular} &                                                                                                                                                                                                                                                                                                                                                                                                             & \begin{tabular}[c]{@{}l@{}}Reducción de la\\ confusión: Debe\\ evitarse la confusión al\\ usuario inexperto por\\ la presentación de\\ excesivas opciones.\end{tabular}                                                                                                            & \begin{tabular}[c]{@{}l@{}}El sistema es bastante\\ claro con cada uno de\\ los botones que\\ proporciona, ya que su\\ función se encuentra en\\ la descripción del\\ mismo.\end{tabular}                                                                                                                                                                                                                                                                                                                                                                          \\ \hline
\begin{tabular}[c]{@{}l@{}}ISO \\ 9241\end{tabular} &                                                                                                                                                                                                                                                                                                                                                                                                             & \begin{tabular}[c]{@{}l@{}}Mensajes al usuario:\\ Los mensajes deben\\ ser claros y útiles. Por\\ ejemplo, en el caso de\\ los mensajes de error,\\ este error debe poder\\ ser corregido con la\\ información\\ proporcionada en el\\ mensaje de aviso.\end{tabular}              & \begin{tabular}[c]{@{}l@{}}Los mensajes que son\\ mostrados al usuario,\\ son claros en cuanto a\\ la excepción que se\\ generó.\end{tabular}                                                                                                                                                                                                                                                                                                                                                                                                                      \\ \hline
\begin{tabular}[c]{@{}l@{}}ISO \\ 9241\end{tabular} &                                                                                                                                                                                                                                                                                                                                                                                                             & \begin{tabular}[c]{@{}l@{}}Iconos: Debe\\ asegurarse que la\\ interfaz proporciona\\ un feedback al usuario\\ apropiado, necesario y\\ consistente.\end{tabular}                                                                                                                   & \begin{tabular}[c]{@{}l@{}}No existen iconos que\\ contengan juegos de\\ palabras o nombres\\ de aplicaciones por lo\\ cual los iconos y sus\\ usos son claros.\end{tabular}                                                                                                                                                                                                                                                                                                                                                                                       \\ \hline
\begin{tabular}[c]{@{}l@{}}ISO \\ 9241\end{tabular} &                                                                                                                                                                                                                                                                                                                                                                                                             & \begin{tabular}[c]{@{}l@{}}Sistemas de ayuda: La\\ ayuda proporcionada\\ debe estar disponible\end{tabular}                                                                                                                                                                        & \begin{tabular}[c]{@{}l@{}}El sistema cuenta con\\ un apartado de ayuda\\ donde el usuario podrá\\ ponerse en contacto a\\ través de correo\\ electrónico para atender\\ sus inquietudes.\end{tabular}                                                                                                                                                                                                                                                                                                                                                             \\ \hline
\end{longtable}

