% Please add the following required packages to your document preamble:
% \usepackage{multirow}
% \usepackage{longtable}
% Note: It may be necessary to compile the document several times to get a multi-page table to line up properly
\begin{longtable}{|l|l|l|l|}
\caption{Identificación de Requisitos de Diseño}
\label{req_dis}\\
\hline
\textbf{\begin{tabular}[c]{@{}l@{}}Módulo de \\ Arquitectura\end{tabular}}              & \textbf{\begin{tabular}[c]{@{}l@{}}Sub-Módulo de \\ Arquitectura\end{tabular}}                                                                                                                                                                                                  & \textbf{\begin{tabular}[c]{@{}l@{}}Requerimiento \\ Funcional\end{tabular}}                                                                 & \textbf{Descripción}                                                                                                                                                                                                                                                                                                                                           \\ \hline
\endfirsthead
%
\multicolumn{4}{c}%
{{\bfseries Tabla \thetable\ Continuación de la página anterior.}} \\
\endhead
%
\begin{tabular}[c]{@{}l@{}}Adquisición\\ de Archivo\\ Base\end{tabular}                 & \begin{tabular}[c]{@{}l@{}}- Lectura de\\ archivo\\ \\ - Selección de\\ archivo.\end{tabular}                                                                                                                                                                                   & \begin{tabular}[c]{@{}l@{}}Requerimiento\\ Funcional 1.1:\\ Obtención y\\ lectura del\\ archivo base.\end{tabular}                          & \begin{tabular}[c]{@{}l@{}}El sistema recibe del\\ usuario un archivo de\\ texto plano en el cual se\\ tiene una lista con los\\ nombres de los\\ compuestos y nombres\\ de las proteínas de la\\ patología que forman\\ parte del objeto de\\ estudio.\end{tabular}                                                                                           \\ 
\multirow{}{}{\begin{tabular}[c]{@{}l@{}}Información\\ de\\ compuestos.\end{tabular}} & \begin{tabular}[c]{@{}l@{}}- Recepción\\ nombre de\\ compuesto.\\ \\ - Obtención de\\ estructura y\\ actividad\\ biológica.\\ \\ - Conexión a\\ DrugBank.\\ \\ -Búsqueda del\\ compuesto.\\ \\ - Adquisición de\\ Estructura\\ molecular.\end{tabular}                          & \begin{tabular}[c]{@{}l@{}}Requerimiento\\ Funcional 1.2:\\ Búsqueda de\\ los compuestos\\ indicados.\end{tabular}                          & \begin{tabular}[c]{@{}l@{}}Por cada elemento de la\\ lista de compuestos en\\ el archivo ingresado por\\ el usuario, el sistema los\\ buscará en la base de\\ datos en línea\\ “DrugBank”.\end{tabular}                                                                                                                                                        \\ \cline{2-4} 
                                                                                        & \begin{tabular}[c]{@{}l@{}}- Recepción\\ nombre de\\ compuesto.\\ \\ - Obtención de\\ estructura y\\ actividad\\ biológica.\\ \\ - Conexión a\\ PubChem.\\ \\ - Conexión a\\ ChemSpider.\\ \\ - Búsqueda del\\ compuesto.\\ \\ -Adquisición de\\ los descriptores.\end{tabular} & \begin{tabular}[c]{@{}l@{}}Requerimiento\\ Funcional 1.3:\\ Búsqueda de\\ los descriptores\\ de los\\ compuestos.\end{tabular}              & \begin{tabular}[c]{@{}l@{}}El sistema, utilizando\\ como referencia la lista\\ de compuestos\\ ingresada por el usuario,\\ realiza una búsqueda de\\ los descriptores físico\\ químicos por cada uno\\ de los elementos de\\ dicha lista en la base de\\ datos “PubChem” y\\ ChemSpider.\end{tabular}                                                          \\ \cline{2-4} 
                                                                                        & \begin{tabular}[c]{@{}l@{}}- Recepción\\ nombre de\\ compuesto.\\ \\ - Obtención de\\ estructura y\\ actividad\\ biológica.\\ \\ - Conexión a\\ DrugBank.\\ \\ - Búsqueda del\\ compuesto.\\ \\ - Adquisición de la\\ actividad\\ biológica.\end{tabular}                       & \begin{tabular}[c]{@{}l@{}}Requerimiento\\ Funcional 1.4:\\ Búsqueda de\\ los mecanismos\\ de acción de los\\ compuestos.\end{tabular}      & \begin{tabular}[c]{@{}l@{}}El sistema, utilizando\\ como referencia la lista\\ de compuestos\\ ingresada por el usuario,\\ realiza la búsqueda de\\ los mecanismos de\\ acción por cada uno de\\ los elementos de dicha\\ lista en la base de datos\\ “DrugBank”.\end{tabular}                                                                                 \\ \hline
\begin{tabular}[c]{@{}l@{}}Información\\ de proteína.\end{tabular}                      & \begin{tabular}[c]{@{}l@{}}- Recepción\\ nombre proteína.\\ \\ - Conexión a PDB.\\ \\ - Búsqueda de la\\ proteína.\\ \\ - Adquisición de\\ estructura\\ molecular.\end{tabular}                                                                                                 & \begin{tabular}[c]{@{}l@{}}Requerimiento\\ Funcional 1.5:\\ Búsqueda de\\ las proteínas\\ indicadas.\end{tabular}                           & \begin{tabular}[c]{@{}l@{}}Por cada elemento de la\\ lista de proteínas en el\\ archivo ingresado por el\\ usuario, el sistema las\\ buscará en la base de\\ datos en línea “PDB”.\end{tabular}                                                                                                                                                                \\ \hline
\begin{tabular}[c]{@{}l@{}}Información\\ de Proteína.\end{tabular}                      & \begin{tabular}[c]{@{}l@{}}Adquisición de\\ estructura\\ molecular\\ proteína.\end{tabular}                                                                                                                                                                                     & \multirow{}{}{\begin{tabular}[c]{@{}l@{}}Requerimiento\\ Funcional 1.6:\\ Confirmación de\\ los resultados\\ de la búsqueda\end{tabular}} & \multirow{}{}{\begin{tabular}[c]{@{}l@{}}El sistema presenta al\\ usuario los resultados\\ de la búsqueda y este\\ último valida que sean\\ correctos.\end{tabular}}                                                                                                                                                                                         \\ \cline{1-2}
\begin{tabular}[c]{@{}l@{}}Información\\ de\\ Compuestos.\end{tabular}                  & \begin{tabular}[c]{@{}l@{}}- Adquisición de\\ estructura\\ molecular\\ compuesto\\ \\ \\ - Adquisición de\\ descriptores.\\ \\ \\ - Adquisición de\\ mecanismos de\\ acción.\end{tabular}                                                                                       &                                                                                                                                             &                                                                                                                                                                                                                                                                                                                                                                \\ \hline
\begin{tabular}[c]{@{}l@{}}Función\\ Archivo\\ “Descriptors”\end{tabular}               & \begin{tabular}[c]{@{}l@{}}- Creación archivo\\ descriptors.\\ \\ - Agregar\\ contenido al\\ archivo.\end{tabular}                                                                                                                                                              & \multirow{}{}{\begin{tabular}[c]{@{}l@{}}Requerimiento\\ Funcional 1.7:\\ Construcción de\\ conjuntos.\end{tabular}}                      & \multirow{}{}{\begin{tabular}[c]{@{}l@{}}El sistema organiza la\\ información recabada\\ durante el proceso de\\ búsqueda, en archivos\\ con estructuras\\ definidas, denominados\\ conjuntos, que permiten\\ continuar a la fase de\\ procesamiento de\\ información.\end{tabular}}                                                                         \\ \cline{1-2}
\begin{tabular}[c]{@{}l@{}}Función\\ Archivo\\ “BioActivity”\end{tabular}               & \begin{tabular}[c]{@{}l@{}}- Creación archivo\\ bioactivity.\\ \\ - Agregar\\ contenido al\\ archivo.\end{tabular}                                                                                                                                                              &                                                                                                                                             &                                                                                                                                                                                                                                                                                                                                                                \\ \hline
\begin{tabular}[c]{@{}l@{}}Construcción \\ Conjunto0\end{tabular}                       & \begin{tabular}[c]{@{}l@{}}- Creación archivo\\ Conjunto0.\\ \\ - Indexación\\ Nombre-Estructura.\\ \\ - Agregar\\ contenido al\\ archivo.\end{tabular}                                                                                                                         & \begin{tabular}[c]{@{}l@{}}Requerimiento\\ Funcional 1.7.1:\\ Construcción\\ del conjunto0\end{tabular}                                     & \begin{tabular}[c]{@{}l@{}}El sistema agrupa los\\ datos de los\\ compuestos (nombre,\\ estructura molecular,\\ mecanismos de acción)\\ en un conjunto de\\ archivos denominado\\ conjunto 0.\end{tabular}                                                                                                                                                     \\ \hline
\begin{tabular}[c]{@{}l@{}}Construcción \\ ConjuntoP\end{tabular}                       & \begin{tabular}[c]{@{}l@{}}- Creación archivo\\ Conjunto0.\\ \\ - Indexación\\ Nombre-Estructura.\\ \\ - Agregar\\ contenido al\\ archivo.\end{tabular}                                                                                                                         & \begin{tabular}[c]{@{}l@{}}Requerimiento\\ Funcional 1.7.2:\\ Construcción\\ del conjuntoP\end{tabular}                                     & \begin{tabular}[c]{@{}l@{}}El sistema agrupa la\\ información de la(s)\\ proteína(s) (nombre de\\ la proteína, estructura\\ molecular de la\\ proteína) de la patología\\ indicada por el usuario,\\ en un archivo\\ denominado conjunto P.\end{tabular}                                                                                                       \\ \hline
\multirow{4}{*}{\begin{tabular}[c]{@{}l@{}}Obtención\\ de\\ Resultados.\end{tabular}}   & \begin{tabular}[c]{@{}l@{}}- Lectura del\\ conjunto0\\ \\ - Descomposición\\ de archivo para\\ datos.\\ \\ - Recepción de\\ datos de\\ compuesto.\end{tabular}                                                                                                                  & \begin{tabular}[c]{@{}l@{}}Requerimiento\\ Funcional 2.1:\\ Adquisición y\\ descomposición\\ del conjunto0.\end{tabular}                    & \begin{tabular}[c]{@{}l@{}}El sistema lee el\\ conjunto 0 (previamente\\ definido)\\ correspondiente a un\\ compuesto, para\\ descomponerlo y \\ obtener nombre,\\ descriptores , estructura\\ molecular y mecanismos\\ de acción.\end{tabular}                                                                                                                \\ \cline{2-4} 
                                                                                        & \begin{tabular}[c]{@{}l@{}}- Lectura del\\ conjuntoP\\ \\ - Descomposición\\ de archivo para\\ datos.\\ \\ - Recepción de\\ datos de la\\ proteína.\end{tabular}                                                                                                                & \begin{tabular}[c]{@{}l@{}}Requerimiento \\ Funcional 2.2:\\ Adquisición y\\ descomposición\\ del conjuntoP.\end{tabular}                   & \begin{tabular}[c]{@{}l@{}}El sistema consigue el\\ nombre de la proteína\\ de interés así como su\\ estructura molecular\\ esta información\\ proveniente del conjunto P.\end{tabular}                                                                                                                                                                        \\ \cline{2-4} 
                                                                                        & \begin{tabular}[c]{@{}l@{}}Optimización\\ por Machine\\ Learning\end{tabular}                                                                                                                                                                                                   & \begin{tabular}[c]{@{}l@{}}Requerimiento\\ Funcional 2.3:\\ Método de\\ Machine\\ Learning\\  para\\ optimización.\end{tabular}             & \begin{tabular}[c]{@{}l@{}}El sistema,   a través de\\ este módulo trabaja  con\\ los datos\\ correspondientes al\\ compuesto  y proteína\\ con el objetivo de\\ optimizar la obtención\\ de resultados.\end{tabular}                                                                                                                                          \\ \cline{2-4} 
                                                                                        & \begin{tabular}[c]{@{}l@{}}- Modelado QSAR\\ \\ -Pre-procesamiento \\ por redes \\ neuronales\\ \\  - Maquina de\\ soporte vectorial\end{tabular}                                                                                                                               & \begin{tabular}[c]{@{}l@{}}Requerimiento\\ Funcional 2.4:\\ Modelado\\ QSAR.\end{tabular}                                                   & \begin{tabular}[c]{@{}l@{}}El sistema producirá\\ con los datos\\ pertenecientes a la\\ proteína y  compuesto\\ de interés  el modelo\\ QSAR propio de la\\ actividad farmacológica\\ entre el fármaco y la\\ proteína objetivo.\end{tabular}                                                                                                                  \\ \hline
\begin{tabular}[c]{@{}l@{}}Graficación\\ de\\ resultados.\end{tabular}                  & \begin{tabular}[c]{@{}l@{}}Adquisición de\\ resultados.\end{tabular}                                                                                                                                                                                                            & \begin{tabular}[c]{@{}l@{}}Requerimiento\\ Funcional 3.1:\\ Lectura y\\ ordenamiento\\ del conjunto\\ final.\end{tabular}                   & \begin{tabular}[c]{@{}l@{}}El sistema recibe el\\ conjunto final que\\ contiene el nombre del\\ compuesto y la  aptitud,\\ dicha información será\\ leída y ordena del más\\ apto al menos apto para\\ enfrentar la proteína de\\ la  patología indicada. A\\ la salida se tiene un\\ archivo de texto plano\\ que contiene una lista\\ ordenada.\end{tabular} \\ \hline
                                                                                        & \begin{tabular}[c]{@{}l@{}}Visualización de\\ resultados.\end{tabular}                                                                                                                                                                                                          & \begin{tabular}[c]{@{}l@{}}Requerimiento\\ Funcional 3.2:\\ Diseño gráfico\\ de resultados.\end{tabular}                                    & \begin{tabular}[c]{@{}l@{}}El sistema obtiene el\\ archivo que contiene la\\ lista ordenada  la cual\\ será utilizada para\\ desplegar de manera\\ gráfica  al usuario.\end{tabular}                                                                                                                                                                           \\ \hline
\end{longtable}