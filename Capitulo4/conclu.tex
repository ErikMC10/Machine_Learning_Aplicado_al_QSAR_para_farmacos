\lhead{Capítulo \ref{ch_6}}
\rhead{\newtitle}
\cfoot{\thepage}
\renewcommand{\headrulewidth}{1pt}
\renewcommand{\footrulewidth}{1pt}

\chapter{Conclusiones}\label{ch_6}


\section{Luis Enrique García Peregrino}{
\noindent La comprehensión y el desarrollo de la metodología de QSAR así como la implementación de algoritmos de Machine Learning permitieron observar la gran cantidad de problemas en las ciencias médico biológicas que al día de hoy son analizados por la bioinformática en busca de mejores soluciones a las ya existentes. El uso de los sistemas en general y de SisPAF en particular para asistir a la farmacología en el desarrollo de nuevos fármacos o en la mejora de los ya existentes permite contemplar la idea de que, para este tipo de análisis, pronto los sistemas de información dejarán de ser una opción experimental para convertirse en la mejor opción disponible.
}

\section{Adolfo Erik Morales Castellanos}{
\noindent Durante la realizaciòn de este proyecto, desde la realización del protocolo, donde surgió la idea de cómo podría mejorarse el proceso de experimentación en el campo quimico-farmaceutico, hasta este punto, finalizando el trabajo terminal 2 me he llevado un gran aprendizaje, desde reforzar todo lo aprendido en la carrera de sistemas computacionales, he adquirido conocimiento y hasta cierto punto experiencia en el campo de la  bioinformática.

\noindent Con el desarrollo de un sistema como es  SisPAF, uno se da cuenta que hay mucho trabajo por delante, si bien, proponemos innovaciones aplicadas al sistema, como lo es la adquisiciòn de la información de manera automática, y el uso de QSAR junto con Machine Learning , con el trabajo realizado, hemos aportado un poco en el avance de la bioinformática explotando todo lo que las ciencias de la computación para trabajar en conjunto y llevar a cabo trabajos interdisciplinarios.
}

\section{Esteban Sánchez Cuevas}{
\noindent Durante el desarrollo de SisPAF, se aprendieron muchos conceptos, que van más allá de la computación, el adentrarse en otro campo como es la biología y la química, logramos observar  como las ciencias computacionales actualmente son requeridas en gran diversidad de ámbitos. He logrado darme cuenta, como la computación, la tecnología puede desarrollarse de manera interdisciplinaria y lograr nuevos avances, o mejorar algo   existente. 

\noindent Para mi caso, he tenido la información de entender, cómo la computación permite abrir nuevos horizontes, en este caso, la bioinformática es una área en la cual hay mucho por aprender e innovar, en mi opinión hemos aportado nuestro grano de arena en crear un sistema que se valga de machine learning para complementar un método de la bioinformática como es QSAR. Aunado a esto, se ha agilizado un proceso que suele requerir tiempo y esfuerzo, la adquisición de datos suele llevar una considerable cantidad de estos, incluso valiéndose de la red, pues se requiere buscar compuesto por compuesto, proteína por proteína, sin considerar los datos que se deben obtener de diferentes bases de datos.

\noindent Sin duda, aún hay cosas que se pueden implementar a SisPAF, trabajo a futuro que puede causar un mayor impacto de esta idea que surgió desde la realización de protocolo, donde comenzó todo a partir del pensamiento de cómo mejorar el proceso de experimentación de fármacos. 
 }
