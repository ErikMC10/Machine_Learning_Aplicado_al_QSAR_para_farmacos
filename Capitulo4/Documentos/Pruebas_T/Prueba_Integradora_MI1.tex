

%%%%%%%%%%%%%%%%%%%%%%%%%%%%%%%%%%%%5

\begin{longtable}{|p{4cm}|p{9.5cm}|}
\caption{Caso de prueba CPI1}\\ 
\hline
\textbf{ID del Caso de prueba}                                                               & CPI1                                                                                                                                                                                                                                           \endfirsthead 
\hline
\textbf{Versión}                                                                             & 1.0                                                                                                                                                                                                                                            \\ 
\hline
\textbf{Nombre}                                                                              & \begin{tabular}[c]{@{}l@{}}Caso de prueba de integración para el módulo de \\búsqueda de información.\end{tabular}                                                                                                                              \\ 
\hline
\begin{tabular}[c]{@{}l@{}}\textbf{Identificador de }\\\textbf{requerimientos.}\end{tabular} & \begin{tabular}[c]{@{}l@{}}RF1.1, RF1.2, RF1.3, RF1.4, RF1.5, RF1.6, RF1.7.1, \\RF1.7.2\end{tabular}                                                                                                                                            \\ 
\hline
\textbf{Propósito~~}                                                                         & \begin{tabular}[c]{@{}l@{}}Poder verificar la continua funcionalidad del módulo \\principal mientras son agregados a el módulo sobre \\el que se está trabajando.\end{tabular}                                                                   \\ 
\hline
\textbf{Dependencias}                                                                        & N/A                                                                                                                                                                                                                                            \\ 
\hline
\textbf{Ambiente de prueba/configuración}                                                    & \begin{tabular}[c]{@{}l@{}}Hardware: Equipo de computo \\(preferentemente portátil)\\Software: Compilador python3, IDE y/o editor de \\texto.\end{tabular}                                                                                       \\ 
\hline
\textbf{Inicialización}                                                                      & \begin{tabular}[c]{@{}l@{}}- Se cuenta con cada uno de los requerimientos \\funcionales que conforman el módulo de la búsqueda \\de información.\\- Se posee el archivo base que es necesario para que \\todo el módulo funcione.\end{tabular}  \\ 
\hline
\textbf{Finalización}                                                                        & N/A                                                                                                                                                                                                                                            \\ 
\hline
\textbf{Acciones}                                                                            & \begin{tabular}[c]{@{}l@{}}- Agregar el módulo por modulo al programa\\ principal, verificando que cada que se agregue un \\componente.\\- Colocar el archivo en el directorio especificado.\end{tabular}                                      \\ 
\hline
\textbf{Descripción de los datos de entrada}
&   \begin{tabular}[c]{@{}l@{}}
- Archivo de texto plano.\\
- Directorio de la ubicación del archivo base.
\end{tabular}\\
\hline
\textbf{Salida esperada}                                                                     & \begin{tabular}[c]{@{}l@{}}
El funcionamiento de la obtención de los datos de \\
compuestos y proteínas a partir de la adquisición del\\
Conjunto Inicial para concluir en la muestra de \\resultados
de la búsqueda de datos.\end{tabular}                                                                                                                              \\ 
\hline
   \textbf{Salida obtenida}                                                                  &        \begin{tabular}[c]{@{}l@{}}
- A Través de MessageBox se notifica al usuario si \\
existe algún problema durante la búsqueda de \\
información, como un cierre inesperado o una pérdida\\ 
de conexión.\\
- Al finalizar la búsqueda exitosamente, los resultados\\
se pueden observar en la interfaz.
\end{tabular}                                                                                                                                            \\ 
\hline
\textbf{Severidad}~                                                                          &     Baja                                                                                                                                                                                                                                           \\ 
\hline
\textbf{Evidencia}                                                                           &  Evidencia en la imagen \ref{M1Integracion}                                                                                                                                                                                                                                               \\ 
\hline
\textbf{Estado}                                                                              & Iniciado.                                                                                                                                                                                                                                   \\
\hline
\end{longtable}
