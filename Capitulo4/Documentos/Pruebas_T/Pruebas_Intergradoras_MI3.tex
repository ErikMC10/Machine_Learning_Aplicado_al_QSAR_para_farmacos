

%%%%%%%%%%%%%%%%%%%%%%%%%%%%%%%%%%%%%%%%%%%%%%%%

\begin{longtable}{|p{4cm}|p{9.5cm}|}
\caption{Caso de prueba CPI3}\\ 
\hline
 \textbf{ID del Caso de prueba}                                                                 & CPI3                                                                                                                                                                                                                                                                                                                  \endfirsthead 
\hline
\textbf{Versión}                                                                                & 1.0                                                                                                                                                                                                                                                                                                                   \\ 
\hline
\textbf{Nombre}                                                                                 & \begin{tabular}[c]{@{}l@{}}Caso de prueba de integración para el módulo de \\búsqueda de información.\end{tabular}                                                                                                                                                                                                     \\ 
\hline
\begin{tabular}[c]{@{}l@{}}\textbf{Identificador de }\\\textbf{requerimientos.} \end{tabular}   & RF3.1, RF3.2                                                                                                                                                                                                                                                                                                          \\ 
\hline
\textbf{Propósito}                                                                              & \begin{tabular}[c]{@{}l@{}}Revisar el adecuado ycontinuo funcionamiento de\\cada uno de los componentes que conforman el\\módulode procesamiento de la información. \\Detectar si algún módulo falla en su\\funcionamiento esencial o al momento de recibir \\información o en la muestra deresultados.\end{tabular}  \\ 
\hline
\textbf{Dependencias}                                                                           & N/A                                                                                                                                                                                                                                                                                                                   \\ 
\hline
\textbf{Ambiente de prueba/configuración}                                                       & \begin{tabular}[c]{@{}l@{}}Hardware: Equipo de cómputo (preferentemente \\portátil)\\Software: Compilador python3, IDE y/o editor de \\texto.\end{tabular}                                                                                                                                                               \\ 
\hline
\textbf{Inicialización}                                                                         & \begin{tabular}[c]{@{}l@{}}- Se Revisa que previamente cada uno de los\\requerimientos que conforman el módulo hayan \\pasado el plan de pruebas unitario.\\- Se cuentan con los datos suficientes para \\poderejecutar los componentes.\end{tabular}                                                                  \\ 
\hline
\textbf{Finalización}                                                                           & N/A                                                                                                                                                                                                                                                                                                                   \\ 
\hline
\textbf{Acciones}                                                                               & \begin{tabular}[c]{@{}l@{}}- Agregar el módulo por modulo al programa\\principal, verificando que cada que se agregue \\un componente.\\-~Verificar que losresultados plasmados en el \\conjunto final se en el formato adecuado y \\legibles.\end{tabular}                                                           \\ 
\hline
\begin{tabular}[c]{@{}l@{}}\textbf{Descripción de los }\\\textbf{datos de entrada}\end{tabular} & \begin{tabular}[c]{@{}l@{}}- Conjunto 0 de cada compuesto disponible u\\encontrado.~\\- Aprobación del usuario para iniciar el\\procesamiento de información.\end{tabular}                                                                                                                                            \\ 
\hline
\textbf{Salida esperada}                                                                        & \begin{tabular}[c]{@{}l@{}}- Notificación de errores en la obtención de resultados \\tras la función del módulo de procesamiento.\\- Notificación que informe un error en la lectura\\ de el contenido del conjunto final.\end{tabular}                                                                                \\ 
\hline
 \textbf{Salida obtenida}                                                                       & \begin{tabular}[c]{@{}l@{}}
- En la terminal, se observan los resultados obtenidos en\\
la regresión lineal y su evaluación en el modelado QSAR.\\
- Interfaz Gráfica que muestra el listado de compuestos\\
ordenados por efectividad.\end{tabular}                                                                                                                                                                                                                                                                                                                       \\ 
\hline
\textbf{Resultado}                                                                              &   \begin{tabular}[c]{@{}l@{}}
- El observar los resultados en la terminal, brinda la\\
seguridad de que el proceso se realizó adecuadamente.\\
- La visualización gráfica del listado ordenado, \\
funciona de igual forma adecuada.

\end{tabular}                                                                                                                                                                                                                                                                                                                     \\ 
\hline
\textbf{Severidad}                                                                              &    Nula                                                                                                                                                                                                                                                                                                                   \\ 
\hline
\textbf{Evidencia}                                                                              &     Evidencia en la imagen \ref{M3Integradora}                                                                                                                                                                                                                                                                                                                 \\ 
\hline
\textbf{Estado}                                                                                 & No Iniciado.                                                                                                                                                                                                                                                                                                          \\
\hline
\end{longtable}
