


%%%%%%%%%%%%%%%%%%%%%%%%%%%%%%%%%%%

\begin{longtable}{|p{4cm}|p{9.5cm}|}
\caption{Caso de prueba CPI2}\\ 
\hline
 \textbf{ID del Caso de prueba}                                                                 & CPI2                                                                                                                                                                                                                                                                                                                        \endfirsthead 
\hline
\textbf{Versión}                                                                                & 1.0                                                                                                                                                                                                                                                                                                                         \\ 
\hline
\textbf{Nombre}                                                                                 & \begin{tabular}[c]{@{}l@{}}Caso de prueba de integración para el módulo de \\búsqueda de información.\end{tabular}                                                                                                                                                                                                           \\ 
\hline
\begin{tabular}[c]{@{}l@{}}\textbf{Identificador de }\\\textbf{requerimientos.} \end{tabular}   & RF2.1, RF2.2, RF2.3, RF2.4                                                                                                                                                                                                                                                                                                  \\ 
\hline
\textbf{Propósito}                                                                              & \begin{tabular}[c]{@{}l@{}}Revisar el adecuado y continuo funcionamiento \\de cada uno de los componentes que conforman \\el módulo de procesamiento de la información. \\Detectar si algún módulo falla en su\\funcionamiento esencial o al momento de recibir \\información o en la transmisión de resultados.\end{tabular}  \\ 
\hline
\textbf{Dependencias}                                                                           & N/A                                                                                                                                                                                                                                                                                                                         \\ 
\hline
\textbf{Ambiente de prueba/configuración}                                                       & \begin{tabular}[c]{@{}l@{}}Hardware: Equipo de cómputo (preferentemente \\portátil)\\Software: Compilador python3, IDE y/o editor de \\texto.\end{tabular}                                                                                                                                                                     \\ 
\hline
\textbf{Inicialización}                                                                         & \begin{tabular}[c]{@{}l@{}}- Se Revisa que previamente cada uno de los\\requerimientos que conforman el módulo hayan \\pasado el plan de pruebas unitario.\\- Se cuentan con los datos suficientes para \\poderejecutar los componentes.\end{tabular}                                                                        \\ 
\hline
\textbf{Finalización}                                                                           & N/A                                                                                                                                                                                                                                                                                                                         \\ 
\hline
\textbf{Acciones}                                                                               & \begin{tabular}[c]{@{}l@{}}- Agregar el módulo por modulo al programa\\principal, verificando que cada que se agregue un \\componente.\\- Verificar que los conjuntos 0 y P de cada\\compuesto y proteína están íntegros.\end{tabular}                                                                                      \\ 
\hline

\begin{tabular}[c]{@{}l@{}}\textbf{Descripción de los }\\\textbf{datos de entrada}\end{tabular} & \begin{tabular}[c]{@{}l@{}}- Conjunto 0 de cada compuesto disponible u\\encontrado.\\- Conjunto P de cada compuesto disponible u\\encontrado.\\- Aprobación del usuario para iniciar el \\procesamiento de información.\end{tabular}                                                                                         \\ 
\hline
\textbf{Salida esperada}                                                                        & \begin{tabular}[c]{@{}l@{}}- Notificación de los resultados obtenidos en la\\búsqueda de información. Notificación que informe \\si existió una falla durante el procesamiento.\end{tabular}                                                                                                                                \\ 
\hline
 \textbf{Salida obtenida}                                                                       &   \begin{tabular}[c]{@{}l@{}}
 - El módulo , contando con el archivo inicial con la\\
información necesaria, de existir algún fallo, el \\
sistema informa al usuario por medio de Message Box,\\
y de ser necesario retienen el funcionamiento continuo.\\
- Al finalizar la búsqueda el usuario puede visualizar\\
que compuestos y proteínas fueron encontrados\\ adecuadamente.
 \end{tabular}                                                                                                                                                                                                                                                                                                                          \\ 
\hline
\textbf{Resultado}                                                                              &   \begin{tabular}[c]{@{}l@{}}
- El módulo en general funciona, contando con el \\
archivo inicial con la información necesaria, de \\
existir algún fallo, el sistema informa al usuario,\\
y de ser necesario retienen el funcionamiento continuo.
 \end{tabular}                                                                                                                                                                                                                                                                                                                          \\ 
\hline
\textbf{Severidad}                                                                              &    Nula                                                                                                                                                                                                                                                                                                                       \\ 
\hline
\textbf{Evidencia}                                                                              &    Evidencia en la imagen \ref{M2Integracion}                                                                                                                                                                                                                                                                                                                         \\ 
\hline
\textbf{Estado}                                                                                 & No Iniciado.                                                                                                                                                                                                                                                                                                                \\
\hline
\end{longtable}
