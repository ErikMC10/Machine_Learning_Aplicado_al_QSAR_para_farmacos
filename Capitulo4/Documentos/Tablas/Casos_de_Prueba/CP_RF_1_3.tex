% Please add the following required packages to your document preamble:
% \usepackage{longtable}
% Note: It may be necessary to compile the document several times to get a multi-page table to line up properly
\begin{longtable}{|l|l|}
\caption{Caso de prueba RF1.3}
\label{CP_RF1_3}\\
\hline
\textbf{ID del Caso de prueba}                                                          & CPRF3                                                                                                                                                                                                                                                          \\ \hline
\endfirsthead
%
\multicolumn{2}{c}%
{{\bfseries Tabla \thetable\ Continuación de la página anterior}} \\
\endhead
%
\textbf{Versión}                                                                        & 1.0                                                                                                                                                                                                                                                            \\ \hline
\textbf{Nombre}                                                                         & \begin{tabular}[c]{@{}l@{}}Caso de prueba para la búsqueda de los \\ descriptores de los compuestos.\end{tabular}                                                                                                                                              \\ \hline
\textbf{\begin{tabular}[c]{@{}l@{}}Identificador de \\ requerimiento\end{tabular}}      & RF1.3                                                                                                                                                                                                                                                          \\ \hline
\textbf{Propósito}                                                                      & \begin{tabular}[c]{@{}l@{}}Identificar el funcionamiento del sistema al \\ momento de conectarse a las dos bases de datos \\ contenedoras de descriptores de compuestos, \\ “ChemSpider” y “PubChem”.\end{tabular}                                             \\ \hline
\textbf{Dependencias}                                                                   & Correcta obtención del archivo base.                                                                                                                                                                                                                           \\ \hline
\textbf{\begin{tabular}[c]{@{}l@{}}Ambiente de \\ prueba/configuración\end{tabular}}    & \begin{tabular}[c]{@{}l@{}}- Hardware: Equipo de computo\\ (preferentemente portatíl)\\ - Software: Compilador python3, \\ IDE y/o editor de texto.\end{tabular}                                                                                               \\ \hline
\textbf{Inicialización}                                                                 & \begin{tabular}[c]{@{}l@{}}- Codificación correspondiente al \\ requerimiento.\\ - Creación del archivo base.\end{tabular}                                                                                                                                     \\ \hline
\textbf{Finalización}                                                                   & N/A                                                                                                                                                                                                                                                            \\ \hline
\textbf{Acciones}                                                                       & \begin{tabular}[c]{@{}l@{}}. Compilar el código correspondiente.\\ - Contar con el archivo base previamente \\ cargado.\end{tabular}                                                                                                                           \\ \hline
\textbf{\begin{tabular}[c]{@{}l@{}}Descripción de los \\ datos de entrada\end{tabular}} & \begin{tabular}[c]{@{}l@{}}- Nombre del compuesto.\\ - Dirección para la conexión a \\ “ChemSpider” y “PubChem”.\end{tabular}                                                                                                                                  \\ \hline
\textbf{Salida esperada}                                                                & \begin{tabular}[c]{@{}l@{}}- Notificación de adecuada estado del \\ compuesto en\\ “ChemSpider” y “PubChem”. (Existente o no).\\ - De existir, informar que el compuesto existe \\ al igual que el archivo .pdb\\ - Adquisición correcta del .pdb\end{tabular} \\ \hline
\textbf{Salida obtenida}                                                                &   \begin{tabular}[c]{@{}l@{}}- Se ha optado por únicamente valerse de \\los descriptores existentes en PubChem.\\
- La notificación de la obtención \\de resultados se realiza al finalizar toda \\la búsqueda de información necesaria.\\
- En la terminal se muestra el avance que \\se lleva con la búsqueda.
\end{tabular}                                                                                                                                                                                                                                             \\ \hline
\textbf{Resultado}                                                                      &   \begin{tabular}[c]{@{}l@{}}- Muestra de avance de la búsqueda general,\\ por terminal y por medio de la interfaz, \\a través de una barra de proceso.
\end{tabular}                                                                                                                                                                                                                                                             \\ \hline
\textbf{Severidad}                                                                      &    - Baja                                                                                                                                                                                                                                                            \\ \hline
\textbf{Evidencia}                                                                      &  Evidencia en la imagen \ref{CPRF2-3-4}                                                                                                                                                                                                                                                         \\ \hline
\textbf{Estado}                                                                         & Iniciado.                                                                                                                                                                                                                                                   \\ \hline
\end{longtable}