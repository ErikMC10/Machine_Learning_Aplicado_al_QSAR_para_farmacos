% Please add the following required packages to your document preamble:
% \usepackage{longtable}
% Note: It may be necessary to compile the document several times to get a multi-page table to line up properly
\begin{longtable}{|l|l|}
\caption{Caso de prueba RF2.3}
\label{CP_RF2_3}\\
\hline
\textbf{ID del Caso de prueba}                                                          & CPRF11                                                                                                                                                                           \\ \hline
\endfirsthead
%
\multicolumn{2}{c}%
{{\bfseries Tabla \thetable\ Continuación de la página anterior}} \\
\endhead
%
\textbf{Versión}                                                                        & 1.0                                                                                                                                                                              \\ \hline
\textbf{Nombre}                                                                         & \begin{tabular}[c]{@{}l@{}}Caso de prueba para el método \\ de  \textit{Machine Learning} para optimización.\end{tabular}                                                                 \\ \hline
\textbf{\begin{tabular}[c]{@{}l@{}}Identificador de \\ requerimiento\end{tabular}}      & RF2.3                                                                                                                                                                            \\ \hline
\textbf{Propósito}                                                                      & \begin{tabular}[c]{@{}l@{}}Identificar el funcionamiento el algoritmo de \\ \textit{Machine Learning} implementado al sistema \\ para la optimización de respuesta.\end{tabular}         \\ \hline
\textbf{Dependencias}                                                                   & \begin{tabular}[c]{@{}l@{}}- Correcta obtención de los datos provenientes\\ del conjunto0 y conjuntoP.\end{tabular}                                                             \\ \hline
\textbf{\begin{tabular}[c]{@{}l@{}}Ambiente de \\ prueba/configuración\end{tabular}}    & \begin{tabular}[c]{@{}l@{}}- Hardware: Equipo de computo\\ (preferentemente portatíl)\\ - Software: Compilador python3, \\ IDE y/o editor de texto.\end{tabular}                 \\ \hline
\textbf{Inicialización}                                                                 & \begin{tabular}[c]{@{}l@{}}- Codificación correspondiente al requerimiento.\\ - Mantenimiento del valor establecido a las \\ variables de los compuesto y proteína.\end{tabular} \\ \hline
\textbf{Finalización}                                                                   & N/A                                                                                                                                                                              \\ \hline
\textbf{Acciones}                                                                       & \begin{tabular}[c]{@{}l@{}}- Compilar el código correspondiente.\\ - Contar los datos necesarios del compuesto \\ y proteína.\end{tabular}                                       \\ \hline
\textbf{\begin{tabular}[c]{@{}l@{}}Descripción de los \\ datos de entrada\end{tabular}} & \begin{tabular}[c]{@{}l@{}}- Nombre del compuesto.\\ - Descriptores del compuesto.\\ - Estructura  molecular del compuesto.\\ - Actividad biológica del compuesto\end{tabular}   \\ \hline
\textbf{Salida esperada}                                                                & \begin{tabular}[c]{@{}l@{}}- Notificación del estado del procesamiento \\ de datos, ya se correcto o incorrecto.\end{tabular}                                                    \\ \hline
\textbf{Salida obtenida}                                                                &   \begin{tabular}[c]{@{}l@{}}
- A través de la terminal se visualiza el \\
resultado del algoritmo de Machine learning \\
usado en en procesamiento de información.
\end{tabular}                                                                                                                                                                               \\ \hline
\textbf{Resultado}                                                                      &   \begin{tabular}[c]{@{}l@{}}
- Se logra observar el resultado de la regresión\\
lineal aplicada al procesamiento de la información.
\end{tabular}                                                                                                                                                                               \\ \hline
\textbf{Severidad}                                                                      &     Nula                                                                                                                                                                             \\ \hline
\textbf{Evidencia}                                                                      &    Evidencia en la pagina \ref{CPRF11}                                                                                                                                                                              \\ \hline
\textbf{Estado}                                                                         & Iniciado.                                                                                                                                                                     \\ \hline
\end{longtable}