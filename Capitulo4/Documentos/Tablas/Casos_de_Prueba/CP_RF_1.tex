% Please add the following required packages to your document preamble:
% \usepackage{longtable}
% Note: It may be necessary to compile the document several times to get a multi-page table to line up properly
\begin{longtable}{|l|l|}
\caption{Caso de prueba RF1}
\label{CP_RF1}\\
\hline
\textbf{ID del Caso de prueba}                                                          & CPRF1                                                                                                                                                            \\ \hline
\endfirsthead
%
\multicolumn{2}{c}%
{{\bfseries Tabla \thetable\ Continuación de la página anterior}} \\
\endhead
%
\textbf{Versión}                                                                        & 1.0                                                                                                                                                              \\ \hline
\textbf{Nombre}                                                                         & Caso de prueba para adquisición de archivo base.                                                                                                                 \\ \hline
\textbf{\begin{tabular}[c]{@{}l@{}}Identificador de \\ requerimiento\end{tabular}}      & RF1.1                                                                                                                                                            \\ \hline
\textbf{Propósito}                                                                      & \begin{tabular}[c]{@{}l@{}}Determinar capacidad del sistema para \\ obtener el archivo base.\end{tabular}                                                        \\ \hline
\textbf{Dependencias}                                                                   & N/A                                                                                                                                                              \\ \hline
\textbf{\begin{tabular}[c]{@{}l@{}}Ambiente de \\ prueba/configuración\end{tabular}}    & \begin{tabular}[c]{@{}l@{}}- Hardware: Equipo de computo\\ (preferentemente portatíl)\\ - Software: Compilador python3, \\ IDE y/o editor de texto.\end{tabular} \\ \hline
\textbf{Inicialización}                                                                 & \begin{tabular}[c]{@{}l@{}}- Codificación correspondiente al \\ requerimiento.\\ - Creación del archivo base.\end{tabular}                                       \\ \hline
\textbf{Finalización}                                                                   & N/A                                                                                                                                                              \\ \hline
\textbf{Acciones}                                                                       & \begin{tabular}[c]{@{}l@{}}. Compilar el código correspondiente.\\ - Colocar el archivo en el directorio \\ especificado.\end{tabular}                           \\ \hline
\textbf{\begin{tabular}[c]{@{}l@{}}Descripción de los \\ datos de entrada\end{tabular}} & \begin{tabular}[c]{@{}l@{}}- Archivo de texto plano.\\ - Directorio de la ubicación del archivo base.\end{tabular}                                               \\ \hline
\textbf{Salida esperada}                                                                & \begin{tabular}[c]{@{}l@{}}- Notificación de adecuada adquisición del \\ archivo base.\end{tabular}                                                              \\ \hline
\textbf{Salida obtenida}                                                                &  \begin{tabular}[c]{@{}l@{}}- Cuadro de diálogo para la obtención de archivos.\\
- Diálogo para indicar que un archivo es incorrecto.\\
- Sin diálogo para cuando el archivo es correcto;\\ Se sustituye por la habilitación de botón de inicio.\end{tabular}                                                                                                                                              
\\ \hline
\textbf{Resultado}                                                                      & \begin{tabular}[c]{@{}l@{}}- El módulo funciona adecuadamente,
aparecen los \\diálogos necesarios para la
interpretación  del usuario.\end{tabular}                                                                                                                                               
\\ \hline
\textbf{Severidad}                                                                      & - Baja                                                                                                                                                      \\ \hline
\textbf{Evidencia}                                                                      &   \begin{tabular}[c]{@{}l@{}} Evidencia en la imagen \ref{CPRF1}.\end{tabular}                                                                                                                                                            \\ \hline
\textbf{Estado}                                                                         & Iniciado                                                                                                                                                    \\ \hline
\end{longtable}