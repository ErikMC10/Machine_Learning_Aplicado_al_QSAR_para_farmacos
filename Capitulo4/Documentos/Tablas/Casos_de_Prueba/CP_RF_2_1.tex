% Please add the following required packages to your document preamble:
% \usepackage{longtable}
% Note: It may be necessary to compile the document several times to get a multi-page table to line up properly
\begin{longtable}{|l|l|}
\caption{Caso de prueba RF2.1}
\label{CP_RF2_1}\\
\hline
\textbf{ID del Caso de prueba}                                                          & CPRF9                                                                                                                                                                          \\ \hline
\endfirsthead
%
\multicolumn{2}{c}%
{{\bfseries Tabla \thetable\ Continuación de la página anterior}} \\
\endhead
%
\textbf{Versión}                                                                        & 1.0                                                                                                                                                                            \\ \hline
\textbf{Nombre}                                                                         & \begin{tabular}[c]{@{}l@{}}Caso de prueba para Adquisición\\  y descomposición del conjunto0.\end{tabular}                                                                     \\ \hline
\textbf{\begin{tabular}[c]{@{}l@{}}Identificador de \\ requerimiento\end{tabular}}      & RF2.1                                                                                                                                                                          \\ \hline
\textbf{Propósito}                                                                      & \begin{tabular}[c]{@{}l@{}}- Identificar el funcionamiento del sistema \\ para adquirir el conjunto0 y \\ descomponerlo en datos.\end{tabular}                                 \\ \hline
\textbf{Dependencias}                                                                   & - Correcta creación del conjunto0                                                                                                                                              \\ \hline
\textbf{\begin{tabular}[c]{@{}l@{}}Ambiente de \\ prueba/configuración\end{tabular}}    & \begin{tabular}[c]{@{}l@{}}- Hardware: Equipo de computo\\ (preferentemente portatíl)\\ - Software: Compilador python3, \\ IDE y/o editor de texto.\end{tabular}               \\ \hline
\textbf{Inicialización}                                                                 & \begin{tabular}[c]{@{}l@{}}- Codificación correspondiente al requerimiento.\\ - Carga  del archivo base.\end{tabular}                                                          \\ \hline
\textbf{Finalización}                                                                   & N/A                                                                                                                                                                            \\ \hline
\textbf{Acciones}                                                                       & \begin{tabular}[c]{@{}l@{}}- Compilar el código correspondiente.\\ - Contar con el archivo conjunto0\\  creado.\end{tabular}                                                   \\ \hline
\textbf{\begin{tabular}[c]{@{}l@{}}Descripción de los \\ datos de entrada\end{tabular}} & \begin{tabular}[c]{@{}l@{}}- Directorio del conjunto0\\ - Adquisición de dicho archivo.\end{tabular}                                                                           \\ \hline
\textbf{Salida esperada}                                                                & \begin{tabular}[c]{@{}l@{}}- Notificación de correcta lectura del \\ conjunto0\\ - El sistema informa que adquirió correctamente \\ la información del conjunto0.\end{tabular} \\ \hline
\textbf{Salida obtenida}                                                                &   \begin{tabular}[c]{@{}l@{}}
- Se realizaron cambios, donde este proceso \\
se lleva a acabo se ejecute mientras la pantalla\\
de “Analizando Datos”, sin embargo en la terminal\\
se muestra el manejo de datos de los compuestos\\
realizado.
\end{tabular}                                                                                                                                                                             \\ \hline
\textbf{Resultado}                                                                      &      \begin{tabular}[c]{@{}l@{}}
- Se visualiza a través de la pantalla de\\
“Analizando Datos”, no se requiere notificar\\
de la adquisición de la información ,  el sistema\\
se asegura de disponer con la información necesaria. 
\end{tabular}                                                                                                                                                                           \\ \hline
\textbf{Severidad}                                                                      &     Media                                                                                                                                                                           \\ \hline
\textbf{Evidencia}                                                                      &    Evidencia en la imagen \ref{CPRF9-10}                                                                                                                                                                            \\ \hline
\textbf{Estado}                                                                         & Iniciado.                                                                                                                                                                   \\ \hline
\end{longtable}