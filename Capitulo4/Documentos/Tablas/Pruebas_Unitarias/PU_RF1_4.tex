% Please add the following required packages to your document preamble:
% \usepackage{longtable}
% Note: It may be necessary to compile the document several times to get a multi-page table to line up properly
\begin{longtable}{|l|l|}
\caption{Prueba unitaria RF1.4}
\label{PU_RF1_4}\\
\hline
\textbf{Requerimiento Funcional}                                                       & \textbf{\begin{tabular}[c]{@{}l@{}}RF1.4 Búsqueda de los mecanismos de \\acción de los compuestos.\end{tabular}}                                                                                                                                                                                                                                                                                                                                                                                                                                                                                                                                                                                                                                                                                                         \\ \hline
\endfirsthead
%
\multicolumn{2}{c}%
{{\bfseries Tabla \thetable\ Continuación de la página anterior}} \\
\endhead
%
\textbf{Perfiles Implicados}                                                           & \begin{tabular}[c]{@{}l@{}}- Desarrollador.\\ - Tester.\end{tabular}                                                                                                                                                                                                                                                                                                                                                                                                                                                                                                                                                                                                                                                                                                                                                        \\ \hline
\textbf{Planificación temporal}                                                        & \begin{tabular}[c]{@{}l@{}}1. Prueba de compilación.\\ 2. Prueba de funcionamiento.\\ 2.1 Verificar que el sistema mantiene el nombre \\ del compuesto sobre del que están obteniendo \\ los datos.\\ 2.2 Comprobar estado de la conexión a la base \\ de datos “DrugBank”.\\ 2.3 Comprobar que se mantiene la dirección\\ correspondiente  a el compuesto sobre el que \\ se está trabajando.\\ 2.4 Adquisición de la representación cuantitativa  \\ de la actividad biológica del compuesto en la \\ bases de datos “DrugBank”. \\ 2.5 Creación del archivo “BioActivity”.\\ 2.6 Almacenado de los datos adquiridos en el \\ archivo correspondiente , recientemente creado.\end{tabular}                                                                                                                                \\ \hline
\textbf{Criterio de verificación}                                                      & \begin{tabular}[c]{@{}l@{}}1. Compilación del código.\\ 2. Ejecutable del módulo, hay muestras de que el \\ código hace algo\\ (algo: definido por los siguientes puntos).\\ 2.1Verificación de integridad de la variable que \\ contiene el nombre del compuesto.\\ 2.2 Conexión a  “DrugBank” y por lo tanto a \\ Internet.\\ 2.3 Adecuada búsqueda del compuesto exacto \\ en la base de datos.\\ 2.4 Correcta adquisición de las cantidades \\ descriptivas de la actividad biológica en \\ “DrugBank”.\\ 2.5 Creación del archivo “BioActivity” \\ correspondiente con el compuesto.\\ 2.6 Corroborar guardado de la información \\ obtenida en el archivo creado\end{tabular}                                                                                                                                       \\ \hline
\textbf{Criterio de aceptación}                                                        & \begin{tabular}[c]{@{}l@{}}1. No hay errores que impidan la compilación \\ del código.\\ 2. Al usar el ejecutable del módulo, hay \\ muestras de que el código hace algo\\ (algo: definido por los siguientes puntos).\\ 2.1 No se distorsiona el nombre del compuesto \\ de interés.\\ 2.2 El sistema informa la correcta conexión a \\ “DrugBank”.\\ 2.3. Se notifica la adecuada obtención de los \\ datos cuantitativos de la actividad biológica \\ del compuesto.\\ 2.4 El sistema informa la correcta creación \\ del archivo.\\ 2.5 Son guardados todos los datos obtenidos \\ de DrugBank en el archivo creado.\end{tabular}                                                                                                                                                                                       \\ \hline
\textbf{\begin{tabular}[c]{@{}l@{}}Definición de\\ verificaciones\end{tabular}}        & \begin{tabular}[c]{@{}l@{}}- Errores de Compilación: Ocurren porque \\ la sintaxis del lenguaje no es correcta, de \\ cajón este tipo de errores no permiten que \\ la aplicación se ejecute. \\ \\ - Conexión a base de datos online: Una conexión \\ a base de datos es un archivo de configuración \\ donde se especifica los detalles físicos de una \\ base de datos como por ejemplo el tipo de \\ base de datos y la versión, y los parámetros que \\ permiten una conexión.\\ \\ - Manejo de Archivos: Un programa no puede \\ manipular los datos de un archivo directamente. \\ Para usar un archivo, un programa siempre \\ abrir el archivo y asignarlo a una variable, que \\ llamaremos el archivo lógico. Todas las \\ operaciones sobre un archivo se realizan a través \\ del archivo lógico.\end{tabular} \\ \hline
\textbf{\begin{tabular}[c]{@{}l@{}}Análisis y evaluación\\ de resultados\end{tabular}} & - Resultados:                                                                                                                                                                                                                                                                                                                                                                                                                                                                                                                                                                                                                                                                                                                                                                                                               \\ \hline
\textbf{Productos  a entregar}                                                         & \begin{tabular}[c]{@{}l@{}}-Búsqueda de la actividad biológica del \\ compuesto de interés funcionando correctamente.\end{tabular}                                                                                                                                                                                                                                                                                                                                                                                                                                                                                                                                                                                                                                                                                          \\ \hline
\end{longtable}