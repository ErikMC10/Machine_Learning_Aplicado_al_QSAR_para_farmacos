% Please add the following required packages to your document preamble:
% \usepackage{longtable}
% Note: It may be necessary to compile the document several times to get a multi-page table to line up properly
\begin{longtable}{|l|l|}
\caption{Prueba unitaria RF1.6}
\label{PU_RF1_6}\\
\hline
\textbf{Requerimiento Funcional}                                                       & \textbf{\begin{tabular}[c]{@{}l@{}}RF1.6  Confirmación de los resultados de la\\ búsqueda.\end{tabular}}                                                                                                                                                                                                                                                                                                                                                                                                                                                                                                                                                                                                                                                                                                                                                                                                                                                                                                               \\ \hline
\endfirsthead
%
\multicolumn{2}{c}%
{{\bfseries Tabla \thetable\ Continuación de la página anterior}} \\
\endhead
%
\textbf{Perfiles Implicados}                                                           & \begin{tabular}[c]{@{}l@{}}- Desarrollador.\\ - Tester.\end{tabular}                                                                                                                                                                                                                                                                                                                                                                                                                                                                                                                                                                                                                                                                                                                                                                                                                                                                                                                                                   \\ \hline
\textbf{Planificación temporal}                                                        & \begin{tabular}[c]{@{}l@{}}1. Prueba de compilación.\\ 2. Prueba de funcionamiento.\\ 2.1 Verificar despliegue de la interfaz que \\ informa el fin de adquisición de datos.\\ 2.2 Chequeo en que el sistema contemple los \\ directorios donde se encuentran cada uno de los \\ archivos creados.\\ 2.2.1 Existencia del directorio y de los archivos\\ correspondientes a la estructura molecular del \\ compuesto.\\ 2.2.2 Existencia del directorio y de los archivos\\ correspondientes a los descriptores del compuesto.\\ 2.2.3 Existencia del directorio y de los archivos\\ correspondientes a la actividad biológica del \\ compuesto.\\ 2.2.4 Existencia del directorio y del archivo \\ correspondiente a la(s) proteina(s)\\ 2.5 Interfaz que indica estado de cada uno de los \\ archivos.\\ 2.5.1 Se indica si la estructura del archivo es \\ correcta con el contenido adecuado.\\ 2.5.2 Se notifica si existe un error en el archivo,\\ especificando este, así como la posible causa.\end{tabular} \\ \hline
\textbf{Criterio de verificación}                                                      & \begin{tabular}[c]{@{}l@{}}1. Compilación del código.\\ 2. Ejecutable del módulo, hay muestras de que \\ el código hace algo\\ (algo: definido por los siguientes puntos).\\ 2.1 Verificación de integridad en los archivos \\ creados para cada uno de los datos perteneciente \\ a los compuestos.\\ 2.2 Rectificar la cantidad de archivos creados \\ para cada compuesto, y que estos coinciden con \\ los estimados.\\ 2.3 Verificación de la creación y estructura \\ adecuada del archivo que contiene la estructura \\ de cada una de la(s) proteínas.\\ 2.4 Chequeo a las pantallas que muestran los \\ resultados para la adquisición de información.\\ 2.5 Aparición en el momento indicado de \\ mensajes que indiquen algún error, falla o \\ información que requiera saber el usuario.\\ 2.4 Creación del archivo “\\ Descriptors” correspondiente con el compuesto.\\ 2.5 Corroborar guardado de la información \\ obtenida en el archivo creado.\end{tabular}                                      \\ \hline
\textbf{Criterio de aceptación}                                                        & \begin{tabular}[c]{@{}l@{}}1. No hay errores que impidan la compilación \\ del código.\\ 2. Al usar el ejecutable del módulo, hay muestras \\ de que el código hace algo\\ (algo: definido por los siguientes puntos).\\ 2.1 No se distorsiona el nombre del compuesto \\ de interés.\\ 2.2 El sistema informa la correcta conexión a\\ “ChemSpider”.\\ 2.3 El sistema informa la correcta conexión a\\ “PubChem”.\\ 2.4 El sistema informa que el mismo compuesto \\ existe en ambos repositorios.\\ 2.5. Se notifica la adecuada obtención de los \\ descriptores del compuesto.\\ 2.6 El sistema informa la correcta creación \\ del archivo.\\ 2.7 El archivo y la información que contiene \\ es integra.\end{tabular}                                                                                                                                                                                                                                                                                            \\ \hline
\textbf{\begin{tabular}[c]{@{}l@{}}Definición de\\ verificaciones\end{tabular}}        & \begin{tabular}[c]{@{}l@{}}- Errores de Compilación: Ocurren porque la \\ sintaxis del lenguaje no es correcta, de cajón \\ este tipo de errores no permiten que la \\ aplicación se ejecute. \\ \\ - Acción de botón: representa un botón que, \\ cuando es presionado, envía información al que \\ pertenece. La función de  un botón representada  \\ el contenido del elemento.\\ \\ - Visualización de pantalla interfaz de usuario.\\ \\  - Manejo de Archivos: Un programa no puede \\ manipular los datos de un archivo directamente. \\ Para usar un archivo, un programa siempre abrir \\ el archivo y asignarlo a una variable, que \\ llamaremos el archivo lógico. Todas las\\ operaciones sobre un archivo se realizan \\ a través del archivo lógico.\end{tabular}                                                                                                                                                                                                                                      \\ \hline
\textbf{\begin{tabular}[c]{@{}l@{}}Análisis y evaluación\\ de resultados\end{tabular}} & - Resultados:                                                                                                                                                                                                                                                                                                                                                                                                                                                                                                                                                                                                                                                                                                                                                                                                                                                                                                                                                                                                          \\ \hline
\textbf{Productos  a entregar}                                                         & \begin{tabular}[c]{@{}l@{}}- Confirmación de los resultados de la búsqueda \\ funcionando correctamente.\end{tabular}                                                                                                                                                                                                                                                                                                                                                                                                                                                                                                                                                                                                                                                                                                                                                                                                                                                                                                  \\ \hline
\end{longtable}