% Please add the following required packages to your document preamble:
% \usepackage{longtable}
% Note: It may be necessary to compile the document several times to get a multi-page table to line up properly
\begin{longtable}{|l|l|}
\caption{Prueba unitaria RF2.2}
\label{PU_RF_2_2}\\
\hline
\textbf{Requerimiento Funcional}                                                        & \textbf{\begin{tabular}[c]{@{}l@{}}RF 2.2 Adquisición y descomposición \\ del conjuntoP.\end{tabular}}                                                                                                                                                                                                                                                                                                                                                                                                                                                                                                                                                                                                                                                                                                                                                                                                                   \\ \hline
\endfirsthead
%
\multicolumn{2}{c}%
{{\bfseries Tabla \thetable\ Continuación de la página anterior}} \\
\endhead
%
\textbf{Perfiles Implicados}                                                            & \begin{tabular}[c]{@{}l@{}}- Desarrollador.\\ - Tester.\end{tabular}                                                                                                                                                                                                                                                                                                                                                                                                                                                                                                                                                                                                                                                                                                                                                                                                                                                     \\ \hline
\textbf{Planificación temporal}                                                         & \begin{tabular}[c]{@{}l@{}}1. Prueba de compilación.\\ 2. Prueba de funcionamiento.\\ 2.1 Verificar que el sistema reconoce el \\ conjuntoP\\  y puede acceder a este.\\ 2.2 Comprobar lectura adecuada del \\ conjuntoP.\\ 2.3 Adquisición de información por medio del \\ contenido de conjuntoP.\\ 2.4 Reconocer que la información se adquiere \\ por iteración, y pertenece únicamente a un \\ compuesto. \\ 2.5. Adquisición de los datos y \\ transferencia a variables.\end{tabular}                                                                                                                                                                                                                                                                                                                                                                                                                             \\ \hline
\textbf{Criterio de verificación}                                                       & \begin{tabular}[c]{@{}l@{}}- Compilación del código.\\ - Ejecutable del módulo, hay muestras de que \\ el código hace algo\\ (algo: definido por los siguientes puntos).\\ 2.1 Verificación de acceso a directorio y archivo \\ conjuntoP.\\ 2.2 Comprobar lectura adecuada del \\ conjuntoP.\\ 2.3 Revisión de la creación de variables para \\ manejo de los datos de compuestos por parte \\ del sistema.\\ 2.3.1 Verificación de integridad y adecuado \\ valor de las  variables creadas.\end{tabular}                                                                                                                                                                                                                                                                                                                                                                                                              \\ \hline
\textbf{Criterio de aceptación}                                                         & \begin{tabular}[c]{@{}l@{}}1. No hay errores que impidan la compilación \\ del código.\\ 2. Al usar el ejecutable del módulo, hay muestras \\ de que el código hace algo\\ (algo: definido por los siguientes puntos).\\ 2.1 Se encuentra fácilmente el archivo conjuntoP\\  y el sistema informa al usuario que se obtuvo \\ adecuadamente, o en caso contrario, muestra la \\ causa de error.\\ 2.2 El sistema puede leer el archivo conjuntoP.\\ 2.3 El sistema adquiere y guarda en variables la \\ información contenida en el conjuntoP.\\ 2.4 Se comprueba que el manejo de la variables \\ que contienen información pueden ser usadas \\ fácilmente.\end{tabular}                                                                                                                                                                                                                                               \\ \hline
\textbf{Definición de verificaciones}                                                   & \begin{tabular}[c]{@{}l@{}}- Errores de Compilación: Ocurren porque la \\ sintaxis del lenguaje no es correcta, de cajón \\ este tipo de errores no permiten que la \\ aplicación se ejecute. \\ - Variables de programación:la variable está \\ formada por un espacio en el sistema de \\ almacenaje (memoria principal de un ordenador) \\ y un nombre simbólico (un identificador) que \\ está asociado a dicho espacio. Ese espacio \\ contiene una cantidad de información \\ conocida o desconocida, es decir un valor.\\ - Visualización de pantalla interfaz de usuario.\\  - Manejo de Archivos: Un programa no \\ puede manipular los datos de un archivo \\ directamente. Para usar un archivo, un\\ programa siempre abrir el archivo y \\ asignarlo a una variable, que llamaremos \\ el archivo lógico. Todas las operaciones sobre \\ un archivo se realizan a través del\\ archivo lógico.\end{tabular}                                                                                                                                                                                                                                                                                                                                                                                                                                                                                                                                                                                                                                                                      \\ \hline
\textbf{\begin{tabular}[c]{@{}l@{}}Análisis y \\ evaluación de resultados\end{tabular}} & - Resultados:                                                                                                                                                                                                                                                                                                                                                                                                                                                                                                                                                                                                                                                                                                                                                                                                                                                                                                            \\ \hline
\textbf{Productos  a entregar}                                                          & \begin{tabular}[c]{@{}l@{}}-Adquisición y descomposición del \\ conjuntoP funcionando correctamente.\end{tabular}                                                                                                                                                                                                                                                                                                                                                                                                                                                                                                                                                                                                                                                                                                                                                                                                        \\ \hline
\end{longtable}