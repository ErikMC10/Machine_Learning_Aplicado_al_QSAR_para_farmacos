% Please add the following required packages to your document preamble:
% \usepackage{longtable}
% Note: It may be necessary to compile the document several times to get a multi-page table to line up properly
\begin{longtable}{|l|l|}
\caption{Prueba unitaria RF1.2}
\label{PU_RF1_2}\\
\hline
\textbf{Requerimiento Funcional}                                                       & \textbf{RF1.2 Búsqueda de los compuestos indicados..}                                                                                                                                                                                                                                                                                                                                                                                                                                                                                                                                  \\ \hline
\endfirsthead
%
\multicolumn{2}{c}%
{{\bfseries Table \thetable\ Continuación de la página anterior}} \\
\endhead
%
\textbf{Perfiles Implicados}                                                           & \begin{tabular}[c]{@{}l@{}}- Desarrollador.\\ - Tester.\end{tabular}                                                                                                                                                                                                                                                                                                                                                                                                                                                                                                                   \\ \hline
\textbf{Planificación temporal}                                                        & \begin{tabular}[c]{@{}l@{}}1. Prueba de compilación.\\ 2. Prueba de funcionamiento.\\ 2.1 Verificar adecuada lectura del archivo base\\ (token “compunds” ).\\ 2.2 Comprobar conexión a la base de datos \\ “DrugBank”.\\ 2.3 Obtención del .pdb de cada uno de los \\ compuestos enlistados.\\ 2.4 Almacenado del .pdb en el sistema.\end{tabular}                                                                                                                                                                                                                                   \\ \hline
\textbf{Criterio de verificación}                                                      & \begin{tabular}[c]{@{}l@{}}1. Compilación del código.\\ 2. Ejecutable del módulo, hay muestras de que el \\ código hace algo\\ (algo: definido por los siguientes puntos).\\ 2.1 Lectura del token que describe los nombres del\\ compuesto en el archivo base.\\ 2.2 Conexión a  DrugBank y por lo tanto a Internet.\\ 2.3 Adquirir el archivo .pdb del compuesto correcto.\\ 2.4 Almacenamiento de un archivo íntegro en el \\ sistema.\end{tabular}                                                                                                                                \\ \hline
\textbf{Criterio de aceptación}                                                        & \begin{tabular}[c]{@{}l@{}}1. No hay errores que impidan la compilación del \\ código.\\ 2. Al usar el ejecutable del módulo, hay muestras \\ de que el código hace algo\\ (algo: definido por los siguientes puntos).\\ 2.1 No se pierde ningún valor ni dato existente en el\\ archivo base.\\ 2.2 El sistema informa la correcta conexión a \\ “DrugBank”.\\ 2.3 El sistema informa que el compuesto existe \\ en la base de datos y confirma la adquisición \\ del .pdb.\\ 2.4 El sistema notifica la correcta obtención del \\ archivo .pdb y correcto  almacenado.\end{tabular} \\ \hline
\textbf{\begin{tabular}[c]{@{}l@{}}Definición de\\ verificaciones\end{tabular}}        & \begin{tabular}[c]{@{}l@{}}- Errores de Compilación: Ocurren porque la sintaxis \\ del lenguaje no es correcta, de cajón este tipo de \\ errores no permiten que la aplicación se ejecute. \\ \\ - Conexión a base de datos online:Una conexión a \\ base de datos es un archivo de configuración donde \\ se especifica los detalles físicos de una base de datos \\ como por ejemplo el tipo de base de datos y la \\ versión, y los parámetros que permiten una conexión\end{tabular}                                                                                               \\ \hline
\textbf{\begin{tabular}[c]{@{}l@{}}Análisis y evaluación\\ de resultados\end{tabular}} &                                                                                                                                                                                                                                                                                                                                                                                                                                                                                                                                                                                        \\ \hline
\textbf{Productos  a entregar}                                                         & \begin{tabular}[c]{@{}l@{}}- Búsqueda de los compuestos indicados \\ funcionando correctamente.\end{tabular}                                                                                                                                                                                                                                                                                                                                                                                                                                                                           \\ \hline
\end{longtable}