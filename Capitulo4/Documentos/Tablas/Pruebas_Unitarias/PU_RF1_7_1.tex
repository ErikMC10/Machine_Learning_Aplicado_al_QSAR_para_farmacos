% Please add the following required packages to your document preamble:
% \usepackage{longtable}
% Note: It may be necessary to compile the document several times to get a multi-page table to line up properly
\begin{longtable}{|l|l|}
\caption{Prueba unitaria RF1.7.1}
\label{PU_RF1_7_1}\\
\hline
\textbf{Requerimiento Funcional}                                                       & \textbf{RF1.7.1 Construcción del conjunto cero.}                                                                                                                                                                                                                                                                                                                                                                                                                                                                                                                                                                                                                                                                                                                                                                                            \\ \hline
\endfirsthead
%
\multicolumn{2}{c}%
{{\bfseries Tabla \thetable\ Continuación de la página anterior}} \\
\endhead
%
\textbf{Perfiles Implicados}                                                           & \begin{tabular}[c]{@{}l@{}}- Desarrollador.\\ - Tester.\end{tabular}                                                                                                                                                                                                                                                                                                                                                                                                                                                                                                                                                                                                                                                                                                                                                                        \\ \hline
\textbf{Planificación temporal}                                                        & \begin{tabular}[c]{@{}l@{}}1. Prueba de compilación.\\ 2. Prueba de funcionamiento.\\ 2.1 Verificar que no existió errores en la \\ obtención de información.\\ 2.2 Comprobar adecuada estructura de cada \\ uno de los archivos que conforman el conjunto 0.\\ 2.3 Chequeo de la correcta apertura  y lectura \\ de cada uno de los archivos creados por proteína \\ del listado.\\ 2.4 Creación del archivo que se define como \\ conjunto0, que contendrá toda la información \\ respecto a descriptores.\\ 2.5  Verificación por cada iteración el cambio \\ al archivo correspondiente para agregar la \\ información necesaria, especialmente que no \\ existe pérdida de información.\end{tabular}                                                                                                                                 \\ \hline
\textbf{Criterio de verificación}                                                      & \begin{tabular}[c]{@{}l@{}}1. Compilación del código.\\ 2. Ejecutable del módulo, hay muestras de que \\ el código hace algo\\ (algo: definido por los siguientes puntos).\\ 2.1 Verificación de errores en la adquisición de \\ datos de cada uno de  los compuestos.\\ 2.2 Revisión de la estructura en cada uno de \\ los archivos pertenecientes a cada compuesto de \\ la lista (Archivo base).\\ 2.3 Lectura de los archivos de los cuales se \\ extraerán los datos que contiene el conjunto0.\\ 2.4 Creación del archivo conjunto0.\\ 2.5 Modificación por cada iteración del archivo\\ conjunto0, en el que se agrega datos de cada \\ compuesto.\end{tabular}                                                                                                                                                                    \\ \hline
\textbf{Criterio de aceptación}                                                        & \begin{tabular}[c]{@{}l@{}}1. No hay errores que impidan la compilación \\ del código.\\ 2. Al usar el ejecutable del módulo, hay muestras \\ de que el código hace algo\\ (algo: definido por los siguientes puntos).\\ 2.1El sistema notifica la adquisición correcta o \\ incorrecta la adquisición completa de los datos \\ referentes a los compuestos.\\ 2.2El sistema muestra mensajes de error, en \\ caso de presentarse alguna falla en los archivos \\ contenedores de datos.\\ 2.3 El sistema accede y lee adecuadamente cada \\ uno de los archivos.\\ 2.4 El sistema informa adecuada creación del \\ conjunto 0 o muestra el error en caso de existir.\\ 2.5. Informa errores en el caso de existir al \\ crear el contenido del conjunto 0.\\ 2.6 El sistema informa la correcta estructura \\ del conjunto0\end{tabular} \\ \hline
\textbf{\begin{tabular}[c]{@{}l@{}}Definición de\\ verificaciones\end{tabular}}        & \begin{tabular}[c]{@{}l@{}}- Errores de Compilación: Ocurren porque la \\ sintaxis del lenguaje no es correcta, de cajón \\ este tipo de errores no permiten que la \\ aplicación se ejecute. \\ \\ - Visualización de pantalla interfaz de usuario.\\ \\  - Manejo de Archivos: Un programa no puede \\ manipular los datos de un archivo directamente. \\ Para usar un archivo, un programa siempre abrir \\ el archivo y asignarlo a una variable, que \\ llamaremos el archivo lógico. Todas las\\ operaciones sobre un archivo se realizan \\ a través del archivo lógico.\end{tabular}                                                                                                                                                                                                                                                \\ \hline
\textbf{\begin{tabular}[c]{@{}l@{}}Análisis y evaluación\\ de resultados\end{tabular}} & - Resultados:                                                                                                                                                                                                                                                                                                                                                                                                                                                                                                                                                                                                                                                                                                                                                                                                                               \\ \hline
\textbf{Productos  a entregar}                                                         & \begin{tabular}[c]{@{}l@{}}- Búsqueda de los compuestos indicados \\ funcionando correctamente.\end{tabular}                                                                                                                                                                                                                                                                                                                                                                                                                                                                                                                                                                                                                                                                                                                                \\ \hline
\end{longtable}