%\rhead{}
%\lhead{}
\renewcommand{\headrulewidth}{0pt}
\addcontentsline{toc}{chapter}{\textbf{Introducción.}} 
\begin{center}
    \Large
    \textbf{\newtitle}
    
    \large
    \vspace{0.4cm}
    \newauthor
    \vspace{0.9cm}
    
    \textbf{Introducción}\\
\end{center}
\noindent Los animales son ampliamente utilizados en la experimentación científica para el desarrollo de medicinas y para probar la seguridad de otros productos. Dados los recientes avances en la Bioinformática y los algoritmos que realizan predicciones como los pertenecientes al área de Machine Learning, el proceso de la experimentación tiene una visión diferente en cuanto a la forma en que se realiza.\\

\noindent SisPAF es un sistema que permite obtener una predicción del comportamiento de los fármacos en términos de capacidad para acoplarse a las proteínas de una patología y evitar que lleven a cabo su función (también denominada aptitud). En pocas palabras, SisPAF permite determinar qué fármacos tienen más posibilidades descomponer las proteínas base de  una enfermedad. Esto implementando técnicas de Bioinformática y optimizando su funcionamiento con algoritmos de Machine Learning.
El objetivo principal de SisPAF es organizar un conjunto predefinido de fármacos del más al menos recomendable para combatir cierta patología.\\

\noindent La experimentación, dentro de la industria farmacéutica, es una parte indispensable para el desarrollo de nuevos medicamentos que contribuyen a mejorar la salud humana. Por desgracia, este proceso implica el uso de animales como “sujetos de prueba”. El proceso de pruebas en seres vivos es, por ahora, imposible de sustituir con las tecnologías actuales, aunque sí puede ser ampliamente optimizado, dando oportunidad a que la tecnología ofrezca alternativas que reduzcan la cantidad de seres vivos que se requieren para realizar este tipo de investigaciones.