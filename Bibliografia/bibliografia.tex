\addcontentsline{toc}{chapter}{\textbf{Bibliografía.}} 
\lhead{Bibliografía}
\rhead{\newtitle}
\cfoot{\thepage}
\renewcommand{\headrulewidth}{1pt}
\renewcommand{\footrulewidth}{1pt}
%\chapter*{Bibliografía}
\begin{thebibliography}{99}
\bibitem{anex_1}H. Kubinyi, «2D QSAR models: Hansch and Free-Wilson analyses» Comput. Med. Chem. Drug. Discov, vol. 5, pp. 539-570, 2004.

\bibitem{anex_2}OpenIntro Statistics: Third Edition, David M Diez, Christopher D Barr, Mine Çetinkaya-Rundel An Introduction to Statistical Learning: with Applications in R (Springer Texts in Statistics) Linear Models with R, Julian J.Faraway

\bibitem{1}“About Animal Testing,”. \textit{Humane Society International}, Mayo 6, 2019. [En línea]. Disponible en: https://www.hsi.org/news-media/about/. [Visitado: Octubre 13, 2019].
\bibitem{2}JP Ebejer. "Using computers to discover new medicines", \textit{Times of Malta}, Octubre 5, 2019. [En línea]. Disponible en: Times of Malta, https://timesofmalta.com/. [Visitado: Octubre 12, 2019].
\bibitem{3}N. Luscombe, D. Greenbaum, M.Gerstein, "What is bioinformatics? A proposed definition and overview of the field.", \textit{Methods of information in medicine}, vol. 40 4, pp. 346-58, 2001.
\bibitem{4}S. Brogi, T. Castro, J. L. Medina, K. Kuca, "In Silico Methods for Drug Design and Discovery" \textit{Frontiers}. [En línea]. Disponible en: https://www.frontiersin.org/research-topics/10032/in-silico-methods-for-drug-design-and-discovery#overview. [Visitado: Agosto 15, 2019].

\bibitem{5}ExpoMed, “EL MODELO IN SILICO: LA EXPERIMENTACIÓN VIRTUAL”, 26 Marzo, 2019. [En línea].
Disponible en: https://expomed.com.mx/en/node/344. [Visitado: 10 Oct, 2019].

\bibitem{6}A. Wadood, N. Ahmed, L. Shah, A. Ahmad, H. Hassan and S. Shams, "In-silico drug design: An approach which revolutionarised the drug discovery process" . \textit{Open Access Publishing London (UK)}, Sep 2013. [En línea]. Disponible en: http://www.oapublishinglondon.com/article/1119. [Visitado: Agosto 11, 2019].

\bibitem{7}J. Lozano-Aponte. "¿Qué sabe Ud. acerca de…QSAR?", \textit{Revista Mexicana de Ciencias Farmacéuticas}, vol. 43, no. 2, pp. 1-2, 2012. [En línea]. Disponible en: ScIELO, http://www.scielo.org.mx/scielo.php?script=sci\_serial&pid=1870-0195&lng=es&nrm=iso. [Visitado: Agosto 11, 2019]

\bibitem{8}M. Usselman et al, "Chemical compound" en Encyclopædia Britannica. Encyclopædia Britannica, inc., [Documento en línea], 2016. Disponible en: Encyclopædia Britannica, https://www.britannica.com/science/chemical-compound [Visitado: Octubre 11, 2019].

\bibitem{9}C. Nantasenamat, C. Isarankura-Na-Ayudhya, T. Naenna, V. Prachayasittikul. "A PRACTICAL OVERVIEW OF
QUANTITATIVE STRUCTURE-ACTIVITY RELATIONSHIP". \textit{EXCLI}, Vol. 8, pp. 3+, Abril 2009. [En línea]. Disponible en: https://www.excli.de/vol8/Prachayasittikul\_04\_2009/Prachayasittikul\_050509\_proof.pdf. [Visitado: Octubre 11, 2019].

\bibitem{10}C. M. Bishop (2006). Pattern Recognition and Machine
Learning. Springer. ISBN 0-387-31073-8

\bibitem{11}Michie, D.; Spiegelhalter, D. J.; Taylor, C. C. (1994). Machine Learning, Neural and Statistical Classification. Ellis Horwood.

\bibitem{12}Lu, Haiping; Plataniotis, K.N.; Venetsanopoulos, A.N.(2011). “A Survey of Multilinear Subspace Learning for Tensor Data” (PDF). Pattern Recognition 44

%%%%%%%%%%%%%%%%%%%%%%%%%%%%%%%%%%%%
\bibitem{13}S. Savale, "In silico softwares", Slideshare.net, 2017. [En línea]. Disponible en: https://www.slideshare.net/sagarsavale1/in-silico-softwares. [Visitado: Agosto 12, 2019].

\bibitem{15}Lengauer T. Computational methods for biomolecular docking. Current Opinion in Structural Biology. 1996 6;6(3):402-406. PMID 8804827

\bibitem{16}Pascual, C., 2020. Tutorial: Understanding Linear Regression And Regression Error Metrics. [En línea] Dataquest. Disponible en: https://www.dataquest.io/blog/understanding-regression-error-metrics/#:~:text=Mean\%20absolute\%20error,residuals\%20do\%20not\%20cancel\%20out.


\end{thebibliography}