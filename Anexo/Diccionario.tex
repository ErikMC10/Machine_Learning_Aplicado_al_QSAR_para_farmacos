\begin{longtable}{|l|l|l|l|l|}
\caption{Diccionario de datos}\\ 
\hline
\label{diccionario}
\textbf{ID}  & \begin{tabular}[c]{@{}l@{}}\textbf{Nombre }\\\textbf{ completo} \end{tabular} & \textbf{Descripción}                                                                                                                                                                                                                                                                                                                                                                                                                                                                                                                                                                                                                                                                                                                                                                                                                                                                                                                        & \textbf{Tipo}                                                       & \begin{tabular}[c]{@{}l@{}}\textbf{Ubicación}\\\textbf{ (módulo o }\\\textbf{ parte del }\\\textbf{sistema }\\\textbf{donde es }\\\textbf{usada)} \end{tabular}  \endfirsthead 
\hline
C0           & Conjunto0                                                                     & \begin{tabular}[c]{@{}l@{}}Al conjunto cero lo \\conforman todos los \\archivos correspondientes \\al resultado de la \\búsqueda de información \\de los compuestos. \\Estos archivos se \\almacenan en un directorio \\que lleva por nombre \\Compounds y es creada \\por SisPAF al comienzo de \\la búsqueda.\textasciitilde{}La figura . \\ilustra el contenido de este \\directorio cuando se abre \\en el explorador de archivos \\del sistema operativo Linux \\(en su distribución Ubuntu). \end{tabular}                                                                                                                                                                                                                                                                                                                                                                                                                            & \begin{tabular}[c]{@{}l@{}}Archivo en \\ formato pdb. \end{tabular} & \begin{tabular}[c]{@{}l@{}}Búsqueda de \\ información. \end{tabular}                                                                                             \\ 
\hline
RM           & REMARK                                                                        & \begin{tabular}[c]{@{}l@{}}Etiqueta correspondiente \\ al campo de comentarios \\ y contiene el nombre del \\ compuesto. \end{tabular}                                                                                                                                                                                                                                                                                                                                                                                                                                                                                                                                                                                                                                                                                                                                                                                                      & Texto (50)                                                          & \begin{tabular}[c]{@{}l@{}}- Archivo \\ Conjunto0\\ - Archivo \\ ConjuntoP \end{tabular}                                                                         \\ 
\hline
HD           & HEADER                                                                        & \begin{tabular}[c]{@{}l@{}}El registro HEADER \\ identifica de forma \\ exclusiva una entrada \\ PDB a través del campo \\ idCode. Este registro \\ también proporciona \\ una clasificación para la \\ entrada. Finalmente, \\ contiene la fecha en que \\ las coordenadas se \\ depositaron en el \\ archivo PDB. \end{tabular}                                                                                                                                                                                                                                                                                                                                                                                                                                                                                                                                                                                                           & Texto (66)                                                          & \begin{tabular}[c]{@{}l@{}}- Archivo \\ Conjunto0 \end{tabular}                                                                                                  \\ 
\hline
HT           & HETATM                                                                        & \begin{tabular}[c]{@{}l@{}}Las coordenadas \\ químicas no poliméricas \\ u otras "no estándar \\ como las moléculas de \\ agua o los átomos \\ presentados en los grupos \\ HET utilizan el tipo de \\ registro HETATM. \\ También presentan el \\ factor de ocupación y \\ temperatura para cada \\ átomo. Los registros \\ ATOM presentan las \\ coordenadas atómicas \\ para residuos estándar. \\ El símbolo del elemento \\ siempre está presente en \\ cada registro HETATM. \end{tabular}                                                                                                                                                                                                                                                                                                                                                                                                                                            & Texto (80)                                                          & \begin{tabular}[c]{@{}l@{}}- Archivo \\ Conjunto0 \end{tabular}                                                                                                  \\ 
\hline
CN           & CONECT                                                                        & \begin{tabular}[c]{@{}l@{}}Los registros CONECT \\ especifican la conectividad \\ entre los átomos para los \\ que se suministran las \\ coordenadas. La \\ conectividad se describe \\ utilizando el número de \\ serie del átomo como se \\ muestra en la entrada. \\ Los registros CONECT \\ son obligatorios para los \\ grupos HET \\ (excluyendo el agua) y \\ para otros enlaces no \\ especificados en la tabla \\ de conectividad de \\ residuos estándar. Estos \\ registros se generan \\ automáticamente. \end{tabular}                                                                                                                                                                                                                                                                                                                                                                                                         & Texto (31)                                                          & \begin{tabular}[c]{@{}l@{}}- Archivo \\ Conjunto0 \end{tabular}                                                                                                  \\ 
\hline
MA           & MASTER                                                                        & \begin{tabular}[c]{@{}l@{}}El registro MASTER es \\ un registro de control \\ para la contabilidad. \\ Enumera el número de \\ líneas en la entrada de \\ coordenadas o archivo \\ para los tipos de registro \\ seleccionados. MASTER \\ registra solo el primer \\ modelo cuando hay \\ varios modelos en las \\ coordenadas. \end{tabular}                                                                                                                                                                                                                                                                                                                                                                                                                                                                                                                                                                                               & Texto (70)                                                          & \begin{tabular}[c]{@{}l@{}}- Archivo\\ Conjunto0 \end{tabular}                                                                                                   \\ 
\hline
EN           & END                                                                           & \begin{tabular}[c]{@{}l@{}}El registro END marca \\ el final del archivo PDB. \end{tabular}                                                                                                                                                                                                                                                                                                                                                                                                                                                                                                                                                                                                                                                                                                                                                                                                                                                 & Texto (6)                                                           & \begin{tabular}[c]{@{}l@{}}- Archivo \\ Conjunto0\\ - Archivo \\ ConjuntoP. \end{tabular}                                                                        \\ 
\hline
AT           & ATOM                                                                          & \begin{tabular}[c]{@{}l@{}}La estructura atómica del \\ compuesto. \end{tabular}                                                                                                                                                                                                                                                                                                                                                                                                                                                                                                                                                                                                                                                                                                                                                                                                                                                            & Texto (80)                                                          & \begin{tabular}[c]{@{}l@{}}- Archivo\\ ConjuntoP \end{tabular}                                                                                                   \\ 
\hline
SQ           & SEQRES                                                                        & \begin{tabular}[c]{@{}l@{}}Los registros SEQRES \\ contienen una lista de los \\ componentes químicos \\ consecutivos unidos \\ covalentemente de forma \\ lineal para formar un \\ polímero. Los componentes \\ químicos incluidos en este \\ listado pueden ser \\ aminoácidos estándar o \\ modificados y residuos \\ de ácido nucleico. \end{tabular}                                                                                                                                                                                                                                                                                                                                                                                                                                                                                                                                                                                   & Texto(80)                                                           & \begin{tabular}[c]{@{}l@{}}- Archivo\\ ConjuntoP \end{tabular}                                                                                                   \\ 
\hline
HE           & HELIX                                                                         & \begin{tabular}[c]{@{}l@{}}Los registros HELIX se \\ utilizan para identificar la \\ posición de las helix en la \\ molécula. Las helix se \\ nombran, enumeran y \\ clasifican por tipo. Se \\ anotan los residuos donde \\ comienza y termina la \\ hélice, así como la longitud \\ total. \end{tabular}                                                                                                                                                                                                                                                                                                                                                                                                                                                                                                                                                                                                                                  & Texto(80)                                                           & \begin{tabular}[c]{@{}l@{}}- Archivo\\ ConjuntoP \end{tabular}                                                                                                   \\ 
\hline
CP           & ConjuntoP                                                                     & \begin{tabular}[c]{@{}l@{}}De manera casi idéntica al \\apartado anterior, el \\conjunto P hace referencia \\a todos los archivos creados \\por SisPAF cuando realiza \\la búsqueda de información \\de las proteínas. Estos \\archivos se almacenan en \\un directorio denominado \\Proteins. La figura \\ilustra lo anterior. \end{tabular}                                                                                                                                                                                                                                                                                                                                                                                                                                                                                                                                                                                               & \begin{tabular}[c]{@{}l@{}}Archivo en \\ formato pdb. \end{tabular} & \begin{tabular}[c]{@{}l@{}}Búsqueda de \\ información. \end{tabular}                                                                                             \\ 
\hline
AI           & \begin{tabular}[c]{@{}l@{}}Archivo \\ inicial \end{tabular}                   & \begin{tabular}[c]{@{}l@{}}El archivo inicial es un archivo \\de texto plano (con extensión \\.txt) que utiliza SisPAF para \\poder iniciar con todo el \\proceso de la búsqueda y el \\análisis de datos.\\Este archivo inicial tiene \\una estructura como la que \\describe la figura \ref{A-1-1}. Como se \\puede observar en dicha figura,\\existen tres elementos que \\se denominan etiquetas y \\sirven para marcar la \\clasificación a la que pertenece \\el conjunto de compuestos \\(escrito como Class en el \\archivo) y el inicio de las dos \\categorías que registra el \\archivo inicial, las cuales son \\compuestos (escrito como \\Compounds en el archivo) y \\proteínas (escrito como \\Proteins en el archivo).\\\end{tabular}                                                                                                                                                                                                 & \begin{tabular}[c]{@{}l@{}}Archivo de \\ texto plano. \end{tabular} & \begin{tabular}[c]{@{}l@{}}Búsqueda de \\ información. \end{tabular}                                                                                             \\ 
\hline
AI           & \begin{tabular}[c]{@{}l@{}}Archivo~\\inicial\end{tabular}                     & \begin{tabular}[c]{@{}l@{}}(Archivo inicial - Continuación)\\La clasificación de los~\\compuestos corresponde a la~\\relación que tienen entre sí,~\\ya sea desde el punto de vista~\\biológico (los clasifica por el~\\cambio anatómico o funcional~\\que inducen) o el punto de~\\vista médico (los clasifica~\\según la patología que tratan).~\\Los compuestos corresponden~\\a todos los fármacos~\\considerados para el estudio~\\de su aptitud mientras que~\\las proteínas representan a~\\los componentes esenciales~\\de una patología en particular.\\La etiqueta Class va sucedida\\del nombre en inglés de la~\\clasificación a la que pertenece~\\el conjunto de compuestos~\\ingresados. Seguidos de las~\\etiquetas de Compounds y~\\Proteins vienen los nombres~\\tanto de los compuestos como~\\de las proteínas (los nombres~\\van escritos en inglés), cada~\\uno debajo de su etiqueta~\\correspondiente.\end{tabular} & \begin{tabular}[c]{@{}l@{}}Archivo de~\\texto plano\end{tabular}    & \begin{tabular}[c]{@{}l@{}}Búsqueda de~\\información.\end{tabular}                                                                                               \\ 
\hline
CF           & \begin{tabular}[c]{@{}l@{}}Conjunto \\ final \end{tabular}                    & \begin{tabular}[c]{@{}l@{}}Documento que contiene \\ una lista de compuestos \\ con sus respectivas \\ aptitudes (capacidad para \\ destruir las proteínas de \\ una patología en particular). \end{tabular}                                                                                                                                                                                                                                                                                                                                                                                                                                                                                                                                                                                                                                                                                                                                & \begin{tabular}[c]{@{}l@{}}Archivo de\\ texto plano \end{tabular}   & \begin{tabular}[c]{@{}l@{}}Administración \\ y presentación \\ de resultados \end{tabular}                                                                       \\
\hline
\end{longtable}
